% !TeX encoding = UTF-8

%% ------------------------------------------------------------------------
%% Copyright (C) 2021-2023 SJTUG
%% 
%% SJTUBeamer Example Document by SJTUG
%% 
%% SJTUBeamer Example Document is licensed under a
%% Creative Commons Attribution-NonCommercial-ShareAlike 4.0 International License.
%% 
%% You should have received a copy of the license along with this
%% work. If not, see <http://creativecommons.org/licenses/by-nc-sa/4.0/>.
%%
%% For a quick start, check out src/doc/sjtubeamerquickstart.tex
%% Join discussions: https://github.com/sjtug/SJTUBeamer/discussions
%% -----------------------------------------------------------------------

\documentclass[xcolor=table,dvipsnames,svgnames,aspectratio=169]{ctexbeamer}
% 可以通过 fontset=macnew / fontset=ubuntu / fontset=windows 选项切换字体集;
% 如遇无法显示的数学符号,尝试对 ctexbeamer 文档类添加 no-math 选项;
% 写纯英文幻灯片可以改用 beamer 文档类。

\usepackage{amsmath}
\usepackage{booktabs}
\usepackage{caption}
\usepackage{colortbl}
\usepackage{datetime2}
\usepackage{fontawesome5}
\usepackage{graphicx}
\usepackage{hologo}
\usepackage{hyperref}
\usepackage{hyperxmp}
\usepackage{listings}
\usepackage{makecell}
\usepackage{multicol}
\usepackage{multirow}
\usepackage{physics}
\usepackage{pifont}
\usepackage{shapepar}
\usepackage{siunitx}
\usepackage{tabularray}
\UseTblrLibrary{booktabs}
\usepackage{tabularx}
\usepackage{tcolorbox}
\usepackage{tikz}
\usetikzlibrary{arrows}
\usepackage{tipa}
\usepackage[normalem]{ulem}
\usepackage{xeCJKfntef}
\usepackage{xcolor}

% 参考文献设置,使用 style=gb7714-2015 样式为标准顺序编码制,
% 使用 style=gb7714-2015ay 样式可以改为著者-出版年制。
\usepackage[backend=biber,style=gb7714-2015]{biblatex}
\addbibresource{ref.bib}

% 该行指定了图像的额外搜索路径
\graphicspath{{figures/}}

\hypersetup{
  pdfcopyright       = {Licensed under CC-BY-SA 4.0. Some rights reserved.},
  pdflicenseurl      = {http://creativecommons.org/licenses/by-sa/4.0/},
  unicode            = true,
  psdextra           = true,
  pdfdisplaydoctitle = true
}

\pdfstringdefDisableCommands{
  \let\\\relax
  \let\quad\relax
  \let\hspace\@gobble
}

\newcommand\link[1]{\href{#1}{\faLink}}
\newcommand\pkg[1]{\texttt{#1}}

\def\cmd#1{\texttt{\color{structure}\footnotesize $\backslash$#1}}
\def\env#1{\texttt{\color{structure}\footnotesize #1}}
\def\cmdxmp#1#2#3{\small{\texttt{\color{structure}$\backslash$#1}\{#2\}
\hspace{1em}\\ $\Rightarrow$\hspace{1em} {#3}\par\vskip1em}}

% \tikzexternalize[prefix=build/]
% 如果您需要缓存 tikz 图像,请取消注释上一行,并在编译选项中添加 -shell-escape。

\lstset{
  language=[LaTeX]TeX,           % 更改高亮语言
  texcsstyle=*\color{cprimary},  % 只在高亮 LaTeX 语言时必须
  tabsize=2,
  basicstyle=\ttfamily\small,%
  keywordstyle=\color{cprimary},%
  stringstyle=\color{csecondary},%
  commentstyle=\color{ctertiary!50!gray},%
  breaklines,%
}

% 10.1 Adding a Title Page
% \title[⟨short title⟩]{⟨title⟩}
% The ⟨short title⟩ is used in headlines and footlines. Inside the ⟨title⟩ line breaks can be inserted using
% the double-backslash command.
\title[基于智能手机的数据和模型双驱动定位方法研究]{\textbf{基于智能手机的数据和模型双驱动定位方法研究}} % 页脚显示标题 | 首页标题
% \subtitle[⟨short subtitle⟩]{⟨subtitle⟩}
% The ⟨short subtitle⟩ is not used by default, but is available via the insert \insertshortsubtitle. The
% subtitle is shown below the title in a smaller font.
\subtitle{博士学位论文预答辩}
% \author[⟨short author names⟩]{⟨author names⟩}
% The names should be separated using the command \and.
\author{钱隆}
% \institute[⟨short institute⟩]{⟨institute⟩}
% If more than one institute is given, they should be separated using the command \and and they should
% be prefixed by the command \inst with different parameters.
\institute[测绘遥感信息工程全国重点实验室]{武汉大学 \  测绘遥感信息工程全国重点实验室}
% \date[⟨short date⟩]{⟨date⟩}
\date{\the\year 年 \the\month 月}
% \subject{⟨text⟩}
% Enters the ⟨text⟩ as the subject text in the pdf document info. It currently has no other effect.
\subject{武汉大学博士学位论文答辩}
% \keywords{⟨text⟩}
% Enters the ⟨text⟩ as keywords in the pdf document info. It currently has no other effect.
\keywords{答辩}

% 15 Themes
% 15.1 Five Flavors of Themes
% \usetheme[⟨options⟩]{⟨name list⟩}
% Installs the presentation theme named ⟨name⟩. Currently, the effect of this command is the same as
% saying
\usetheme[max,blue]{whubeamer}
% 使用 maxplus/max/min 切换标题页样式
% 使用 red/blue 切换主色调
% 使用 light/dark 切换亮/暗色模式
% 使用外样式关键词以获得不同的边栏样式
%   miniframes infolines  sidebar
%   default    smoothbars split	 
%   shadow     tree       smoothtree
% 使用 topright/bottomright 切换徽标位置
% 使用逗号分隔列表以同时使用多种选项

\setbeamertemplate{background}{}
% 对于 max 主题,如果需要关闭正文背景图,请取消注释上一行。

\begin{document}

% 使用节目录
\AtBeginSection[]{
  \begin{frame}
    %% 使用传统节目录,也可以将 subsectionstyle=... 换成 hideallsubsections 以隐藏所有小节信息
    \tableofcontents[currentsection,subsectionstyle=show/hide/hide]
    %% 或者使用节页
    % \sectionpage
  \end{frame}
}

% 使用小节目录
%\AtBeginSubsection[]{		       % 在每小节开始
%  \begin{frame}
%    %% 使用传统小节目录
%    \tableofcontents[currentsection,subsectionstyle=show/shaded/hide]
%    %% 或者使用小节页
%    % \subsectionpage
%  \end{frame}
%}

\maketitle

\begin{frame}{目录}
  % 10.5 Adding a Table of Contents
  % \tableofcontents[⟨comma-separated option list⟩]
  % Inserts a table of contents into the current frame.
  \vspace{-0.5cm}
  \tableofcontents[hideallsubsections]	% 隐藏所有小节信息
\end{frame}

% !TeX encoding = UTF-8
% !TeX root = ../whu-defense-qianlong.tex

%% ------------------------------------------------------------------------
%% Copyright (C) 2021-2023 SJTUG
%% 
%% SJTUBeamer Example Document by SJTUG
%% 
%% SJTUBeamer Example Document is licensed under a
%% Creative Commons Attribution-NonCommercial-ShareAlike 4.0 International License.
%% 
%% You should have received a copy of the license along with this
%% work. If not, see <http://creativecommons.org/licenses/by-nc-sa/4.0/>.
%% -----------------------------------------------------------------------

% 10.2 Adding Sections and Subsections
% \section<⟨mode specification⟩>[⟨short section name⟩]{⟨section name⟩}
% Starts a section. No heading is created. By default the ⟨section name⟩ is shown in the table of contents
% and in the navigation bars; if ⟨short section name⟩ is specified, it will be used in the navigation bars
% instead; if ⟨short section name⟩ is explicitly empty, it will not appear in the navigation
\section{研究背景与意义}

\begin{frame}[t]
	\frametitle{研究背景} 
	\framesubtitle{国家层面}
	中华人民共和国国民经济和社会发展第十四个五年规划和2035年远景目标纲要
	% 12.7 Splitting a Frame into Multiple Columns
	% \begin{columns}<⟨action specification⟩>[⟨options⟩]
	%   ⟨environment contents⟩
	% \end{columns}
	% • b will cause the bottom lines of the columns to be vertically aligned.
	% • c will cause the columns to be centered vertically relative to each other. Default, unless the global
	% option t is used.
	% • onlytextwidth is the same as totalwidth=\textwidth. This option can also be set for the whole
	% document with the onlytextwidth class option:
	% \documentclass[onlytextwidth]{beamer}
	% • t will cause the first lines of the columns to be aligned. Default if global option t is used.
	% • T is similar to the t option, but T aligns the tops of the first lines while t aligns the so-called
	% baselines of the first lines. If strange things seem to happen in conjunction with the t option (for
	% example if a graphic suddenly “drops down” with the t option instead of “going up”), try using
	% this option instead.
	% • totalwidth=⟨width⟩ will cause the columns to occupy not the whole page width, but only ⟨width⟩,
	% all told. Note that this means that any margins are ignored.
	% • height=⟨height⟩ will set a fixed height for the columns.
	\begin{columns}[t]
		\begin{column}{0.3\textwidth}
		   	\begin{figure}
    			\includegraphics[height=4.2cm]{中华人民共和国国民经济和社会发展第十四个五年规划和2035年远景目标纲要_封面.png}
		   	\end{figure}
		\end{column}   
		\begin{column}{0.7\textwidth}
		    \begin{block}{专栏 4 制造业核心竞争力提升}
				\begin{tabularx}{\linewidth}{@{\extracolsep{\fill}}l X}
		            05 & 北斗产业化应用 \\
		               & 突破通信导航一体化融合等技术,建设北斗应用产业创新平台,在通信、金融、能源、民航等行业开展典型示范,\CJKunderline{推动北斗在车载导航、智能手机、穿戴设备等消费领域市场化规模化应用}。	
		   		\end{tabularx}		   		 
			\end{block}
		\end{column}
	\end{columns}
\end{frame}

\begin{frame}[t]
	\frametitle{研究背景} 
	\framesubtitle{产业层面}
	2025中国卫星导航与位置服务产业发展白皮书
	\begin{columns}[t]
		\begin{column}{0.3\textwidth}
%		    \vspace{-0.4cm}
		   	\begin{figure}
    			\includegraphics[height=4.6cm]{2025中国卫星导航与位置服务产业发展白皮书_封面.png}
		   	\end{figure}
		\end{column}   
		\begin{column}{0.7\textwidth}
		    \begin{block}{大众消费领域将成北斗规模应用主战场}
		        \begin{itemize}
		          \item 已有约2.88亿部智能手机支持北斗定位功能,占比达 \SI{98}{\percent}
		          \item 大众地图软件基于北斗高精度的车道级导航功能已覆盖全国 \SI{99}{\percent} 以上的城市和乡镇道路
		          \item 11家主要电子地图服务供应商提供位置服务日均超 1 万亿次,日均提供导航服务总里程超 40 亿公里
		        \end{itemize}
			\end{block}
		\end{column}
	\end{columns}  
\end{frame}

% !TeX encoding = UTF-8
% !TeX root = ../whu-defense-qianlong.tex

%% ------------------------------------------------------------------------
%% Copyright (C) 2021-2023 SJTUG
%% 
%% SJTUBeamer Example Document by SJTUG
%% 
%% SJTUBeamer Example Document is licensed under a
%% Creative Commons Attribution-NonCommercial-ShareAlike 4.0 International License.
%% 
%% You should have received a copy of the license along with this
%% work. If not, see <http://creativecommons.org/licenses/by-nc-sa/4.0/>.
%% -----------------------------------------------------------------------

\section{国内外研究现状}

\begin{frame}
	\frametitle{行人航迹推算数据集}
	\begin{columns}[t]
		\begin{column}{0.6\textwidth}
		{
		    \tiny
		    \setlength{\tabcolsep}{2pt}
			\begin{tabular*}{1.2\textwidth}{@{\extracolsep{\fill}} cccrrrrcccccc}
				\toprule
				\multirow{3.5}{*}{数据集} & \multirow{3.5}{*}{日期} & \multirow{3.5}{*}{\makecell[c]{有\\效\\性}} & \multicolumn{6}{c}{规模与多样性} & \multicolumn{2}{c}{惯性传感器} & \multicolumn{2}{c}{参考真值}  \\ 
				\cmidrule{4-9} \cmidrule{10-11} \cmidrule{12-13}
				& & & \multirow{2}{*}{\makecell[c]{长度\\$\left(\unit{\km}\right)$}} 
				& \multirow{2}{*}{\makecell[c]{时长\\$\left(\unit{\hour}\right)$}} 
				& \multirow{2}{*}{轨迹} & \multirow{2}{*}{\makecell[c]{采集\\人数}} 
				& \multirow{2}{*}{\makecell[c]{携带\\方式}} 
				& \multirow{2}{*}{\makecell[c]{运动\\模式}}
				& \multirow{2}{*}{数量} 
				& \multirow{2}{*}{\makecell[c]{采样率\\$\left(\unit{\hertz}\right)$}} 
				& \multirow{2}{*}{设备} 
				& \multirow{2}{*}{\makecell[c]{采样率\\$\left(\unit{\hertz}\right)$}} \\
				& & & & & & & & & & & & \\
				\midrule
		        \href{https://yanhangpublic.github.io/ridi/}{RIDI}                          & 2017 & \ding{51} & 10.4 &   1.6 &   72 &         6 & 4 & 2 &   2 & 200 & Tango         & 200 \\
		   		\href{https://zenodo.org/records/1476931}{ADVIO}                            & 2018 & \ding{51} &  4.5 &   1.1 &   23 &         1 & 1 & 2 &   3 & 100 & Tango         & 100 \\
		   		\href{http://deepio.cs.ox.ac.uk/}{OxIOD}                                    & 2018 & \ding{51} & 42.5 &  14.7 &  158 &         5 & 4 & 2 &   4 & 100 & Vicon         & 250 \\
		   		\href{https://cvg.cit.tum.de/data/datasets/visual-inertial-dataset}{TUM VI} & 2018 & \ding{51} & 20   &   3.8 &   28 &         1 & 1 & 2 &   1 & 200 & OptiTrack     & 120 \\
		   		\href{https://ronin.cs.sfu.ca/}{RoNIN}                                      & 2020 & \ding{51} & 56.6 &  42.7 &  276 &       100 & 4 & 2 &   4 & 200 & Tango         & 200 \\
		   		\href{https://github.com/MAPS-Lab/smartphone-tracking-dataset}{MAPS Lab}    & 2021 & \ding{51} &  0.7 &   2.0 &    3 &         1 & 2 & 2 &   2 & 200 & Vicon         &   5 \\
		   		\href{https://github.com/LF1952987278/SIMD_Repository}{SIMD}                & 2023 & \ding{51} & 717  & 190   & 4562 & $\geq$150 & 4 & 2 & 572 &  50 & GNSS | Marker &   1 \\
		   		\href{https://github.com/BehnamZeinali/IMUNet}{IMUNet}                      & 2024 & \ding{51} & 35.9 &   9   &  126 &         4 & 4 & 2 &   5 & 200 & Tango         & 200 \\
				\bottomrule	
			\end{tabular*}        
		}
		\end{column}  
		\begin{column}{0.1\textwidth}
		\end{column} 
		\begin{column}{0.3\textwidth}
		    \begin{itemize}
				\item 数据量
				\item 多样性
			\end{itemize}
		\end{column}
	\end{columns}
   	\begin{figure}
		\includegraphics[height=2cm]{data_pipeline.png}
   	\end{figure}
\end{frame}

\begin{frame}
	\frametitle{载具航迹推算方法数据集}
	\vspace{-0.5cm}
	\begin{columns}[t]
		\begin{column}{0.7\textwidth}
		{
		    \tiny
		    \setlength{\tabcolsep}{2pt}
			\begin{tabular*}{\textwidth}{@{\extracolsep{\fill}}c c c l rrrc lc}
				\toprule
				\multirow{2}{*}{数据集} & \multirow{3}{*}{日期} & \multirow{2}{*}{\makecell[c]{有\\效\\性}} & \multirow{2}{*}{载具} &\multicolumn{4}{c}{规模} & \multicolumn{2}{c}{真值} \\
				\cmidrule{5-8} \cmidrule{9-10}
				& & & & 轨迹 & \makecell[c]{长度\\$\left(\unit{\km}\right)$} & \makecell[c]{时长\\$\left(\unit{\hour}\right)$} & \makecell[c]{采样率\\$\left(\unit{\hertz}\right)$} & \multicolumn{1}{c}{设备} & \makecell[c]{采样率\\$\left(\unit{\hertz}\right)$} \\
				\midrule
				\href{https://www.cvlibs.net/datasets/kitti/eval_odometry.php}{KITTI} 
				& 2012 & \ding{51} & LV                &  11 &   22.2 & 40.1 &    200 & GNSS/INS                       & 100       \\
				\href{https://projects.asl.ethz.ch/datasets/doku.php?id=kmavvisualinertialdatasets}{EuRoC MAV} 
				& 2016 & \ding{51} & MR                &  11 &    0.9 &  0.4 &    200 & Leica MS50 \& Vicon            & 20 \& 100 \\
				\href{https://sites.google.com/view/complex-urban-dataset/home}{Complex Urban Dataset} 
				& 2018 & \ding{51} & LV                &  41 &  356.1 &  $-$ &    100 & SLAM                           & 100       \\
				\href{https://github.com/onyekpeu/IO-VNBD}{IO-VNBD} 
				& 2021 & \ding{51} & LV                &  43 & 4400   & 58   &    100 & GNSS                           &  10       \\
				\href{https://www.kaggle.com/competitions/smartphone-decimeter-2022/data}{GSDC} 
				& 2022 & \ding{51} & LV                &  62 & 2123.6 & 30.3 & 50-200 & GNSS/INS                       &   1       \\
				\href{https://www.kaggle.com/competitions/smartphone-decimeter-2023}{GSDC} 
				& 2023 & \ding{51} & LV                &  65 & 1685.3 & 29.4 & 50-200 & GNSS/INS                       &   1       \\
				\href{https://figshare.com/articles/dataset/Multiple_and_Gyro-Free_Inertial_Datasets/26927089/1?file=48979765}{MAGF-ID} 
				& 2024 & \ding{51} & LV \& MR          & 115 &    $-$ &  5.6 &    200 & $\mathrm{GNSS}^{\mathrm{RTK}}$ & 200        \\
				\bottomrule	
			\end{tabular*}   
		}
		\end{column}   
		\begin{column}{0.3\textwidth}
		    \begin{itemize}
				\item 以智能手机为中心的车载数据采集
			\end{itemize}
		\end{column}
	\end{columns}
	\begin{columns}[t]
		\begin{column}{0.3\textwidth}
		   	\begin{figure}
				\includegraphics[height=2.5cm]{KITTIRecordingPlatform.jpg}
		   	\end{figure}
		\end{column}
		\begin{column}{0.3\textwidth}
		    \vspace{-0.5cm}
		   	\begin{figure}
				\includegraphics[height=2.5cm]{ComplexUrbanDatasetRecordingPlatform.png}
		   	\end{figure}
		\end{column}
		\begin{column}{0.3\textwidth}
		    \vspace{-0.5cm}
		   	\begin{figure}
				\includegraphics[height=2.5cm]{GSDC2023RecordingPlatform.png}
		   	\end{figure}
		\end{column}
	\end{columns}
\end{frame}

\begin{frame}
	\frametitle{国内外研究现状}
	\framesubtitle{数据驱动行人航迹推算方法}
	\vspace{-0.5cm}
	\begin{columns}[t]
		\begin{column}{0.5\textwidth}
		{
		    \tiny
		    \setlength{\tabcolsep}{2pt}
			\begin{tabular*}{\textwidth}{@{\extracolsep{\fill}}lclc}
				\toprule
				\multicolumn{1}{c}{模型} & 日期 & 期刊/会议 & 数据集 \\
				\midrule
				\multirow{2}{*}{IONet} & 2018 & AAAI                            & \multirow{2}{*}{OxIOD} \\
				                       & 2019 & IEEE. Trans. Mob. Comput.       &                        \\
				               L-IONet & 2020 & IEEE Internet Things J.         & OxIOD                  \\
				                 RoNIN & 2020 & ICRA                            & RoNIN                  \\
				        Extended IONet & 2021 & Expert Syst. Appl.              & OxIOD                  \\ 
				                   DIO & 2021 & IEEE Trans. Instrum. Meas.      & RoNIN \& Private       \\
				                  IDOL & 2021 & AAAI                            & IDOL                   \\
				                   RIO & 2022 & CVPR                            & Public \& Private      \\
				                 HNNTA & 2022 & IEEE Trans. Instrum. Meas.      & RoNIN \& Private       \\
				                  CTIN & 2022 & AAAI                            & Public \& Private      \\
				                  RIOT & 2023 & Sensors                         & OxIOD                  \\
				              DO IONet & 2023 & IEEE Access                     & OxIOD                  \\
				                 SSHNN & 2023 & IEEE Sens. J.                   & Private                \\
				                  SIMD & 2023 & IEEE Trans. Instrum. Meas.      & SIMD                   \\
				                IMUNet & 2024 & IEEE Trans. Instrum. Meas.      & Public                 \\
				              ResMixer & 2024 & IEEE Sens. J.                   & RoNIN \& Private       \\
				\bottomrule
			\end{tabular*}     
		}
		\end{column}
		\begin{column}{0.5\textwidth}
		    \vspace{-1.0cm}
			\begin{block}{数据驱动模型估计精度}
			    {
			        \footnotesize
					\begin{itemize}
						\item 应用先进深度神经网络模型
						\item 设计合理损失函数
						\item 设计合理损失函数
					\end{itemize}			    
			    }
			\end{block}
		    \begin{block}{数据驱动模型输出维度}
  			    {
  			        \footnotesize
					\begin{itemize}
						\item 三维位姿表达
						\item 位置和姿态多任务学习
					\end{itemize}
				}
			\end{block}
			\begin{block}{数据驱动模型计算量}
			    {
			        \footnotesize
					\begin{itemize}
						\item 应用轻量级网络减小计算量
					\end{itemize}
				}
			\end{block}
		\end{column}
	\end{columns}	
\end{frame}

\begin{frame}
	\frametitle{数据和模型双驱动行人航迹推算方法}
	\begin{columns}[t]
		\begin{column}{0.5\textwidth}
		{
		    \tiny
		    \setlength{\tabcolsep}{2pt}
			\begin{tabular*}{\linewidth}{@{\extracolsep{\fill}}cccc}
				\toprule
				\multicolumn{1}{c}{模型} & 日期 & 期刊/会议 & 数据集 \\
				\midrule
				    \rowcolor{gray!50} RIDI & 2018 & ECCV                            & RIDI                     \\
				                    EKF+CNN & 2018 & MLSP                            & Public                    \\
				  \multirow{2}{*}{EKF+LSTM} & 2018 & IPIN                            & \multirow{2}{*}{Private} \\
				                            & 2019 & IEEE Sens. J.                   &                          \\
				                      AZUPT & 2019 & GLOBECOM                        & Private                  \\	
				    \rowcolor{gray!50} TLIO & 2019 & IEEE Robot. Autom. Lett.        & Private                  \\
				\rowcolor{gray!50} IEKF+CNN & 2020 & IROS                            & RIDI \& Private          \\
				                   EKF+LSTM & 2020 & IROS                            & Private                  \\
				                     DeepIT & 2021 & IMWUT                           & Private                  \\
				                       MINN & 2022 & IEEE Sens. J.                   & SLE \& WDE               \\
				                   TinyOdom & 2022 & IMWUT                           & Public \& Private        \\
				                 MSCKF+LLIO & 2022 & IEEE Trans. Instrum. Meas.      & Private                  \\
				                 SCEKF+LLIO & 2022 & IEEE Internet Things J.         & Private                  \\
				                 EKF+ResNet & 2023 & ICARM                           & RoNIN                    \\
				\bottomrule
			\end{tabular*}          
		}
		\end{column}   
		\begin{column}{0.5\textwidth}
			\begin{block}{数据驱动模型估计精度}
			   {
			       \footnotesize
					\begin{itemize}
						\item 三维位姿表达
					\end{itemize}
				}
			\end{block}
		    \begin{block}{数据驱动模型输出观测量}数据驱动模型估计精度
  			    {
  			        \footnotesize
					\begin{itemize}
						\item 三维位姿表达
						\item 位置和姿态多任务学习
					\end{itemize}
				}
			\end{block}
			\begin{block}{数据驱动模型计算量}
			    {
			        \footnotesize
					\begin{itemize}
						\item 应用轻量级网络减小计算量
					\end{itemize}
				}
			\end{block}
		\end{column}
	\end{columns}	
\end{frame}

\begin{frame}
	\frametitle{数据驱动载具航迹推算方法}
	\begin{columns}[t]
		\begin{column}{0.5\textwidth}
		{
		    \tiny
		    \setlength{\tabcolsep}{2pt}
			\begin{tabular*}{\linewidth}{@{\extracolsep{\fill}}cccc}
				\toprule
				\multicolumn{1}{c}{模型} & 日期 & 期刊/会议 & 数据集 \\
				\midrule
				OriNet  & 2020 & IEEE Robot. Autom. Lett.  & EuRoC MAV \\
				DeepVIP & 2022 & IEEE Trans. Veh. Technol. & Private   \\ % IEEE Transactions on Vehicular Technology ( Volume: 71, Issue: 12, December 2022) | 17 August 2022
				LSTM    & 2023 & Machines                  & JKK       \\
				DI-EME  & 2024 & IEEE Sens. J.             & Private   \\
				\bottomrule
			\end{tabular*}         
		}
		\end{column}   
		\begin{column}{0.5\textwidth}
		    \begin{block}{研究目标}
				\begin{itemize}
					\item 提升模型估计精度
					\item 降低模型计算开销
				\end{itemize}
			\end{block}
			\begin{block}{提升模型估计精度}
				\begin{itemize}
					\item maxplus
				\end{itemize}
			\end{block}
		\end{column}
	\end{columns}	
\end{frame}

\begin{frame}
    \frametitle{数据和模型双驱动载具航迹推算方法}
	\begin{columns}[t]
		\begin{column}{0.5\textwidth}
		{
		    \tiny
		    \setlength{\tabcolsep}{2pt}
	   		\begin{tabular*}{\linewidth}{@{\extracolsep{\fill}}cccc}
				\toprule
				\multicolumn{1}{c}{模型} & 日期 & 期刊/会议 & 数据集 \\
				\midrule
				AbolDeepIO & 2019 & IEEE Trans. Intell. Transp. Syst. & EuRoC          \\ % IEEE Transactions on Intelligent Transportation Systems
				RINS-W     & 2019 & IROS                              & KAIST          \\ % 2019 IEEE/RSJ International Conference on Intelligent Robots and Systems (IROS)
				AI-IMU     & 2020 & IEEE T. Intell. Veh.              & KITTI Odometry \\ % IEEE Transactions on Intelligent Vehicles
				RAN        & 2021 & IEEE Trans. Veh. Technol.         & Private         \\ % IEEE Transactions on Vehicular Technology
				RNN        & 2021 & ICRA                              & EuRoC \& KAIST    \\ 
				TinyOdom   & 2022 & IMWUT                             & Public \& Private \\
				RNN-IEKF   & 2022 & IROS                              & KITTI Odometry    \\
				OdoNet     & 2022 & IEEE Sens. J.                     & Private         \\ % IEEE Sensors Journal
				SdoNet     & 2023 & IEEE Internet Things J.           & KITTI Odometry \\ % IEEE Internet of Things Journal
				DeepOdo    & 2023 & IEEE Trans. Instrum. Meas.        & Private         \\ % IEEE Transactions on Instrumentation and Measurement
				IEKF+CNN   & 2023 & IEEE Trans. Ind. Electron.        & KITTI \& Private         \\ % IEEE Transactions on Industrial Electronics | 15 August 2023
				KF+VHRNet  & 2025 & IEEE Sens. J.                     & Private         \\ % IEEE Sensors Journal
				\bottomrule
	   		\end{tabular*}
		}
		\end{column}   
		\begin{column}{0.5\textwidth}
		    \begin{block}{研究目标}
				\begin{itemize}
					\item 提升模型估计精度
					\item 降低模型计算开销
				\end{itemize}
			\end{block}
			\begin{block}{提升模型估计精度}
				\begin{itemize}
					\item maxplus
				\end{itemize}
			\end{block}
		\end{column}
	\end{columns}
\end{frame}

% !TeX encoding = UTF-8
% !TeX root = ../whu-defense-qianlong.tex

%% ------------------------------------------------------------------------
%% Copyright (C) 2021-2023 SJTUG
%% 
%% SJTUBeamer Example Document by SJTUG
%% 
%% SJTUBeamer Example Document is licensed under a
%% Creative Commons Attribution-NonCommercial-ShareAlike 4.0 International License.
%% 
%% You should have received a copy of the license along with this
%% work. If not, see <http://creativecommons.org/licenses/by-nc-sa/4.0/>.
%% -----------------------------------------------------------------------

\section{研究内容}

\subsection{研究内容1 数据驱动行人航迹推算方法研究}

\begin{frame}[t]
	\frametitle{研究内容1 数据驱动行人航迹推算方法研究}	
	\framesubtitle{数据驱动行人航迹推算方法架构}
	\vspace{-1cm}
	\begin{columns}[t]
		\begin{column}{0.5\textwidth}
		    \begin{block}{存在问题}
		    {
		    	\small
		        \begin{itemize}
					\item 只能输出模型预设的状态量
					\item 高频模型输出需要计算资源
		        \end{itemize}
    		 } 
			\end{block}
			\begin{block}{研究思路}
			 {
 		    	\small
				\begin{itemize}
					\item 基于滤波框架降低模型输出频率
					\item 量化手机握持姿态解决姿态多变的问题
				\end{itemize}
			}
			\end{block}
		\end{column}   
		\begin{column}{0.5\textwidth}
		    \begin{block}{状态定义}
      		{
  		    	\small
				\begin{equation*}
				   	\vb*{x}\left(t_n\right) := \left( \vb*{\chi}\left(t_n\right), \vb*{\theta}\left(t_n\right), \vb*{R}_s^b\left(t_n\right)\right) 
				\end{equation*}
			}
			\end{block}
			\begin{figure}
				\centering
				\includegraphics[height=2.2cm]{PedestrianDeadReckoningAlgorithmArchitecture.pdf}
			\end{figure}
		\end{column}
	\end{columns} 
	\begin{block}{数据驱动观测量}
		\begin{equation*}
			\vb*{R}_{b}^{n}\left(t_{n+1}\right), \vb*{R}_{s}^{b}\left(t_{n+1}\right),v^{b}_{b, \mathrm{lon}}\left(t_{n+1}\right)
			=
			f_{\mathrm{Estimator}}\left(\{\tilde{\vb*{\omega}}\left(t_i\right), \tilde{\vb*{f}}\left(t_i\right)\}_{i = n-W_{\mathrm{est}}}^{n}\right)
		\end{equation*}
	\end{block}
\end{frame}

\begin{frame}
	\frametitle{研究内容1 数据驱动行人航迹推算方法研究}
	\framesubtitle{时间更新}
	\begin{block}{状态量时间更新}
		{
			\vspace{-0.4cm}
			{
				\footnotesize
				\begin{align*}
					\vb*{R}\left(t_{n+1}\right) &= \vb*{R}\left(t_{n+1}\right) + \vb*{R}\left(t_{n}\right) \vb*{\Gamma}_{0}\left( \vb*{\omega}\left(t_{n}\right) \Delta t \right) \\
					\vb*{v}\left(t_{n+1}\right) &= \vb*{v}\left(t_{n}\right) 
					+ \vb*{R}\left(t_{n}\right) \vb*{\Gamma}_{1}\left( \vb*{\omega}\left(t_{n}\right) \Delta t \right) \vb*{f}\left(t_{n}\right) \Delta t 
					+ \vb*{g}\Delta t \\
					\vb*{p}\left(t_{n+1}\right) &= \vb*{p}\left(t_{n}\right) 
					+ \vb*{v}\left(t_{n}\right) \Delta t
					+ \vb*{R}\left(t_{n}\right) \vb*{\Gamma}_{2}\left( \vb*{\omega}\left(t_{n}\right) \Delta t \right) \vb*{f}\left(t_{n}\right) \left(\Delta t\right)^2 
					+ \frac{1}{2}\vb*{g}\left(\Delta t\right)^2 \\
					\vb*{\delta \omega}\left(t_{n+1}\right) &= \vb*{\delta \omega}\left(t_{n}\right) \\
					\vb*{\delta f}\left(t_{n+1}\right) &= \vb*{\delta f}\left(t_{n}\right)
				\end{align*}
			}
			\vspace{-0.6cm}
			\begin{equation*}
				\vb*{R}_s^b\left(t_{n+1}\right) = \vb*{R}_s^b\left(t_{n}\right) 
			\end{equation*}
		}
	\end{block}
	\begin{block}{状态量协方差时间更新}
		{\footnotesize 		
			\begin{equation*}
				\vb*{P}\left(t_{n+1}\right) = \vb*{F}\left(t_{n}\right) \vb*{P}\left(t_{n+1}\right) \vb*{F}^{\mathrm{T}}\left(t_{n}\right)
				+ \vb*{G}\left(t_{n}\right) \vb*{Q}\left(t_{n}\right) \vb*{G}^{\mathrm{T}}\left(t_{n}\right)
			\end{equation*}
		}
	\end{block} 
\end{frame}

\begin{frame}
	\frametitle{研究内容1 数据驱动行人航迹推算方法研究}
	\framesubtitle{量测更新}
	\vspace{-0.8cm}
	\begin{columns}[t]
		\begin{column}{0.5\textwidth}
			\begin{block}{观测方程}					
    			\scalebox{0.7}{
    				{\tiny
	    				\setlength{\tabcolsep}{2pt}
						\begin{tabular*}{\textwidth}{@{\extracolsep{\fill}}lll}
				            $ \tilde{\vb*{y}}_{\vb*{R}_s^b} = \tilde{\vb*{R}}_s^b $ 
				            & $ \hat{\vb*{y}}_{\vb*{R}_s^b} = \hat{\vb*{R}}_s^b $ 
				            & $ \vb*{r}_{\vb*{R}_s^b} = \log_{\mathrm{G}_{\mathrm{SO}\left(3\right)}}\left(\tilde{\vb*{y}}_{\vb*{R}_s^b}\left(\hat{\vb*{y}}_{\vb*{R}_s^b}\right)^{\mathrm{T}}\right) $ 
				            \\
				            $ \tilde{\vb*{y}}_{\vb*{R}_b^n} = \tilde{\vb*{R}}_b^n $
				            & $ \hat{\vb*{y}}_{\vb*{R}_b^n} = \hat{\vb*{R}}_s^n \left(\hat{\vb*{R}}_s^b\right)^{\mathrm{T}} $ 
				            & $ \vb*{r}_{\vb*{R}_b^n} = \log_{\mathrm{G}_{\mathrm{SO}\left(3\right)}}\left(\tilde{\vb*{y}}_{\vb*{R}_b^n}\left(\hat{\vb*{y}}_{\vb*{R}_b^n}\right)^{\mathrm{T}}\right) $ 
				            \\
				            $ \tilde{\vb*{y}}_{\vb*{v}_b^b} = 
				            \begin{pmatrix}
													           		0 & \tilde{v}_{\mathrm{lat}}^b & 0
													           	\end{pmatrix}^{\mathrm{T}} $ 
				            & $ \hat{\vb*{y}}_{\vb*{v}_b^b} = \hat{\vb*{R}}_s^b \left(\hat{\vb*{R}}_s^n\right)^{\mathrm{T}} \hat{\vb*{v}}_s^n $ 
				            & $ \vb*{r}_{\vb*{v}_b^b} = \tilde{\vb*{y}}_{\vb*{v}_b^b} - \hat{\vb*{y}}_{\vb*{v}_b^b} $
				   		\end{tabular*}
			   		}
		   		}
		   		\vspace{-0.5cm}
		   		\scalebox{0.7}{\parbox{.5\linewidth}{%
			   		{\tiny
						\begin{align*}
			    			\begin{pmatrix}
								\vb*{H}^{\vb*{R}_{s}^b} \\
								\vb*{H}^{\vb*{R}_b^n}  \\
								\vb*{H}^{\vb*{v}^b_{b}}
							\end{pmatrix} 
							&= 
							\begin{pmatrix}
								\vb*{I}_3 & \vb*{0}_3 & \vb*{0}_3 & \vb*{0}_3 & \vb*{I}_3 & \vb*{0}_3 \\
								\left(\vb*{R}_s^n \left(\vb*{R}_{s}^b\right)^{\mathrm{T}}\right)^{\mathrm{T}} & \vb*{0}_3 & \vb*{0}_3 & \vb*{0}_3 & \vb*{0}_3 & -\vb*{I}_3 \\
								\vb*{0}_3 & \hat{\vb*{R}}_s^v \left(\hat{\vb*{R}}_s^w\right)^{\mathrm{T}} & \vb*{0}_3 & \vb*{0}_3 & \vb*{0}_3 & -\left( \hat{\vb*{R}}_s^v \left(\hat{\vb*{R}}_s^w\right)^{\mathrm{T}} \hat{\vb*{v}}_s^w \right)_{\times}
							\end{pmatrix}
						\end{align*}
					}
		   		}}
			\end{block}
		\end{column}   
		\begin{column}{0.5\textwidth}
			\begin{block}{状态量量测更新}
			\vspace{-0.5cm}
			{
				\tiny	
				\begin{align*}
			    	\vb*{S}\left(t_{n+1}\right) &= \vb*{H}\left(t_{n+1}\right) \vb*{P}\left(t_{n+1}\right) \vb*{H}^{\mathrm{T}}\left(t_{n+1}\right) + \vb*{N}\left(t_{n+1}\right) \\
			    	\vb*{K}\left(t_{n+1}\right) &= \vb*{P}\left(t_{n+1}\right) \vb*{H}^{\mathrm{T}}\left(t_{n+1}\right) \vb*{S}^{-1}\left(t_{n+1}\right) \\
				    \vb*{e}\left(t_{n+1}\right) &= \vb*{H}\left(t_{n+1}\right) \vb*{r}\left(t_{n+1}\right) \\
					\vb*{\chi}\left(t_{n+1}^+\right)   &= \exp_{\mathrm{G}_{\mathrm{SE}_2\left(3\right)}}\left( \vb*{e}_{\vb*{\xi}_{\vb*{\chi}}}\left(t_{n+1}^+\right) \right) \vb*{\chi}\left(t_{n+1}\right)\\
					\vb*{\theta}\left(t_{n+1}^+\right) &= \vb*{\theta}\left(t_{n+1}\right) + \vb*{e}_{\vb*{\xi}_{\vb*{\theta}}}\left(t_{n+1}^+\right)
				\end{align*}
			}
			\vspace{-1.0cm}
			{
				\scriptsize	
				\begin{align*}
					\vb*{R}_s^b\left(t_{n+1}^+\right)  &= \exp_{\mathrm{G}_{\mathrm{SO}\left(3\right)}}\left( \vb*{e}_{\vb*{\xi}_{\vb*{R}_s^b}}\left(t_{n+1}^+\right) \right) \vb*{R}_s^b\left(t_{n+1}\right)
				\end{align*}
			}
			\end{block} 
		\end{column}
	\end{columns} 
	\begin{block}{状态量协方差量测更新}
		{\scriptsize%
			\begin{equation*}
				\vb*{P}\left(t_{n+1}^+\right) = \left(\vb*{I} - \vb*{K}\left(t_{n+1}\right)\vb*{H}\left(t_{n+1}\right)\right) \vb*{P}\left(t_{n+1}\right) \left(\vb*{I} - \vb*{K}\left(t_{n+1}\right)\vb*{H}\left(t_{n+1}\right)\right)^{\mathrm{T}} + \vb*{K}\left(t_{n+1}\right)\vb*{N}\left(t_{n+1}\right) \vb*{K}^{\mathrm{T}}\left(t_{n+1}\right)
			\end{equation*}
		}
	\end{block} 
\end{frame}

\begin{frame}[t]
	\frametitle{研究内容1 数据驱动行人航迹推算方法研究}
	\framesubtitle{RoNIN 数据集不同方法对比轨迹评价结果}
    {   
        \tiny        
		\begin{tabular*}{\linewidth}{@{\extracolsep{\fill}} cl cccccc}
			\toprule
			测试集                   & 评价指标                                        & PDR  & IONet & RoNIN & SSHNN & IMUNet & Proposed \\
			\midrule
			\multirow{2}{*}{Seen}   & $\vb*{e}^{\mathrm{APE}}_{\mathrm{RMSE}}$        & 29.5 & 21.1  & 3.5   & 4.96  & 3.7    & 4.0 \\
			                        & $\vb*{e}^{\mathrm{RPE}}_{RMSE,\SI{1}{\minute}}$ & 21.4 & 24.6  & 2.7   & 3.48  & 2.7    & 2.2 \\
			\multirow{2}{*}{Unseen} & $\vb*{e}^{\mathrm{APE}}_{\mathrm{RMSE}}$        & 27.7 & 32.0  & 5.1   & 6.80  & 6.1    & 3.5 \\
						            & $\vb*{e}^{\mathrm{RPE}}_{RMSE,\SI{1}{\minute}}$ & 23.2 & 26.9  & 4.4   & 5.55  & 4.7    & 2.0 \\
			\bottomrule 
		\end{tabular*}     
   	}
	\begin{columns}[t]
		\begin{column}{0.5\textwidth}
		   	\begin{figure}
    			\includegraphics[height=4.2cm]{中华人民共和国国民经济和社会发展第十四个五年规划和2035年远景目标纲要_封面.png}
		   	\end{figure}
		\end{column}   
		\begin{column}{0.5\textwidth}

		\end{column}
	\end{columns} 
\end{frame}

\begin{frame} 
 	\frametitle{数据驱动行人航迹推算方法研究}
 	\framesubtitle{RoNIN数据集 Seen 测试集观测量约束位姿评价结果}
    {\footnotesize
    \setlength{\tabcolsep}{2pt}
		\begin{tabular*}{\linewidth}{@{\extracolsep{\fill}}lcrrrrrrrrrrrrr}
			\toprule
			\multirow{2}{*}{轨迹} & \multirow{2}{*}{\makecell{里程\\$\left(\unit{m}\right)$}} & \multirow{2}{*}{\makecell{时长\\$\left(\unit{s}\right)$}} 
			& \multicolumn{4}{c}{InEKF + $\vb*{v}_s^n$} & \multicolumn{4}{c}{InEKF + $\vb*{R}_b^n$ + $\vb*{v}$} & \multicolumn{4}{c}{InEKF + $\vb*{R}_b^n$ + $\vb*{v}_b^b$}\\
			\cmidrule{4-7} \cmidrule{8-11} \cmidrule{12-15}
			& & & $\vb*{e}^{\mathrm{AOE}}$ & $\vb*{e}^{\mathrm{ROE}}$ & $\vb*{e}^{\mathrm{APE}}$ & $\vb*{e}^{\mathrm{RPE}}$ & $\vb*{e}^{\mathrm{AOE}}$ & $\vb*{e}^{\mathrm{ROE}}$ & $\vb*{e}^{\mathrm{APE}}$ & $\vb*{e}^{\mathrm{RPE}}$ & $\vb*{e}^{\mathrm{AOE}}$ & $\vb*{e}^{\mathrm{ROE}}$ & $\vb*{e}^{\mathrm{APE}}$ & $\vb*{e}^{\mathrm{RPE}}$  \\
			\midrule
			a000\_7 & 287 & 294 & 12.3 & 10.3 & 2.1 & 2.4 & 8.9 & 5.5 & 6.5 & 5.0 & 0.9 & 0.9 & 1.9 & 1.1 \\
			a000\_11 & 274 & 280 & 43.8 & 21.6 & 1.9 & 1.3 & 4.1 & 3.5 & 9.9 & 11.6 & 0.6 & 0.8 & 5.9 & 6.9 \\
			a001\_2 & 454 & 630 & 14.7 & 22.5 & 4.3 & 4.2 & 2.8 & 3.6 & 14.8 & 12.4 & 2.6 & 2.9 & 2.6 & 2.0 \\
			a003\_3 & 388 & 487 & 66.6 & 81.9 & 9.8 & 3.3 & 3.3 & 6.1 & 11.1 & 6.0 & 3.3 & 5.8 & 5.7 & 2.1 \\
			a004\_3 & 252 & 256 & 21.0 & 38.8 & 1.7 & 0.7 & 3.0 & 1.6 & 2.7 & 3.7 & 0.8 & 0.9 & 1.1 & 0.6 \\
			a005\_3 & 415 & 591 & 48.5 & 73.6 & 6.6 & 2.0 & 1.8 & 2.2 & 5.3 & 4.0 & 1.0 & 1.3 & 2.4 & 1.3 \\
			a009\_1 & 271 & 414 & 36.5 & 64.5 & 5.0 & 5.2 & 2.2 & 2.5 & 14.2 & 4.4 & 1.5 & 1.7 & 6.5 & 2.5 \\
			a010\_2 & 385 & 653 & 14.3 & 19.7 & 9.8 & 4.6 & 13.2 & 5.7 & 12.9 & 6.7 & 2.4 & 2.4 & 2.4 & 1.4 \\
			a011\_2 & 377 & 684 & 14.4 & 8.6 & 2.4 & 0.7 & 12.7 & 3.7 & 7.6 & 7.0 & 0.9 & 0.7 & 5.8 & 2.8 \\
			a012\_2 & 296 & 593 & 14.3 & 23.9 & 6.0 & 2.9 & 10.5 & 4.6 & 8.3 & 4.9 & 3.5 & 2.2 & 7.8 & 2.2 \\
			平均值 & 420 & 555 & 31.1 & 38.8 & 5.9 & 3.6 & 6.7 & 4.1 & 8.4 & 7.0 & 1.9 & 1.7 & 4.0 & 2.2 \\
			\bottomrule 
		\end{tabular*}
	}	
\end{frame}

\begin{frame} 
 	\frametitle{数据驱动行人航迹推算方法研究}
 	\framesubtitle{RoNIN数据集 Unseen 测试集观测量约束位姿评价结果}
    {\footnotesize
    \setlength{\tabcolsep}{2pt}
		\begin{tabular*}{\linewidth}{@{\extracolsep{\fill}}lcrrrrrrrrrrrrr}
			\toprule
			\multirow{2}{*}{轨迹} & \multirow{2}{*}{\makecell{里程\\$\left(\unit{m}\right)$}} & \multirow{2}{*}{\makecell{时长\\$\left(\unit{s}\right)$}} 
			& \multicolumn{4}{c}{InEKF + $\vb*{v}_s^n$} & \multicolumn{4}{c}{InEKF + $\vb*{R}_b^n$ + $\vb*{v}$} & \multicolumn{4}{c}{InEKF + $\vb*{R}_b^n$ + $\vb*{v}_b^b$}\\
			\cmidrule{4-7} \cmidrule{8-11} \cmidrule{12-15}
			& & & $\vb*{e}^{\mathrm{AOE}}$ & $\vb*{e}^{\mathrm{ROE}}$ & $\vb*{e}^{\mathrm{APE}}$ & $\vb*{e}^{\mathrm{RPE}}$ & $\vb*{e}^{\mathrm{AOE}}$ & $\vb*{e}^{\mathrm{ROE}}$ & $\vb*{e}^{\mathrm{APE}}$ & $\vb*{e}^{\mathrm{RPE}}$ & $\vb*{e}^{\mathrm{AOE}}$ & $\vb*{e}^{\mathrm{ROE}}$ & $\vb*{e}^{\mathrm{APE}}$ & $\vb*{e}^{\mathrm{RPE}}$  \\
			\midrule
			a006\_2 & 284 & 838 & 1.9 & 1.5 & 1.6 & 1.1 & 1.8 & 1.4 & 15.4 & 10.4 & 1.8 & 1.4 & 6.1 & 5.6 \\
			a019\_3 & 343 & 443 & 1.1 & 1.9 & 4.7 & 2.6 & 1.4 & 2.4 & 9.5 & 5.6 & 1.3 & 2.1 & 4.2 & 2.3 \\
			a024\_1 & 387 & 458 & 0.6 & 0.8 & 1.4 & 1.1 & 0.7 & 1.0 & 7.7 & 4.1 & 0.6 & 0.9 & 2.4 & 1.4 \\
			a024\_3 & 229 & 519 & 0.9 & 1.6 & 2.0 & 2.0 & 1.0 & 1.7 & 13.8 & 10.7 & 1.0 & 1.6 & 4.7 & 3.0 \\
			a029\_1 & 353 & 421 & 1.4 & 2.0 & 9.0 & 5.4 & 0.7 & 1.0 & 7.4 & 4.9 & 0.6 & 0.9 & 3.1 & 1.5 \\
			a029\_2 & 301 & 523 & 1.6 & 2.7 & 3.0 & 2.5 & 1.5 & 2.6 & 7.9 & 6.8 & 1.4 & 2.5 & 1.4 & 1.5 \\
			a032\_1 & 394 & 428 & 0.4 & 0.6 & 1.2 & 0.9 & 0.5 & 0.7 & 9.1 & 5.0 & 0.5 & 0.6 & 3.1 & 2.2 \\
			a032\_3 & 541 & 693 & 0.4 & 0.6 & 2.5 & 1.6 & 0.5 & 0.8 & 5.4 & 5.8 & 0.4 & 0.7 & 2.1 & 1.9 \\
			a042\_2 & 412 & 717 & 1.2 & 1.9 & 5.6 & 1.7 & 1.2 & 1.9 & 12.9 & 8.8 & 1.1 & 1.9 & 2.6 & 1.8 \\
			a049\_1 & 297 & 347 & 0.7 & 1.3 & 4.0 & 3.3 & 0.8 & 1.3 & 10.7 & 5.4 & 0.7 & 1.2 & 3.8 & 3.6 \\
			平均值 & 456 & 571 & 1.9 & 3.1 & 5.3 & 3.4 & 1.3 & 1.9 & 10.2 & 6.0 & 1.2 & 1.8 & 3.5 & 2.0 \\	
			\bottomrule 
		\end{tabular*}
	}	
\end{frame}


% !TeX encoding = UTF-8
% !TeX root = ../whu-defense-qianlong.tex

%% ------------------------------------------------------------------------
%% Copyright (C) 2021-2023 SJTUG
%% 
%% SJTUBeamer Example Document by SJTUG
%% 
%% SJTUBeamer Example Document is licensed under a
%% Creative Commons Attribution-NonCommercial-ShareAlike 4.0 International License.
%% 
%% You should have received a copy of the license along with this
%% work. If not, see <http://creativecommons.org/licenses/by-nc-sa/4.0/>.
%% -----------------------------------------------------------------------

\subsection{研究内容2 数据驱动车载航迹推算方法研究}

\begin{frame}
	\frametitle{数据驱动车载航迹推算方法研究} 
   	\begin{figure}
   	\centering
   	    \includegraphics[width=0.9\textwidth]{VehicleDeadReckoningAlgorithmArchitecture.pdf}
   	\end{figure} 
\end{frame}

\begin{frame}
	\frametitle{数据驱动车载航迹推算方法研究}	
	\framesubtitle{状态定义}	
	\begin{equation*}
	   	\vb*{x}\left(t_n\right) := \left( \vb*{\chi}\left(t_n\right), \vb*{\theta}\left(t_n\right), \vb*{R}_s^v\left(t_n\right), \vb*{p}_s^v\left(t_n\right)\right) 
	   	\label{eq:InvariantExtendedKalmanFilterPedestrianState}
	\end{equation*} 
\end{frame}

\begin{frame}
	\frametitle{数据驱动车载航迹推算方法研究}	
	\framesubtitle{时间更新}
	\vspace{-0.2cm}
	{\small
		状态量时间更新
		\begin{align*}
			\vb*{R}\left(t_{n+1}\right) &= \vb*{R}\left(t_{n+1}\right) + \vb*{R}\left(t_{n}\right) \vb*{\Gamma}_{0}\left( \vb*{\omega}\left(t_{n}\right) \Delta t \right) \\
			\vb*{v}\left(t_{n+1}\right) &= \vb*{v}\left(t_{n}\right) 
			+ \vb*{R}\left(t_{n}\right) \vb*{\Gamma}_{1}\left( \vb*{\omega}\left(t_{n}\right) \Delta t \right) \vb*{f}\left(t_{n}\right) \Delta t 
			+ \vb*{g}\Delta t \\
			\vb*{p}\left(t_{n+1}\right) &= \vb*{p}\left(t_{n}\right) 
			+ \vb*{v}\left(t_{n}\right) \Delta t
			+ \vb*{R}\left(t_{n}\right) \vb*{\Gamma}_{2}\left( \vb*{\omega}\left(t_{n}\right) \Delta t \right) \vb*{f}\left(t_{n}\right) \left(\Delta t\right)^2 
			+ \frac{1}{2}\vb*{g}\left(\Delta t\right)^2 \\
			\vb*{\delta \omega}\left(t_{n+1}\right) &= \vb*{\delta \omega}\left(t_{n}\right) \\
			\vb*{\delta f}\left(t_{n+1}\right) &= \vb*{\delta f}\left(t_{n}\right) \\
			\vb*{R}_s^v\left(t_{n+1}\right) &= \vb*{R}_s^v\left(t_{n}\right) \\
			\vb*{p}_s^v\left(t_{n+1}\right) &= \vb*{p}_s^v\left(t_{n}\right) 
		\end{align*}
		
		状态量协方差时间更新
		\begin{equation*}
			\vb*{P}\left(t_{n+1}\right) = \vb*{F}\left(t_{n}\right) \vb*{P}\left(t_{n+1}\right) \vb*{F}^{\mathrm{T}}\left(t_{n}\right)
			+ \vb*{G}\left(t_{n}\right) \vb*{Q}\left(t_{n}\right) \vb*{G}^{\mathrm{T}}\left(t_{n}\right)
		\end{equation*}
	}
 
\end{frame}

\begin{frame}

	\frametitle{数据驱动车载航迹推算方法研究}	
	\framesubtitle{量测更新}
	
	{\small
		数据驱动观测量
		
		\begin{equation*}
			\label{eq:DataDrivenPedestrianMeasurementEstimator}
			\vb*{R}_{s}^{n}\left(t_{n+1}\right), v^{v}_{v, \mathrm{lon}}\left(t_{n+1}\right)
			=
			f_{\mathrm{Estimator}}\left(\{\tilde{\vb*{\omega}}\left(t_i\right), \tilde{\vb*{f}}\left(t_i\right)\}_{i = n-W_{\mathrm{est}}}^{n}\right)
		\end{equation*}
				
		观测方程
		
		\begin{center}
			\begin{tabular*}{\linewidth}{@{\extracolsep{\fill}}lll}
	            $ \tilde{\vb*{y}}_{\vb*{R}_s^n} = \tilde{\vb*{R}}_s^n $ 
	            & $ \hat{\vb*{y}}_{\vb*{R}_s^n} = \hat{\vb*{R}}_s^n $ 
	            & $ \vb*{r}_{\vb*{R}_s^n} = \log_{\mathrm{G}_{\mathrm{SO}\left(3\right)}}\left(\tilde{\vb*{y}}_{\vb*{R}_s^n}\left(\hat{\vb*{y}}_{\vb*{R}_s^n}\right)^{\mathrm{T}}\right) $ 
	            \\
	            $ \tilde{\vb*{y}}_{\vb*{v}_v^v} = 
	            \begin{pmatrix}
	           		0 & \tilde{v}_{\mathrm{lat}}^v & 0
	           	\end{pmatrix}^{\mathrm{T}} $ 
	            & $ \hat{\vb*{y}}_{\vb*{v}_v^v} = \hat{\vb*{R}}_s^v \left( \left(\hat{\vb*{R}}_s^n\right)^{\mathrm{T}} \hat{\vb*{v}}_s^n + \left(\vb*{\omega}^s\right)_{\times} \hat{\vb*{p}}_v^s \right) $ 
	            & $ \vb*{r}_{\vb*{v}_v^v} = \tilde{\vb*{y}}_{\vb*{v}_b^b} - \hat{\vb*{y}}_{\vb*{v}_b^b} $
	   		\end{tabular*}
   		\end{center}       
	}
 
\end{frame}

\begin{frame}

	\frametitle{数据驱动车载航迹推算方法研究}	
	\framesubtitle{量测更新}
	\vspace{-0.2cm}
	{\small
		状态量量测更新
	}
	{\scriptsize
		\begin{align*}
	    	\vb*{S}\left(t_{n+1}\right) &= \vb*{H}\left(t_{n+1}\right) \vb*{P}\left(t_{n+1}\right) \vb*{H}^{\mathrm{T}}\left(t_{n+1}\right) + \vb*{N}\left(t_{n+1}\right) \\
	    	\vb*{K}\left(t_{n+1}\right) &= \vb*{P}\left(t_{n+1}\right) \vb*{H}^{\mathrm{T}}\left(t_{n+1}\right) \vb*{S}^{-1}\left(t_{n+1}\right) \\
		    \vb*{e}\left(t_{n+1}\right) &= \vb*{H}\left(t_{n+1}\right) \vb*{r}\left(t_{n+1}\right) \\
			\vb*{\chi}\left(t_{n+1}^+\right)   &= \exp_{\mathrm{G}_{\mathrm{SE}_2\left(3\right)}}\left( \vb*{e}_{\vb*{\xi}_{\vb*{\chi}}}\left(t_{n+1}^+\right) \right) \vb*{\chi}\left(t_{n+1}\right)\\
			\vb*{\theta}\left(t_{n+1}^+\right) &= \vb*{\theta}\left(t_{n+1}\right) + \vb*{e}_{\vb*{\xi}_{\vb*{\theta}}}\left(t_{n+1}^+\right) \\
			\vb*{R}_s^v\left(t_{n+1}^+\right)  &= \exp_{\mathrm{G}_{\mathrm{SO}\left(3\right)}}\left( \vb*{e}_{\vb*{\xi}_{\vb*{R}_s^v}}\left(t_{n+1}^+\right) \right) \vb*{R}_s^vb\left(t_{n+1}\right) \\
			\vb*{p}_s^v\left(t_{n+1}^+\right)  &= \vb*{p}_s^v\left(t_{n+1}\right) + \vb*{e}_{\vb*{\xi}_{\vb*{p}_s^v}}\left(t_{n+1}^+\right)
		\end{align*}
	}
	{\small		
		状态量协方差量测更新
	}
	{\scriptsize
		\begin{equation*}
			\vb*{P}\left(t_{n+1}^+\right) = \left(\vb*{I} - \vb*{K}\left(t_{n+1}\right)\vb*{H}\left(t_{n+1}\right)\right) \vb*{P}\left(t_{n+1}\right) \left(\vb*{I} - \vb*{K}\left(t_{n+1}\right)\vb*{H}\left(t_{n+1}\right)\right)^{\mathrm{T}} + \vb*{K}\left(t_{n+1}\right)\vb*{N}\left(t_{n+1}\right) \vb*{K}^{\mathrm{T}}\left(t_{n+1}\right)
		\end{equation*}
	}
 
\end{frame}

\begin{frame} 
 	\frametitle{数据驱动车载航迹推算方法研究}
 	\framesubtitle{重庆停车场数据集}
   	\begin{figure}
   	\centering
	    \includegraphics[width=0.7\textwidth]{ChongqinDatasetOverview.png}
   	\end{figure}  
\end{frame}

\begin{frame} 
 	\frametitle{数据驱动车载航迹推算方法研究}
 	\framesubtitle{KITTI Odometry数据集}
   	\begin{figure}
   	\centering
	    \includegraphics[width=0.7\textwidth]{KITTIOdometryOverview.png}
   	\end{figure}  
\end{frame}

\begin{frame} 
 	\frametitle{数据驱动车载航迹推算方法研究}
 	\framesubtitle{重庆数据集万达广场(大渡口店)停车场航迹推算姿态误差}
    {\small
    \setlength{\tabcolsep}{2pt}
		\begin{tabular*}{\linewidth}{@{\extracolsep{\fill}}ccrrrrrrrrrrr}
			\toprule
			\multicolumn{2}{c}{轨迹} & 08 & 09 & 10 & 11 & 12 & 13 & 14 & 15 & 16 & 17 & 18 \\
			\midrule
			\multirow{4}{*}{窗口长度\SI{1}{\second}} 
			& $\vb*{e}^{\mathrm{AOE}}_{MAE}$ & 7.6 & \textbf{1.6} & 9.0 & \textbf{7.6} & \textbf{2.4} & \textbf{6.7} & \textbf{4.2} & \textbf{10.1} & \textbf{8.9} & \textbf{4.6} & \textbf{20.0} \\
			& $\vb*{e}^{\mathrm{AOE}}_{\mathrm{RMSE}}$ & 8.0 & \textbf{1.8} & 10.1 & \textbf{8.3} & \textbf{2.6} & \textbf{7.4} & \textbf{5.0} & \textbf{10.9} & \textbf{11.1} & \textbf{5.4} & \textbf{22.9} \\
			& $\vb*{e}^{\mathrm{ROE}}_{MAE}$ & 12.4 & \textbf{2.5} & 17.3 & \textbf{14.5} & \textbf{3.7} & \textbf{13.1} & \textbf{6.8} & \textbf{20.0} & \textbf{16.8} & \textbf{7.8} & \textbf{39.5} \\
			& $\vb*{e}^{\mathrm{ROE}}_{\mathrm{RMSE}}$ & 12.5 & \textbf{2.8} & 18.6 & \textbf{15.2} & \textbf{3.9} & \textbf{14.1} & \textbf{7.6} & \textbf{20.9} & 20.2 & \textbf{9.2} & \textbf{44.4} \\ \addlinespace[1mm]
			\multirow{4}{*}{窗口长度\SI{2}{\second}}
			& $\vb*{e}^{\mathrm{AOE}}_{MAE}$ & \textbf{4.1} & 3.8 & \textbf{6.1} & 8.7 & 25.5 & 17.5 & 11.3 & 71.1 & 10.3 & 12.0 & 33.5 \\
			& $\vb*{e}^{\mathrm{AOE}}_{\mathrm{RMSE}}$ & \textbf{5.1} & 4.5 & \textbf{6.9} & 10.0 & 31.1 & 19.8 & 13.0 & 85.9 & 11.6 & 13.9 & 38.5 \\
			& $\vb*{e}^{\mathrm{ROE}}_{MAE}$ & \textbf{4.0} & 6.4 & \textbf{11.8} & 15.2 & 48.4 & 33.7 & 22.1 & 99.4 & 16.9 & 21.7 & 65.6 \\
			& $\vb*{e}^{\mathrm{ROE}}_{\mathrm{RMSE}}$ & \textbf{4.8} & 6.9 & \textbf{12.6} & 17.1 & 55.5 & 36.2 & 24.7 & 112.6 & \textbf{19.7} & 25.2 & 74.5 \\
			\bottomrule 
		\end{tabular*}
	}	
\end{frame}

\begin{frame} 
 	\frametitle{数据驱动车载航迹推算方法研究}
 	\framesubtitle{基于重庆数据集对比输出状态量因素速度估计结果}
    {\small
    \setlength{\tabcolsep}{2pt}
		\begin{tabular*}{\linewidth}{@{\extracolsep{\fill}}clccccccccccc}
			\toprule
			\multicolumn{2}{c}{轨迹} & 08 & 09 & 10 & 11 & 12 & 13 & 14 & 15 & 16 & 17 & 18 \\
			\midrule
			\multirow{4}{*}{$\vb*{v}_y$} 
			& $\vb*{e}^{\mathrm{AVE}}_{MAE}$ 
			& 0.410 & 0.401 & 0.336 & 0.434 & \textbf{0.282} & \textbf{0.333} & \textbf{0.266} & \textbf{0.297} & \textbf{0.510} & \textbf{0.274} & \textbf{0.426} \\         
			& $\vb*{e}^{\mathrm{AVE}}_{\mathrm{RMSE}}$          
			& \textbf{0.491} & 0.492 & 0.424 & 0.533 & \textbf{0.358} & \textbf{0.415} & \textbf{0.357} & \textbf{0.374} & \textbf{0.679} & \textbf{0.346} & \textbf{0.527} \\ 
			& $\vb*{e}^{\mathrm{RVE}}_{MAE, \Delta t}$ 
			& \textbf{0.419} & 0.584 & 0.407 & 0.644 & \textbf{0.376} & 0.382 & \textbf{0.393} & \textbf{0.414} & \textbf{0.607} & \textbf{0.401} & \textbf{0.509} \\
			& $\vb*{e}^{\mathrm{RVE}}_{RMSE,\Delta t}$ 
			& \textbf{0.510} & 0.753 & 0.487 & 0.824 & \textbf{0.463} & 0.531 & \textbf{0.522} & \textbf{0.515} & \textbf{0.777} & \textbf{0.507} & \textbf{0.646} \\ \addlinespace[1mm]
			\multirow{4}{*}{$\Delta\vb*{v}_y$} 
			& $\vb*{e}^{\mathrm{AVE}}_{MAE}$ 
			& 1.884 & 0.689 & \textbf{0.241} & 1.451 & 0.481 & 0.605 & 0.942 & 1.339 & 1.573 & 1.374 & 1.326 \\         
			& $\vb*{e}^{\mathrm{AVE}}_{\mathrm{RMSE}}$         
			& 2.155 & 0.883 & 0.406 & 1.596 & 0.680 & 0.673 & 1.065 & 1.427 & 1.848 & 1.519 & 1.646 \\
			& $\vb*{e}^{\mathrm{RVE}}_{MAE, \Delta t}$ 
			& 0.927 & \textbf{0.526} & \textbf{0.227} & 1.259 & 0.503 & \textbf{0.216} & 0.707 & 0.770 & 1.243 & 2.197 & 2.145 \\
			& $\vb*{e}^{\mathrm{RVE}}_{RMSE,\Delta t}$ 
			& 1.326 & 0.847 & \textbf{0.390} & 1.676 & 0.760 & \textbf{0.344} & 0.959 & 1.036 & 1.594 & 2.528 & 2.475 \\
			\bottomrule 
		\end{tabular*}
	}	
\end{frame}

\begin{frame} 
 	\frametitle{数据驱动车载航迹推算方法研究}
 	\framesubtitle{基于重庆数据集对比输入维度因素速度估计误差}
    {\footnotesize
    \setlength{\tabcolsep}{2pt}
		\begin{tabular*}{\linewidth}{@{\extracolsep{\fill}}clccccccccccc}
			\toprule
			\multicolumn{2}{c}{轨迹} & 08 & 09 & 10 & 11 & 12 & 13 & 14 & 15 & 16 & 17 & 18 \\
			\midrule
			\multirow{4}{*}{\makecell{$\left( \vb*{\omega}_s^s, \vb*{f}_s^s, P_s^s \right)$\\$\mathbb{R}^7$}}
			& $\vb*{e}^{\mathrm{AVE}}_{MAE}$ 
			& 0.410 & 0.401 & 0.336 & 0.434 & \textbf{0.282} & \textbf{0.333} & \textbf{0.266} & \textbf{0.297} & \textbf{0.510} & \textbf{0.274} & \textbf{0.426} \\         
			& $\vb*{e}^{\mathrm{AVE}}_{\mathrm{RMSE}}$          
			& \textbf{0.491} & 0.492 & 0.424 & 0.533 & \textbf{0.358} & \textbf{0.415} & \textbf{0.357} & \textbf{0.374} & \textbf{0.679} & \textbf{0.346} & \textbf{0.527} \\ 
			& $\vb*{e}^{\mathrm{RVE}}_{MAE, \Delta t}$ 
			& \textbf{0.419} & 0.584 & 0.407 & 0.644 & \textbf{0.376} & 0.382 & \textbf{0.393} & \textbf{0.414} & \textbf{0.607} & \textbf{0.401} & \textbf{0.509} \\
			& $\vb*{e}^{\mathrm{RVE}}_{RMSE,\Delta t}$ 
			& \textbf{0.510} & 0.753 & 0.487 & 0.824 & \textbf{0.463} & 0.531 & \textbf{0.522} & \textbf{0.515} & \textbf{0.777} & \textbf{0.507} & \textbf{0.646} \\ \addlinespace[1mm]
			\multirow{4}{*}{\makecell{$\left( \vb*{\omega}_s^s, \vb*{f}_s^s\right)$\\$\mathbb{R}^6$}} 
			& $\vb*{e}^{\mathrm{AVE}}_{MAE}$ 
			& 0.208 & 0.315 & 0.325 & 0.381 & 0.232 & 0.141 & 0.241 & 0.290 & 0.481 & 0.245 & 0.383 \\       
			& $\vb*{e}^{\mathrm{AVE}}_{\mathrm{RMSE}}$         
			& 0.273 & 0.377 & 0.404 & 0.502 & 0.294 & 0.172 & 0.327 & 0.367 & 0.630 & 0.328 & 0.508 \\
			& $\vb*{e}^{\mathrm{RVE}}_{MAE, \Delta t}$ 
			& 0.303 & 0.483 & 0.439 & 0.581 & 0.336 & 0.121 & 0.361 & 0.413 & 0.623 & 0.353 & 0.540 \\
			& $\vb*{e}^{\mathrm{RVE}}_{RMSE,\Delta t}$ 
			& 0.369 & 0.598 & 0.540 & 0.772 & 0.418 & 0.154 & 0.467 & 0.522 & 0.809 & 0.449 & 0.713 \\ \addlinespace[1mm]
			\multirow{4}{*}{\makecell{$\left( \vb*{\omega}_s^s, \vb*{f}_s^s, \vb*{M}_s^s\right)$\\$\mathbb{R}^9$}} 
			& $\vb*{e}^{\mathrm{AVE}}_{MAE}$ 
			& \textbf{0.365} & \textbf{0.351} & 0.288 & \textbf{0.416} & 0.341 & 0.360 & 0.279 & 0.362 & 0.591 & 0.378 & 0.530 \\
			& $\vb*{e}^{\mathrm{AVE}}_{\mathrm{RMSE}}$          
			& 0.504 & \textbf{0.430} & \textbf{0.361} & \textbf{0.517} & 0.435 & 0.461 & 0.378 & 0.471 & 0.744 & 0.502 & 0.666 \\
			& $\vb*{e}^{\mathrm{RVE}}_{MAE, \Delta t}$ 
			& 0.509 & 0.542 & 0.390 & \textbf{0.614} & 0.449 & 0.449 & 0.433 & 0.534 & 0.738 & 0.566 & 0.644 \\ 
			& $\vb*{e}^{\mathrm{RVE}}_{RMSE,\Delta t}$ 
			& 0.626 & \textbf{0.662} & 0.506 & \textbf{0.787} & 0.558 & 0.565 & 0.571 & 0.663 & 0.965 & 0.711 & 0.820 \\
			\bottomrule 
		\end{tabular*}
	}	
\end{frame}

\begin{frame} 
 	\frametitle{数据驱动车载航迹推算方法研究}
 	\framesubtitle{基于重庆数据集对比输入采样率因素速度估计结果}
    {\footnotesize
    \setlength{\tabcolsep}{2pt}
		\begin{tabular*}{\linewidth}{@{\extracolsep{\fill}}clccccccccccc}
			\toprule
			\multicolumn{2}{c}{轨迹} & 08 & 09 & 10 & 11 & 12 & 13 & 14 & 15 & 16 & 17 & 18 \\
			\midrule
			\multirow{2}{*}{\SI{50}{Hz}} 
			& $\vb*{e}^{\mathrm{AVE}}_{MAE}$ 
			& 0.410 & 0.401 & 0.336 & 0.434 & 0.282 & 0.333 & 0.266 & 0.297 & 0.510 & 0.274 & 0.426 \\
			& $\vb*{e}^{\mathrm{RVE}}_{MAE, \Delta t}$ 
			& 0.419 & 0.584 & 0.407 & 0.644 & 0.376 & 0.382 & 0.393 & 0.414 & \textbf{0.607} & 0.401 & \textbf{0.509} \\ \addlinespace[1mm]
			\multirow{2}{*}{\SI{100}{Hz}} 
			& $\vb*{e}^{\mathrm{AVE}}_{MAE}$ 
			& \textbf{0.277} & \textbf{0.308} & \textbf{0.274} & \textbf{0.361} & \textbf{0.206} & \textbf{0.237} & \textbf{0.225} & 0.277 & 0.480 & \textbf{0.229} & 0.411 \\ 
			& $\vb*{e}^{\mathrm{RVE}}_{MAE, \Delta t}$ 
			& 0.411 & \textbf{0.448} & \textbf{0.382} & 0.587 & 0.309 & \textbf{0.302} & \textbf{0.319} & 0.391 & 0.640 & \textbf{0.345} & 0.573 \\ \addlinespace[1mm]
			\multirow{2}{*}{\SI{150}{Hz}} 
			& $\vb*{e}^{\mathrm{AVE}}_{MAE}$ 
			& 0.289 & 0.348 & 0.281 & 0.362 & 0.212 & 0.273 & 0.238 & \textbf{0.252} & \textbf{0.452} & 0.276 & \textbf{0.391} \\
			& $\vb*{e}^{\mathrm{RVE}}_{MAE, \Delta t}$ 
			& 0.412 & 0.485 & 0.403 & \textbf{0.574} & \textbf{0.301} & 0.359 & 0.339 & \textbf{0.351} & 0.625 & 0.395 & 0.493 \\ \addlinespace[1mm]
			\multirow{2}{*}{\SI{200}{Hz}} 
			& $\vb*{e}^{\mathrm{AVE}}_{MAE}$ 
			& 0.285 & 0.311 & 0.322 & 0.466 & 0.222 & 0.284 & 0.261 & 0.261 & 0.457 & 0.276 & 0.394 \\         
			& $\vb*{e}^{\mathrm{RVE}}_{MAE, \Delta t}$ 
			& \textbf{0.401} & 0.488 & 0.474 & 0.673 & 0.312 & 0.374 & 0.358 & 0.370 & 0.647 & 0.395 & 0.527 \\
			\bottomrule 
		\end{tabular*}
	}	
\end{frame}

\begin{frame} 
 	\frametitle{数据驱动车载航迹推算方法研究}
 	\framesubtitle{重庆数据集万达广场(大渡口店)停车场航迹推算位置误差结果}
    {\footnotesize
    \setlength{\tabcolsep}{2pt}
		\begin{tabular*}{\linewidth}{@{\extracolsep{\fill}}clrrrrrrrrrrr}
			\toprule
			\multicolumn{2}{c}{轨迹} & 08 & 09 & 10 & 11 & 12 & 13 & 14 & 15 & 16 & 17 & 18 \\
			\midrule
			\multicolumn{2}{c}{里程$\left(\unit{m}\right)$} & 282  & 285  & 285  & 520  & 525  & 524  & 844  & 1556 & 593   & 839  & 1553   \\
			\multicolumn{2}{c}{时长$\left(\unit{s}\right)$} & 124  & 122  & 208  & 168  & 216  & 370  & 316  & 529  & 459   & 418  & 807    \\
			\multirow{2}{*}{AI-IMU} & $\vb*{e}^{\mathrm{RPE}}_{MAE,unit \Delta l}$ & 15.9 & 24.1 & 65.2 & 11.3 & 26.0 & 1683.8 & 20.8 & 12.6 & 18383.1 & 4829.5 & 176752.2 \\
			& $\vb*{e}^{\mathrm{RPE,horizontal}}_{MAE,unit \Delta l}$ & 15.8 & 24.1 & 65.0 & 11.0 & 25.8 & 1656.0 & 20.7 & 12.4 & 6380.6 & 4098.8 & 36066.2 \\
			\multirow{2}{*}{DeepOdo} & $\vb*{e}^{\mathrm{RPE}}_{MAE,unit \Delta l}$ & 5.0 & 9.8 & 18.4 & 13.3 & 9.0 & 22.3 & 9.4 & 10.1 & 41.1 & 13.9 & 27.4 \\
			& $\vb*{e}^{\mathrm{RPE,horizontal}}_{MAE,unit \Delta l}$ & 4.7 & 9.7 & 18.4 & 13.2 & 9.0 & 22.3 & 9.3 & 10.0 & 41.0 & 13.6 & 27.2 \\
			\multirow{2}{*}{DeepOri} & $\vb*{e}^{\mathrm{RPE}}_{MAE,unit \Delta l}$ & 5.1 & 1.9 & 8.5 & 6.3 & 2.4 & 3.5 & 2.6 & 5.2 & 5.0 & 3.7 & 7.8 \\
			& $\vb*{e}^{\mathrm{RPE,horizontal}}_{MAE,unit \Delta l}$ & 4.5 & 1.4 & 8.4 & 5.2 & 1.7 & 3.4 & 1.4 & 2.7 & 4.2 & 2.2 & 6.0 \\
			\multirow{2}{*}{\textbf{Proposed}} & $\vb*{e}^{\mathrm{RPE}}_{MAE,unit \Delta l}$ & \textbf{2.1} & \textbf{1.2} & \textbf{0.9} & \textbf{2.8} & \textbf{1.4} & \textbf{0.8} & \textbf{1.9} & \textbf{3.3} & \textbf{2.8} & \textbf{2.9} & \textbf{3.4} \\
			& $\vb*{e}^{\mathrm{RPE,horizontal}}_{MAE,unit \Delta l}$ & \textbf{0.4} & \textbf{0.4} & \textbf{0.4} & \textbf{0.4} & \textbf{0.3} & \textbf{0.3} & \textbf{0.3} & \textbf{0.3} & \textbf{0.5} & \textbf{0.6} & \textbf{0.5} \\
			\bottomrule 
		\end{tabular*}
	}	
\end{frame}

\begin{frame} 
 	\frametitle{数据驱动车载航迹推算方法研究}
 	\framesubtitle{KITTI Odometry数据集航迹推算位置误差结果}
    {\footnotesize
    \setlength{\tabcolsep}{2pt}
		\begin{tabular*}{\linewidth}{@{\extracolsep{\fill}}ccrrrrrrrrrr}
			\toprule
            \multicolumn{2}{c}{轨迹} & 00 & 01 & 02 & 04 & 05 & 06 & 07 & 08 & 09 & 10 \\
			\midrule
			\multicolumn{2}{c}{里程$\left(\unit{m}\right)$} & 3722 & 2603 & 5089 & 416 & 2214 & 1245 & 693 & 4245 & 1715 & 922 \\
			\multicolumn{2}{c}{时长$\left(\unit{s}\right)$} &  471 &  121 &  483 &  29 &  288 &  115 & 114 &  537 &  165 & 126 \\
			\multirow{2}{*}{AI-IMU}            & $\vb*{e}^{\mathrm{RPE}}_{MAE,unit \Delta l}$ & 11.8 & 4.0 & 4.5 & 7.0 & 19.1 & 3.5 & 1.8 & 3.2 & 7.4 & 4.0 \\
			                                   & $\vb*{e}^{\mathrm{RPE,horizontal}}_{MAE,unit \Delta l}$ & 11.4 & 3.6 & 2.5 & 6.7 & 18.8 & 2.9 & 1.2 & 1.9 & 5.4 & 2.0 \\
			\multirow{2}{*}{DeepOdo}           & $\vb*{e}^{\mathrm{RPE}}_{MAE,unit \Delta l}$ & 7.4 & 5.0 & 3.9 & 2.6 & 3.7 & 2.3 & 1.7 & 4.8 & 5.2 & 5.0 \\
			                                   & $\vb*{e}^{\mathrm{RPE,horizontal}}_{MAE,unit \Delta l}$ & 6.8 & 4.0 & 1.3 & 1.3 & 3.3 & 1.5 & 0.9 & 3.7 & 2.7 & 3.0 \\
			\multirow{2}{*}{DeepOri}           & $\vb*{e}^{\mathrm{RPE}}_{MAE,unit \Delta l}$ & 4.1 & 4.4 & 4.4 & 9.8 & 2.1 & 3.6 & 1.7 & \textbf{3.0} & 5.2 & 3.8 \\
			                                   & $\vb*{e}^{\mathrm{RPE,horizontal}}_{MAE,unit \Delta l}$ & 3.3 & 3.5 & 2.3 & 9.8 & 1.6 & 3.1 & 0.9 & \textbf{1.7} & 2.8 & 1.8 \\
			\multirow{2}{*}{\textbf{Proposed}} & $\vb*{e}^{\mathrm{RPE}}_{MAE,unit \Delta l}$ & \textbf{2.1} & \textbf{3.2} & \textbf{3.3} & \textbf{2.0} & \textbf{1.3} & \textbf{1.4} & \textbf{1.5} & 8.2 & \textbf{4.1} & \textbf{3.7} \\
			                                   & $\vb*{e}^{\mathrm{RPE,horizontal}}_{MAE,unit \Delta l}$ & \textbf{1.2} & \textbf{1.9} & \textbf{1.0} & \textbf{0.4} & \textbf{0.4} & \textbf{0.5} & \textbf{0.7} & 4.1 & \textbf{1.2} & \textbf{1.1} \\
			\bottomrule 
		\end{tabular*}
	}	
\end{frame}


% !TeX encoding = UTF-8
% !TeX root = ../whu-defense-qianlong.tex

%% ------------------------------------------------------------------------
%% Copyright (C) 2021-2023 SJTUG
%% 
%% SJTUBeamer Example Document by SJTUG
%% 
%% SJTUBeamer Example Document is licensed under a
%% Creative Commons Attribution-NonCommercial-ShareAlike 4.0 International License.
%% 
%% You should have received a copy of the license along with this
%% work. If not, see <http://creativecommons.org/licenses/by-nc-sa/4.0/>.
%% -----------------------------------------------------------------------

\subsection{研究内容3 数据和模型双驱动定位方法研究}

\begin{frame}[t]
	\frametitle{数据和模型双驱动定位方法研究}	
	\begin{columns}[t]
		\begin{column}{0.5\textwidth}
		    \begin{block}{存在问题}
		    {
		    	\small
		        \begin{itemize}
					\item 单一模型驱动方法定位频率受限
		        \end{itemize}
    		 } 
			\end{block}
			\begin{block}{研究思路}
			 {
 		    	\small
				\begin{itemize}
					\item 数据和模型双驱动保持定位精度和鲁棒性
				\end{itemize}
			}
			\end{block}
		\end{column}   
		\begin{column}{0.5\textwidth}
		   	\begin{figure}
		   	\centering
		   	    \includegraphics[width=\textwidth]{AudioPositioningSystem.pdf}
		   	\end{figure}
		\end{column}
	\end{columns}
\end{frame}

\begin{frame}[t]
	\frametitle{数据和模型双驱动定位方法研究}
	\framesubtitle{模型驱动几何观测量}	
	\begin{columns}[t]
		\begin{column}{0.5\textwidth}
		    \vspace{-1.0cm}
			\begin{block}{模型驱动音频 TDoA 观测量}
			{
				\tiny
				\begin{equation*}
					\Delta t_{ji}\left(t_{n+1}\right)
					=
					f_{\mathrm{Acoustic}}\left(\tilde{\vb*{s}}_{\mathrm{Sound}},\tilde{\vb*{s}}_{\mathrm{BLE}}\left(t_{n+1}\right)\right)
				\end{equation*}
			}
				观测方程
			{
				\tiny	
				\begin{equation*}
					\vb*{H}^{\Delta r_{ji}} 
					= 
					\begin{pmatrix}
						\vb*{H}^{\Delta r_{ji}}_{\vb*{R}_s^w}
						~\vb*{H}^{\Delta r_{ji}}_{\vb*{v}_s^w}
						~\vb*{H}^{\Delta r_{ji}}_{\vb*{p}_s^w}
						~\vb*{H}^{\Delta r_{ji}}_{\vb*{\delta}_{\vb*{\omega}_s^s}} 
						~\vb*{H}^{\Delta r_{ji}}_{\vb*{\delta}_{\vb*{f}_s^s}} 
					\end{pmatrix}
				\end{equation*}		
		   	}
		   	\end{block}
		   	\vspace{1.5cm}
		   	{
				\tiny
				\begin{center}
					\setlength{\tabcolsep}{2pt}
					\begin{tabular*}{\linewidth}{@{\extracolsep{\fill}}lll}
						$ \vb*{H}^{\Delta r_{ji}} _{\vb*{R}_s^w}
						=
							       			\left(\frac{\vb*{P}_j-\hat{\vb*{p}}}{\left|\vb*{P}_j-\hat{\vb*{p}}\right|} - \frac{\vb*{P}_i-\hat{\vb*{p}}}{\left|\vb*{P}_i-\hat{\vb*{p}}\right|}\right)^{\mathrm{T}} \left(\hat{\vb*{p}}\right)_{\times}  $ 
						& $ \vb*{H}^{\Delta r_{ji}} _{\vb*{v}_s^w} = \vb*{0}_{1 \times 3} $ 
						& $ \vb*{H}^{\Delta r_{ji}} _{\vb*{p}_s^w}
							       			=
							       			-\left(\frac{\vb*{P}_i-\hat{\vb*{p}}}{\left|\vb*{P}_i-\hat{\vb*{p}}\right|}-\frac{\vb*{P}_j-\hat{\vb*{p}}}{\left|\vb*{P}_j-\hat{\vb*{p}}\right|}\right)^{\mathrm{T}} $
						\\
						$ \vb*{H}^{\Delta r_{ji}} _{\vb*{\delta}_{\vb*{\omega}_s^s}} = \vb*{0}_{1 \times 3} $ 
						& $ \vb*{H}^{\Delta r_{ji}}_{\vb*{\delta}_{\vb*{f}_s^s}} = \vb*{0}_{1 \times 3} $ 
						& 
			   		\end{tabular*}
		   		\end{center} 
			}
		\end{column}   
		\begin{column}{0.5\textwidth}
		    \vspace{-1.5cm}
		   	\begin{figure}
		   	\centering
		   	    \includegraphics[width=\textwidth]{PedestrianAcousticPositioningAlgorithmArchitecture.pdf}
		   	\end{figure}
		\end{column}
	\end{columns} 
\end{frame}
 
\begin{frame} 
 	\frametitle{数据和模型双驱动定位方法研究}
 	\framesubtitle{南京南高铁站数据集}
	\begin{columns}[t]
		\begin{column}{0.5\textwidth}
		   	\begin{figure}
			    \includegraphics[width=\textwidth]{NanjingNJNRSDatasetTrackOverview.png}
		   	\end{figure}  
		\end{column}   
		\begin{column}{0.5\textwidth}
		   	\begin{figure}
		   	\centering
		   	    \includegraphics[width=\textwidth]{NanjingNJNRSDatasetAcousticSignalZoomedMap.png}
		   	\end{figure}
		\end{column}
	\end{columns}
\end{frame}
 
\begin{frame} 
 	\frametitle{数据和模型双驱动定位方法研究}
 	\framesubtitle{南京南高铁站数据集不同定位方法评价结果}
	\begin{columns}[t]
		\begin{column}{0.5\textwidth}
	    \scalebox{0.8}{{
	    	\tiny
	    	\setlength{\tabcolsep}{2pt}
			\begin{tabular*}{1.2\textwidth}{@{\extracolsep{\fill}}llrrrrrr}
				\toprule
	            \multicolumn{2}{c}{轨迹} & L1 & L2 & Z1 & Z2 & Z3 & Dataset \\
				\midrule
				\multicolumn{2}{c}{参考里程$\left(\unit{m}\right)$} & 737 & 737 & 593 & 590 & 590 & 3247 \\
				\midrule
				\multicolumn{2}{c}{参考时长$\left(\unit{s}\right)$} & 566 & 548 & 508 & 486 & 461 & 2569 \\
				\midrule
				\multirow{2}{*}{\makecell[c]{MD (TDoA)\\ LS}}
				& $\vb*{e}^{\mathrm{APE,horizontal}}_{MAE}$ & 0.422 & 0.284 & 0.673 & 1.149 & 0.444 & 0.584 \\
				& $\vb*{e}^{\mathrm{APE,horizontal}}_{\SI{95}{\percent}}$ & 0.905 & 0.482 & 1.561 & 2.540 & 1.113 & 1.215 \\
				\midrule
				\multirow{2}{*}{\makecell[l]{MD (TDoA)\\ Classic EKF}}
				& $\vb*{e}^{\mathrm{APE,horizontal}}_{MAE}$ & 0.639 & 0.337 & 1.508 & 1.025 & 0.562 & 0.806 \\
				& $\vb*{e}^{\mathrm{APE,horizontal}}_{\SI{95}{\percent}}$ & 1.433 & 0.717 & 2.967 & 2.615 & 1.410 & 1.882 \\
				\midrule
				\multirow{2}{*}{\makecell[l]{MD (TDoA)\\ Invariant EKF}}
				& $\vb*{e}^{\mathrm{APE,horizontal}}_{MAE}$ & 0.318 & 0.213 & 0.574 & 0.869 & 0.348 & 0.456 \\
				& $\vb*{e}^{\mathrm{APE,horizontal}}_{\SI{95}{\percent}}$ & \textbf{0.675} & 0.439 & 1.184 & 2.133 & 0.788 & 0.887 \\
				\midrule
				\multirow{2}{*}{\makecell[l]{DD ($\vb*{R}$) MD (TDoA)\\ Invariant EKF}}
				& $\vb*{e}^{\mathrm{APE,horizontal}}_{MAE}$ & 0.370 & 0.208 & 0.494 & 0.611 & \textbf{0.312} & 0.395 \\
				& $\vb*{e}^{\mathrm{APE,horizontal}}_{\SI{95}{\percent}}$ & 0.829 & 0.384 & 0.987 & 1.193 & \textbf{0.711} & 0.812 \\
				\midrule
				\multirow{2}{*}{\makecell[l]{DD ($\vb*{R}$, $\vb*{v}$) MD (TDoA)\\ Invariant EKF}}
				& $\vb*{e}^{\mathrm{APE,horizontal}}_{MAE}$ & \textbf{0.316} & \textbf{0.198} & \textbf{0.433} & \textbf{0.539} & 0.368 & \textbf{0.366} \\
				& $\vb*{e}^{\mathrm{APE,horizontal}}_{\SI{95}{\percent}}$ & 0.690 & \textbf{0.364} & \textbf{0.970} & \textbf{1.014} & 0.807 & \textbf{0.785} \\
				\bottomrule 
			\end{tabular*}	
		}}
		\end{column}   
		\begin{column}{0.5\textwidth}
		    \vspace{-4.5cm}
			\begin{columns}[t]
					\begin{column}{0.3\textwidth}
					   	\begin{figure}
						    \includegraphics[height=0.9\textheight]{NanjingNJNRSL1TrackMethodComparation.png}
					   	\end{figure}
					   	\vspace{-0.5cm}
		   			   	{\tiny L1 轨迹位置估计结果}
					\end{column}   
					\begin{column}{0.3\textwidth}
					   	\begin{figure}
					   	\centering
					   	    \includegraphics[height=0.9\textheight]{NanjingNJNRSZ1TrackMethodComparation.png}
					   	\end{figure}
					   	\vspace{-0.5cm}
		   			   	{\tiny Z1 轨迹位置估计结果}
					\end{column}
					\begin{column}{0.4\textwidth}
					    \vspace{1cm}
						{\small 						
							\begin{itemize}
								\item 改进滤波模型可以提升位置估计精度
								\item 增加动态位姿约束可以提升位置估计精度
							\end{itemize}
						}
					\end{column}
				\end{columns}
		\end{column}
	\end{columns}
\end{frame}
 
\begin{frame}
 	\frametitle{数据和模型双驱动定位方法研究}
 	\framesubtitle{南京南高铁站数据集不同 TDoA 组合频率结果对比}
	\begin{columns}[t]
		\begin{column}{0.5\textwidth}
	    {
	    	\tiny
	    	\setlength{\tabcolsep}{2pt}
			\begin{tabular*}{\linewidth}{@{\extracolsep{\fill}}rlrrrrrr}
				\toprule
			    \multicolumn{2}{c}{轨迹} & L1 & L2 & Z1 & Z2 & Z3 & Dataset \\
				\midrule
				\multicolumn{2}{c}{参考里程$\left(\unit{m}\right)$} & 737 & 737 & 593 & 590 & 590 & 3247 \\
				\midrule
				\multicolumn{2}{c}{参考时长$\left(\unit{s}\right)$} & 566 & 548 & 508 & 486 & 461 & 2569 \\
				\midrule
				\multirow{1}{*}{$\SI{1}{\hertz}$}
				& $\vb*{e}^{\mathrm{APE,horizontal}}_{MAE}$ & \textbf{0.316} & \textbf{0.198} & \textbf{0.433} & \textbf{0.539} & \textbf{0.368} & \textbf{0.366} \\
				\midrule
				\multirow{1}{*}{$\SI{0.5}{\hertz}$}
				& $\vb*{e}^{\mathrm{APE,horizontal}}_{MAE}$ & 0.351 & 0.227 & 0.518 & 0.733 & 0.464 & 0.450 \\
				\midrule
				\multirow{1}{*}{$\SI{0.3}{\hertz}$}
				& $\vb*{e}^{\mathrm{APE,horizontal}}_{MAE}$ & 0.374 & 0.263 & 0.690 & 0.674 & 0.519 & 0.495 \\
				\midrule
				\multirow{1}{*}{$\SI{0.2}{\hertz}$}
				& $\vb*{e}^{\mathrm{APE,horizontal}}_{MAE}$ & 0.438 & 0.330 & 0.740 & 0.916 & 0.723 & 0.616 \\		
				\bottomrule 
			\end{tabular*}	
		}
		\end{column}   
		\begin{column}{0.5\textwidth}
		    \vspace{-2.5cm}
			\begin{columns}[t]
					\begin{column}{0.3\textwidth}
					   	\begin{figure}
						    \includegraphics[height=0.9\textheight]{NanjingNJNRSL2TrackTDoAIntegratedFrequencyComparation.png}
					   	\end{figure}
					   	\vspace{-0.5cm}
		   			   	\hspace{0.0cm} {\tiny L2 轨迹位置估计结果}
					\end{column}   
					\begin{column}{0.3\textwidth}
					   	\begin{figure}
					   	\centering
					   	    \includegraphics[height=0.9\textheight]{NanjingNJNRSZ2TrackTDoAIntegratedFrequencyComparation.png}
					   	\end{figure}
					   	\vspace{-0.5cm}
		   			   	\hspace{0.0cm} {\tiny Z2 轨迹位置估计结果}
					\end{column}
					\begin{column}{0.4\textwidth}
					    \vspace{1cm}
						{\small 						
							\begin{itemize}
								\item 降低模型驱动高精度定位信号更新频率对于整体精度影响较小
							\end{itemize}
						}
					\end{column}
				\end{columns}
		\end{column}
	\end{columns}
\end{frame}

\begin{frame}
 	\frametitle{数据和模型双驱动定位方法研究}
 	\framesubtitle{南京南高铁站数据集不同智能手机结果对比}
	\begin{columns}[t]
		\begin{column}{0.5\textwidth}
	    \scalebox{0.8}{{
	    	\tiny
	    	\setlength{\tabcolsep}{2pt}
			\begin{tabular*}{1.2\linewidth}{@{\extracolsep{\fill}}llrrrrrr}
				\toprule
				\multicolumn{2}{c}{轨迹} & L1 & L2 & Z1 & Z2 & Z3 & Dataset \\
				\midrule
				\multicolumn{2}{c}{参考里程$\left(\unit{m}\right)$} & 737 & 737 & 593 & 590 & 590 & 3247 \\
				\midrule
				\multicolumn{2}{c}{参考时长$\left(\unit{s}\right)$} & 566 & 548 & 508 & 486 & 461 & 2569 \\
				\midrule
				\multirow{1}{*}{\makecell[l]{HUAWEI Mate 30\\ LS}}
				& $\vb*{e}^{\mathrm{APE,horizontal}}_{MAE}$ & 0.675 & 0.607 & 2.586 & 1.397 & 1.365 & 1.276 \\
				& $\vb*{e}^{\mathrm{APE,horizontal}}_{\SI{95}{\percent}}$ & 1.500 & 1.349 & 4.687 & 3.534 & 3.168 & 2.722 \\
				\midrule
				\multirow{1}{*}{\makecell[l]{HUAWEI Mate 30\\ Invariant EKF}}
				& $\vb*{e}^{\mathrm{APE,horizontal}}_{MAE}$ & \textbf{0.184} & 0.199 & 0.865 & 0.611 & 0.485 & 0.447 \\
				& $\vb*{e}^{\mathrm{APE,horizontal}}_{\SI{95}{\percent}}$ & \textbf{0.375} & 0.367 & 2.053 & 1.254 & 1.268 & 0.926 \\
				\midrule
				\multirow{1}{*}{\makecell[l]{HUAWEI Nova 7\\ LS}}
				& $\vb*{e}^{\mathrm{APE,horizontal}}_{MAE}$ & 0.767 & 0.771 & 3.492 & 1.448 & 1.096 & 1.467 \\
				& $\vb*{e}^{\mathrm{APE,horizontal}}_{\SI{95}{\percent}}$ & 1.724 & 2.034 & 4.731 & 3.406 & 2.213 & 2.918 \\
				\midrule
				\multirow{1}{*}{\makecell[l]{HUAWEI Nova 7\\ Invariant EKF}}
				& $\vb*{e}^{\mathrm{APE,horizontal}}_{MAE}$ & 0.220 & 0.206 & 0.694 & \textbf{0.495} & 0.516 & 0.409 \\
				& $\vb*{e}^{\mathrm{APE,horizontal}}_{\SI{95}{\percent}}$ & 0.480 & 0.406 & 1.889 & 1.054 & 1.036 & 0.860 \\
				\midrule
				\multirow{1}{*}{\makecell[l]{SAMSUNG Galaxy S9\\ LS}}
				& $\vb*{e}^{\mathrm{APE,horizontal}}_{MAE}$ & 0.422 & 0.284 & 0.673 & 1.149 & 0.444 & 0.584 \\
				& $\vb*{e}^{\mathrm{APE,horizontal}}_{\SI{95}{\percent}}$ & 0.905 & 0.482 & 1.561 & 2.540 & 1.113 & 1.215 \\
				\midrule
				\multirow{1}{*}{\makecell[l]{SAMSUNG Galaxy S9\\ Invariant EKF}}
				& $\vb*{e}^{\mathrm{APE,horizontal}}_{MAE}$ & 0.316 & \textbf{0.198} & \textbf{0.433} & 0.539 & \textbf{0.368} & \textbf{0.366} \\
				& $\vb*{e}^{\mathrm{APE,horizontal}}_{\SI{95}{\percent}}$ & 0.690 & \textbf{0.364} & \textbf{0.970} & \textbf{1.014} & \textbf{0.807} & \textbf{0.785} \\
				\bottomrule 
			\end{tabular*} 	
		}}
		\end{column}   
		\begin{column}{0.5\textwidth}
		    \vspace{-4.5cm}
			\begin{columns}[t]
					\begin{column}{0.3\textwidth}
					   	\begin{figure}
						    \includegraphics[height=0.9\textheight]{NanjingNJNRSL1TrackSmartphoneComparation.png}
					   	\end{figure}
					   	\vspace{-0.5cm}
		   			   	\hspace{0.0cm} {\tiny L2 轨迹位置估计结果}
					\end{column}   
					\begin{column}{0.3\textwidth}
					   	\begin{figure}
					   	\centering
					   	    \includegraphics[height=0.9\textheight]{NanjingNJNRSZ3TrackSmartphoneComparation.png}
					   	\end{figure}
					   	\vspace{-0.5cm}
		   			   	\hspace{0.0cm} {\tiny Z2 轨迹位置估计结果}
					\end{column}
					\begin{column}{0.4\textwidth}
					    \vspace{1cm}
						{\small 						
							\begin{itemize}
								\item 维持了模型的泛化性
							\end{itemize}
						}
					\end{column}
				\end{columns}
		\end{column}
	\end{columns}
\end{frame}

\begin{frame}
	\frametitle{数据和模型双驱动定位方法研究}
	\framesubtitle{模型驱动位置观测量}
	\begin{columns}[t]
		\begin{column}{0.5\textwidth}
		    \vspace{-1.0cm}
			\begin{block}{模型驱动 GNSS 观测量}
			{
				\tiny
				\begin{equation*}
					\vb*{p}_{s}^{n}\left(t_{n+1}\right)
					=
					f_{\mathrm{GNSS}}\left(\tilde{\vb*{s}}_{\mathrm{GNSS}}\left(t_i\right)\right)
				\end{equation*}
			}
				观测方程
			{
				\tiny	
				\begin{equation*}
					\vb*{H}^{\vb*{p}^s_{n}} 
					= 
					\begin{pmatrix}
						\left(\hat{\vb*{p}}\right)_{\times}
						&\vb*{0}_3
						&\vb*{I}_3
						&\vb*{0}_3
						&\vb*{0}_3
						&\vb*{0}_3
						&\vb*{0}_3
					\end{pmatrix}
				\end{equation*}	
		   	}
		   	\end{block}
		\end{column}   
		\begin{column}{0.5\textwidth}
		    \vspace{-1.5cm}
		   	\begin{figure}
		   	\centering
		   	    \includegraphics[width=\textwidth]{VehicleGnssPositioningAlgorithmArchitecture.pdf}
		   	\end{figure}
		\end{column}
	\end{columns}  
\end{frame}

 \begin{frame} 
 	\frametitle{数据和模型双驱动定位方法研究}
 	\framesubtitle{GSDC2023-2024 数据集训练集轨迹图}
	\begin{columns}
		\begin{column}{0.5\textwidth}
		   	\begin{figure}
			    \includegraphics[width=\textwidth]{GSDC2023DatasetSanFranciscoAreaTrackOverview.png}
			    \caption{旧金山地区}
		   	\end{figure} 
		\end{column}   
		\begin{column}{0.5\textwidth}
		   	\begin{figure}
			    \includegraphics[width=\textwidth]{GSDC2023DatasetLosAngelesAreaTrackOverview.png}
			    \caption{洛杉矶地区}
		   	\end{figure} 
		\end{column}
	\end{columns} 
\end{frame}
 
 \begin{frame} 
 	\frametitle{数据和模型双驱动定位方法研究}
 	\framesubtitle{GSDC2023-2024 数据集结果对比}
	\begin{columns}[t]
		\begin{column}{0.6\textwidth}
			\scalebox{0.9}{{   
				\tiny   
				\setlength{\tabcolsep}{2pt}     
				\begin{tabular*}{1.2\linewidth}{@{\extracolsep{\fill}}lcccl rrr rrr}
					\toprule
					\multirow{2.5}{*}{轨迹采集名称} 
					& \multirow{2.5}{*}{编号} 
					& \multirow{2.5}{*}{\makecell[c]{里程\\$\left(\unit{\kilo\meter}\right)$}} 
					& \multirow{2.5}{*}{\makecell[c]{时长\\$\left(\unit{\second}\right)$}} 
					& \multirow{2.5}{*}{\makecell[c]{手机型号}} 
					& \multicolumn{3}{c}{\makecell[c]{RTKLIB\\$\vb*{e}^{\mathrm{APE,horizontal}}$}}
					& \multicolumn{3}{c}{\makecell[c]{RTKLIB Invariant EKF\\$\vb*{e}^{\mathrm{APE,horizontal}}$}} \\
					\cmidrule{6-8} \cmidrule{9-11}
					& & & & & $\SI{50}{\percent}$  & $\SI{95}{\percent}$  & score & $\SI{50}{\percent}$  & $\SI{95}{\percent}$  & score \\
					\midrule
					\multirow{3}{*}{\makecell[c]{2023-05-19-20-10\\-us-ca-mtv-ie2}} & \multirow{3}{*}{\makecell[c]{46}} & \multirow{3}{*}{\makecell[r]{5.9}} & \multirow{3}{*}{\makecell[r]{1229}} & Pixel 5 & 1.079 & 2.572 & 1.826 & 1.080 & 2.318 & \textbf{1.699} \\
					& & & & Pixel 7 Pro & 1.238 & 2.503 & 1.871 & 1.209 & 2.485 & \textbf{1.847} \\
					& & & & Galaxy S22 Ultra & 1.202 & 2.942 & \textbf{2.072} & 1.204 & 2.989 & 2.097 \\
					\midrule
					\multirow{3}{*}{\makecell[c]{2023-05-25-20-11\\-us-ca-sjc-he2}} & \multirow{3}{*}{\makecell[c]{51}} & \multirow{3}{*}{\makecell[r]{10.7}} & \multirow{3}{*}{\makecell[r]{1263}} & Pixel 5 & 0.942 & 2.321 & 1.632 & 0.925 & 2.147 & \textbf{1.536} \\
					& & & & Pixel 7 Pro & 0.668 & 2.005 & 1.336 & 0.669 & 1.783 & \textbf{1.226} \\
					& & & & Galaxy S22 Ultra & 0.660 & 1.463 & 1.062 & 0.592 & 1.360 & \textbf{0.976} \\
					\midrule
					\multirow{2}{*}{\makecell[c]{2023-05-26-18-51\\-us-ca-sjc-ge2}} & \multirow{2}{*}{\makecell[c]{53}} & \multirow{2}{*}{\makecell[r]{7.4}} & \multirow{2}{*}{\makecell[r]{1461}} & Pixel 5 & 1.309 & 2.320 & 1.814 & 1.265 & 2.252 & \textbf{1.759} \\
					& & & & Pixel 7 Pro & 1.231 & 2.716 & 1.974 & 1.198 & 2.427 & \textbf{1.813} \\
					\bottomrule
				\end{tabular*}	
		   	}}
			\begin{itemize}
				\item 数据和模型双驱动方法提升 GNSS 观测质量变差或者信号缺失状态下的定位精度
			\end{itemize}
		\end{column} 
		\begin{column}{0.4\textwidth}
    		\vspace{-3.2cm}  
		   	\begin{figure}
    			\includegraphics[width=0.8\textwidth]{2023-05-25-20-11-us-ca-sjc-he2/sm-s908b/ComparedSubTraceT631GeographicPlot.png}
		   	\end{figure}
		   	\vspace{-0.5cm}
		   	\hspace{0.2cm}\scalebox{0.7}{{\tiny 2023-05-25-20-11-us-ca-sjc-he2/Galaxy S22 Ultra 轨迹 \SI{631}{\second}不同方法结果对比}}
		   	\vspace{-0.2cm}
		   	\begin{figure}
				\includegraphics[width=0.8\textwidth]{2023-05-26-18-51-us-ca-sjc-ge2/pixel7pro/ComparedSubTraceT289GeographicPlot.png}
		   	\end{figure}
		   	\vspace{-0.5cm}
		   	\hspace{0.2cm}\scalebox{0.7}{{\tiny 2023-05-26-18-51-us-ca-sjc-ge2/Pixel 7 Pro 轨迹 \SI{289}{\second}不同方法结果对比}}
		\end{column}    
	\end{columns}
    {\small
        \setlength{\tabcolsep}{2pt}

	}	
\end{frame}

\begin{frame}
	\frametitle{研究内容3 小结}
		\begin{block}{提出了一种数据和模型双驱动定位方法}
		{   
		    \footnotesize
		    阐述了模型驱动位置观测量和以几何观测量在不变扩展卡尔曼滤波框架下的更新融合推导。
		} 
		\end{block}		
		\begin{block}{在南京南高铁站自采数据集和 GSDC 公开数据集上进行实验验证}
		{
		    \footnotesize
		    结合具体的音频 TDoA 几何观测量和 GNSS 位置观测量实验验证了数据和模型双驱动定位方法的优越性。。
		} 
		\end{block}
\end{frame}

% !TeX encoding = UTF-8
% !TeX root = ../main.tex

%% ------------------------------------------------------------------------
%% Copyright (C) 2021-2023 SJTUG
%% 
%% SJTUBeamer Example Document by SJTUG
%% 
%% SJTUBeamer Example Document is licensed under a
%% Creative Commons Attribution-NonCommercial-ShareAlike 4.0 International License.
%% 
%% You should have received a copy of the license along with this
%% work. If not, see <http://creativecommons.org/licenses/by-nc-sa/4.0/>.
%% -----------------------------------------------------------------------

\section{总结与展望}

\begin{frame}{常见 \LaTeX{} 困惑}
  \begin{itemize}
    \item \alert{编译不通过} 缺少必要宏包,命令拼写错误,括号未配对等
    \item \alert{表格图片乱跑} 非问题,\LaTeX{} 浮动定位算法
          \link{https://liam.page/2017/04/30/floats-in-LaTeX-the-positioning-algorithm/}
    \item \alert{段落间距变大} 非问题,\LaTeX{} 排版算法
    \item \alert{参考文献} 推荐使用 \BibTeX{} 或者 Bib\LaTeX{}(视模板而定),也可以手写 \cmd{bibitem}
          \link{https://github.com/hushidong/biblatex-gb7714-2015}
  \end{itemize}
\end{frame}

% Magic Comments
% Encoding
% !TeX encoding = UTF-8
% TeX Root
% !TeX root = ../whu-defense-qianlong.tex

%% ------------------------------------------------------------------------
%% Copyright (C) 2021-2023 SJTUG
%% 
%% SJTUBeamer Example Document by SJTUG
%% 
%% SJTUBeamer Example Document is licensed under a
%% Creative Commons Attribution-NonCommercial-ShareAlike 4.0 International License.
%% 
%% You should have received a copy of the license along with this
%% work. If not, see <http://creativecommons.org/licenses/by-nc-sa/4.0/>.
%% -----------------------------------------------------------------------

\section{个人成果}

\begin{frame}
	% 8.2.6 The Frame Title
	% \frametitle<⟨overlay specification⟩>[⟨short frame title⟩]{⟨frame title text⟩}
	% 页标题
	\frametitle{攻博期间发表的论文}
	% \framesubtitle<⟨overlay specification⟩>{⟨frame subtitle text⟩}
	% 页子标题 
	% \framesubtitle{}

	\begin{enumerate}
		\item 陈锐志, \textbf{钱隆}, 牛晓光, 徐诗豪, 陈亮, 裘超. 2022. 基于数据与模型双驱动的音频/惯性传感器耦合定位方法[J]. 测绘学报, 51(7):1160-1171. (EI)
		
		\item 陈锐志, 郭光毅, 叶锋, \textbf{钱隆}, 徐诗豪, 李正. 2021. 智能手机音频信号与MEMS传感器的紧耦合室内定位方法[J]. 测绘学报, 50(2): 143-152. (EI)
		
		\item \textbf{QIAN L}, LIN X, NIU X, HUANG Q, LI L, GUO G, WANG Z, CHEN R. 2025. AVNet: learning attitude and velocity for vehicular dead reckoning using smartphone by adapting an invariant EKF[J]. Satellite Navigation, 6(1), 15. (SCI 一区)  
		
		\item LI Z, CHEN R, GUO G, YE F, \textbf{QIAN L}, XU S, HUANG L, CHEN L. 2024. Dual-step acoustic chirp signals detection using pervasive smartphones in multipath and NLOS indoor environments[J]. IEEE Internet of Things Journal, 11(4): 6494–6507. (SCI 二区) 

	\end{enumerate}
\end{frame}

\begin{frame}
	% 8.2.6 The Frame Title
	% \frametitle<⟨overlay specification⟩>[⟨short frame title⟩]{⟨frame title text⟩}
	% 页标题
	\frametitle{攻博期间发表的专利}
	% \framesubtitle<⟨overlay specification⟩>{⟨frame subtitle text⟩}
	% 页子标题 
	% \framesubtitle{}

	\begin{enumerate}
		
		\item 陈锐志, \textbf{钱隆}, 牛晓光, 徐诗豪. 基于数据与模型结合的多源融合定位方法、系统及终端:CN202210311275.1[P].2023-03-03.
		
		\item 郭光毅, 陈锐志, \textbf{钱隆}, 李正, 徐诗豪, 叶锋, 刘克强. 一种基于射频增强的广域音频室内定位方法、系统及终端:CN202211222425.8[P].2023-05-02.
  
		\item 陈锐志, 黄李雄, 刘克强, 叶锋, 郭光毅, 徐诗豪, \textbf{钱隆}, 李正, 林欣创. 一种基于音频的定位寻物方法. CN202210794893.6[P].2023-05-02.
		
		\item 郭光毅, 陈锐志, 徐诗豪, \textbf{钱隆}, 李正. 一种基于5G信号和声波信号的电子设备室内定位系统和方法:CN202210022292.3[P].2023-04-07.

	\end{enumerate}
\end{frame}

%\begin{frame}
%  \frametitle{参考文献}
%  \printbibliography[heading=none]
%\end{frame}

\makebottom

\end{document}
