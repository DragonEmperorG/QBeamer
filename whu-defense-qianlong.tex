% !TeX encoding = UTF-8

%% ------------------------------------------------------------------------
%% Copyright (C) 2021-2023 SJTUG
%% 
%% SJTUBeamer Example Document by SJTUG
%% 
%% SJTUBeamer Example Document is licensed under a
%% Creative Commons Attribution-NonCommercial-ShareAlike 4.0 International License.
%% 
%% You should have received a copy of the license along with this
%% work. If not, see <http://creativecommons.org/licenses/by-nc-sa/4.0/>.
%%
%% For a quick start, check out src/doc/sjtubeamerquickstart.tex
%% Join discussions: https://github.com/sjtug/SJTUBeamer/discussions
%% -----------------------------------------------------------------------

\documentclass[xcolor=table,dvipsnames,svgnames,aspectratio=169]{ctexbeamer}
% 可以通过 fontset=macnew / fontset=ubuntu / fontset=windows 选项切换字体集;
% 如遇无法显示的数学符号,尝试对 ctexbeamer 文档类添加 no-math 选项;
% 写纯英文幻灯片可以改用 beamer 文档类。

\usepackage{tikz}
\usetikzlibrary{arrows}

\usepackage{amsmath}
\usepackage{booktabs}
\usepackage{caption}
\usepackage{colortbl}
\usepackage{datetime2}
\usepackage{fontawesome5}
\usepackage{graphicx}
\usepackage{hologo}
\usepackage{hyperref}
\usepackage{hyperxmp}
\usepackage{listings}
\usepackage{makecell}
\usepackage{multicol}
\usepackage{multirow}
\usepackage{physics}
\usepackage{pifont}
\usepackage{shapepar}
\usepackage{siunitx}
\usepackage{tipa}
\usepackage[normalem]{ulem}

% 参考文献设置,使用 style=gb7714-2015 样式为标准顺序编码制,
% 使用 style=gb7714-2015ay 样式可以改为著者-出版年制。
\usepackage[backend=biber,style=gb7714-2015]{biblatex}
\addbibresource{ref.bib}

% 该行指定了图像的额外搜索路径
\graphicspath{{figures/}}

\hypersetup{
  pdfcopyright       = {Licensed under CC-BY-SA 4.0. Some rights reserved.},
  pdflicenseurl      = {http://creativecommons.org/licenses/by-sa/4.0/},
  unicode            = true,
  psdextra           = true,
  pdfdisplaydoctitle = true
}

\pdfstringdefDisableCommands{
  \let\\\relax
  \let\quad\relax
  \let\hspace\@gobble
}

\newcommand\link[1]{\href{#1}{\faLink}}
\newcommand\pkg[1]{\texttt{#1}}

\def\cmd#1{\texttt{\color{structure}\footnotesize $\backslash$#1}}
\def\env#1{\texttt{\color{structure}\footnotesize #1}}
\def\cmdxmp#1#2#3{\small{\texttt{\color{structure}$\backslash$#1}\{#2\}
\hspace{1em}\\ $\Rightarrow$\hspace{1em} {#3}\par\vskip1em}}

% \tikzexternalize[prefix=build/]
% 如果您需要缓存 tikz 图像,请取消注释上一行,并在编译选项中添加 -shell-escape。

\lstset{
  language=[LaTeX]TeX,           % 更改高亮语言
  texcsstyle=*\color{cprimary},  % 只在高亮 LaTeX 语言时必须
  tabsize=2,
  basicstyle=\ttfamily\small,%
  keywordstyle=\color{cprimary},%
  stringstyle=\color{csecondary},%
  commentstyle=\color{ctertiary!50!gray},%
  breaklines,%
}

% 10.1 Adding a Title Page
% \title[⟨short title⟩]{⟨title⟩}
% The ⟨short title⟩ is used in headlines and footlines. Inside the ⟨title⟩ line breaks can be inserted using
% the double-backslash command.
\title[基于智能手机的数据和模型双驱动定位方法研究]{\textbf{基于智能手机的数据和模型双驱动定位方法研究}} % 页脚显示标题 | 首页标题
% \subtitle[⟨short subtitle⟩]{⟨subtitle⟩}
% The ⟨short subtitle⟩ is not used by default, but is available via the insert \insertshortsubtitle. The
% subtitle is shown below the title in a smaller font.
\subtitle{博士学位论文预答辩}
% \institute[⟨short institute⟩]{⟨institute⟩}
% If more than one institute is given, they should be separated using the command \and and they should
% be prefixed by the command \inst with different parameters.
% \author{}
\institute[测绘遥感信息工程全国重点实验室]{武汉大学 \  测绘遥感信息工程全国重点实验室}
% \date[⟨short date⟩]{⟨date⟩}
\date{\the\year 年 \the\month 月}
% \subject{⟨text⟩}
% Enters the ⟨text⟩ as the subject text in the pdf document info. It currently has no other effect.
\subject{武汉大学博士学位论文答辩}
% \keywords{⟨text⟩}
% Enters the ⟨text⟩ as keywords in the pdf document info. It currently has no other effect.
\keywords{答辩}

% 15 Themes
% 15.1 Five Flavors of Themes
% \usetheme[⟨options⟩]{⟨name list⟩}
% Installs the presentation theme named ⟨name⟩. Currently, the effect of this command is the same as
% saying
\usetheme[max,blue]{whubeamer}
% 使用 maxplus/max/min 切换标题页样式
% 使用 red/blue 切换主色调
% 使用 light/dark 切换亮/暗色模式
% 使用外样式关键词以获得不同的边栏样式
%   miniframes infolines  sidebar
%   default    smoothbars split	 
%   shadow     tree       smoothtree
% 使用 topright/bottomright 切换徽标位置
% 使用逗号分隔列表以同时使用多种选项

\setbeamertemplate{background}{}
% 对于 max 主题,如果需要关闭正文背景图,请取消注释上一行。

\begin{document}

% 使用节目录
\AtBeginSection[]{
  \begin{frame}
    %% 使用传统节目录,也可以将 subsectionstyle=... 换成 hideallsubsections 以隐藏所有小节信息
    \tableofcontents[currentsection,subsectionstyle=show/show/hide]
    %% 或者使用节页
    % \sectionpage
  \end{frame}
}

% 使用小节目录
\AtBeginSubsection[]{		       % 在每小节开始
  \begin{frame}
    %% 使用传统小节目录
    \tableofcontents[currentsection,subsectionstyle=show/shaded/hide]
    %% 或者使用小节页
    % \subsectionpage
  \end{frame}
}

\maketitle

\begin{frame}{目录}
  % 10.5 Adding a Table of Contents
  % \tableofcontents[⟨comma-separated option list⟩]
  % Inserts a table of contents into the current frame.
  \vspace{-0.5cm}
  \tableofcontents[hideallsubsections]	% 隐藏所有小节信息
\end{frame}

% !TeX encoding = UTF-8
% !TeX root = ../whu-defense-qianlong.tex

%% ------------------------------------------------------------------------
%% Copyright (C) 2021-2023 SJTUG
%% 
%% SJTUBeamer Example Document by SJTUG
%% 
%% SJTUBeamer Example Document is licensed under a
%% Creative Commons Attribution-NonCommercial-ShareAlike 4.0 International License.
%% 
%% You should have received a copy of the license along with this
%% work. If not, see <http://creativecommons.org/licenses/by-nc-sa/4.0/>.
%% -----------------------------------------------------------------------

% 10.2 Adding Sections and Subsections
% \section<⟨mode specification⟩>[⟨short section name⟩]{⟨section name⟩}
% Starts a section. No heading is created. By default the ⟨section name⟩ is shown in the table of contents
% and in the navigation bars; if ⟨short section name⟩ is specified, it will be used in the navigation bars
% instead; if ⟨short section name⟩ is explicitly empty, it will not appear in the navigation
\section{研究背景与意义}

\begin{frame}[t]
	\frametitle{研究背景} 
	\framesubtitle{国家层面}
	中华人民共和国国民经济和社会发展第十四个五年规划和2035年远景目标纲要
	% 12.7 Splitting a Frame into Multiple Columns
	% \begin{columns}<⟨action specification⟩>[⟨options⟩]
	%   ⟨environment contents⟩
	% \end{columns}
	% • b will cause the bottom lines of the columns to be vertically aligned.
	% • c will cause the columns to be centered vertically relative to each other. Default, unless the global
	% option t is used.
	% • onlytextwidth is the same as totalwidth=\textwidth. This option can also be set for the whole
	% document with the onlytextwidth class option:
	% \documentclass[onlytextwidth]{beamer}
	% • t will cause the first lines of the columns to be aligned. Default if global option t is used.
	% • T is similar to the t option, but T aligns the tops of the first lines while t aligns the so-called
	% baselines of the first lines. If strange things seem to happen in conjunction with the t option (for
	% example if a graphic suddenly “drops down” with the t option instead of “going up”), try using
	% this option instead.
	% • totalwidth=⟨width⟩ will cause the columns to occupy not the whole page width, but only ⟨width⟩,
	% all told. Note that this means that any margins are ignored.
	% • height=⟨height⟩ will set a fixed height for the columns.
	\begin{columns}[t]
		\begin{column}{0.3\textwidth}
		   	\begin{figure}
    			\includegraphics[height=4.2cm]{中华人民共和国国民经济和社会发展第十四个五年规划和2035年远景目标纲要_封面.png}
		   	\end{figure}
		\end{column}   
		\begin{column}{0.7\textwidth}
		    \begin{block}{专栏 4 制造业核心竞争力提升}
				\begin{tabularx}{\linewidth}{@{\extracolsep{\fill}}l X}
		            05 & 北斗产业化应用 \\
		               & 突破通信导航一体化融合等技术,建设北斗应用产业创新平台,在通信、金融、能源、民航等行业开展典型示范,\CJKunderline{推动北斗在车载导航、智能手机、穿戴设备等消费领域市场化规模化应用}。	
		   		\end{tabularx}		   		 
			\end{block}
		\end{column}
	\end{columns}
\end{frame}

\begin{frame}[t]
	\frametitle{研究背景} 
	\framesubtitle{产业层面}
	2025中国卫星导航与位置服务产业发展白皮书
	\begin{columns}[t]
		\begin{column}{0.3\textwidth}
%		    \vspace{-0.4cm}
		   	\begin{figure}
    			\includegraphics[height=4.6cm]{2025中国卫星导航与位置服务产业发展白皮书_封面.png}
		   	\end{figure}
		\end{column}   
		\begin{column}{0.7\textwidth}
		    \begin{block}{大众消费领域将成北斗规模应用主战场}
		        \begin{itemize}
		          \item 已有约2.88亿部智能手机支持北斗定位功能,占比达 \SI{98}{\percent}
		          \item 大众地图软件基于北斗高精度的车道级导航功能已覆盖全国 \SI{99}{\percent} 以上的城市和乡镇道路
		          \item 11家主要电子地图服务供应商提供位置服务日均超 1 万亿次,日均提供导航服务总里程超 40 亿公里
		        \end{itemize}
			\end{block}
		\end{column}
	\end{columns}  
\end{frame}

% !TeX encoding = UTF-8
% !TeX root = ../whu-defense-qianlong.tex

%% ------------------------------------------------------------------------
%% Copyright (C) 2021-2023 SJTUG
%% 
%% SJTUBeamer Example Document by SJTUG
%% 
%% SJTUBeamer Example Document is licensed under a
%% Creative Commons Attribution-NonCommercial-ShareAlike 4.0 International License.
%% 
%% You should have received a copy of the license along with this
%% work. If not, see <http://creativecommons.org/licenses/by-nc-sa/4.0/>.
%% -----------------------------------------------------------------------

\section{国内外研究现状}

\begin{frame}
	\frametitle{行人航迹推算数据集}
	\begin{columns}[t]
		\begin{column}{0.6\textwidth}
		{
		    \tiny
		    \setlength{\tabcolsep}{2pt}
			\begin{tabular*}{1.2\textwidth}{@{\extracolsep{\fill}} cccrrrrcccccc}
				\toprule
				\multirow{3.5}{*}{数据集} & \multirow{3.5}{*}{日期} & \multirow{3.5}{*}{\makecell[c]{有\\效\\性}} & \multicolumn{6}{c}{规模与多样性} & \multicolumn{2}{c}{惯性传感器} & \multicolumn{2}{c}{参考真值}  \\ 
				\cmidrule{4-9} \cmidrule{10-11} \cmidrule{12-13}
				& & & \multirow{2}{*}{\makecell[c]{长度\\$\left(\unit{\km}\right)$}} 
				& \multirow{2}{*}{\makecell[c]{时长\\$\left(\unit{\hour}\right)$}} 
				& \multirow{2}{*}{轨迹} & \multirow{2}{*}{\makecell[c]{采集\\人数}} 
				& \multirow{2}{*}{\makecell[c]{携带\\方式}} 
				& \multirow{2}{*}{\makecell[c]{运动\\模式}}
				& \multirow{2}{*}{数量} 
				& \multirow{2}{*}{\makecell[c]{采样率\\$\left(\unit{\hertz}\right)$}} 
				& \multirow{2}{*}{设备} 
				& \multirow{2}{*}{\makecell[c]{采样率\\$\left(\unit{\hertz}\right)$}} \\
				& & & & & & & & & & & & \\
				\midrule
		        \href{https://yanhangpublic.github.io/ridi/}{RIDI}                          & 2017 & \ding{51} & 10.4 &   1.6 &   72 &         6 & 4 & 2 &   2 & 200 & Tango         & 200 \\
		   		\href{https://zenodo.org/records/1476931}{ADVIO}                            & 2018 & \ding{51} &  4.5 &   1.1 &   23 &         1 & 1 & 2 &   3 & 100 & Tango         & 100 \\
		   		\href{http://deepio.cs.ox.ac.uk/}{OxIOD}                                    & 2018 & \ding{51} & 42.5 &  14.7 &  158 &         5 & 4 & 2 &   4 & 100 & Vicon         & 250 \\
		   		\href{https://cvg.cit.tum.de/data/datasets/visual-inertial-dataset}{TUM VI} & 2018 & \ding{51} & 20   &   3.8 &   28 &         1 & 1 & 2 &   1 & 200 & OptiTrack     & 120 \\
		   		\href{https://ronin.cs.sfu.ca/}{RoNIN}                                      & 2020 & \ding{51} & 56.6 &  42.7 &  276 &       100 & 4 & 2 &   4 & 200 & Tango         & 200 \\
		   		\href{https://github.com/MAPS-Lab/smartphone-tracking-dataset}{MAPS Lab}    & 2021 & \ding{51} &  0.7 &   2.0 &    3 &         1 & 2 & 2 &   2 & 200 & Vicon         &   5 \\
		   		\href{https://github.com/LF1952987278/SIMD_Repository}{SIMD}                & 2023 & \ding{51} & 717  & 190   & 4562 & $\geq$150 & 4 & 2 & 572 &  50 & GNSS | Marker &   1 \\
		   		\href{https://github.com/BehnamZeinali/IMUNet}{IMUNet}                      & 2024 & \ding{51} & 35.9 &   9   &  126 &         4 & 4 & 2 &   5 & 200 & Tango         & 200 \\
				\bottomrule	
			\end{tabular*}        
		}
		\end{column}  
		\begin{column}{0.1\textwidth}
		\end{column} 
		\begin{column}{0.3\textwidth}
		    \begin{itemize}
				\item 数据量
				\item 多样性
			\end{itemize}
		\end{column}
	\end{columns}
   	\begin{figure}
		\includegraphics[height=2cm]{data_pipeline.png}
   	\end{figure}
\end{frame}

\begin{frame}
	\frametitle{载具航迹推算方法数据集}
	\vspace{-0.5cm}
	\begin{columns}[t]
		\begin{column}{0.7\textwidth}
		{
		    \tiny
		    \setlength{\tabcolsep}{2pt}
			\begin{tabular*}{\textwidth}{@{\extracolsep{\fill}}c c c l rrrc lc}
				\toprule
				\multirow{2}{*}{数据集} & \multirow{3}{*}{日期} & \multirow{2}{*}{\makecell[c]{有\\效\\性}} & \multirow{2}{*}{载具} &\multicolumn{4}{c}{规模} & \multicolumn{2}{c}{真值} \\
				\cmidrule{5-8} \cmidrule{9-10}
				& & & & 轨迹 & \makecell[c]{长度\\$\left(\unit{\km}\right)$} & \makecell[c]{时长\\$\left(\unit{\hour}\right)$} & \makecell[c]{采样率\\$\left(\unit{\hertz}\right)$} & \multicolumn{1}{c}{设备} & \makecell[c]{采样率\\$\left(\unit{\hertz}\right)$} \\
				\midrule
				\href{https://www.cvlibs.net/datasets/kitti/eval_odometry.php}{KITTI} 
				& 2012 & \ding{51} & LV                &  11 &   22.2 & 40.1 &    200 & GNSS/INS                       & 100       \\
				\href{https://projects.asl.ethz.ch/datasets/doku.php?id=kmavvisualinertialdatasets}{EuRoC MAV} 
				& 2016 & \ding{51} & MR                &  11 &    0.9 &  0.4 &    200 & Leica MS50 \& Vicon            & 20 \& 100 \\
				\href{https://sites.google.com/view/complex-urban-dataset/home}{Complex Urban Dataset} 
				& 2018 & \ding{51} & LV                &  41 &  356.1 &  $-$ &    100 & SLAM                           & 100       \\
				\href{https://github.com/onyekpeu/IO-VNBD}{IO-VNBD} 
				& 2021 & \ding{51} & LV                &  43 & 4400   & 58   &    100 & GNSS                           &  10       \\
				\href{https://www.kaggle.com/competitions/smartphone-decimeter-2022/data}{GSDC} 
				& 2022 & \ding{51} & LV                &  62 & 2123.6 & 30.3 & 50-200 & GNSS/INS                       &   1       \\
				\href{https://www.kaggle.com/competitions/smartphone-decimeter-2023}{GSDC} 
				& 2023 & \ding{51} & LV                &  65 & 1685.3 & 29.4 & 50-200 & GNSS/INS                       &   1       \\
				\href{https://figshare.com/articles/dataset/Multiple_and_Gyro-Free_Inertial_Datasets/26927089/1?file=48979765}{MAGF-ID} 
				& 2024 & \ding{51} & LV \& MR          & 115 &    $-$ &  5.6 &    200 & $\mathrm{GNSS}^{\mathrm{RTK}}$ & 200        \\
				\bottomrule	
			\end{tabular*}   
		}
		\end{column}   
		\begin{column}{0.3\textwidth}
		    \begin{itemize}
				\item 以智能手机为中心的车载数据采集
			\end{itemize}
		\end{column}
	\end{columns}
	\begin{columns}[t]
		\begin{column}{0.3\textwidth}
		   	\begin{figure}
				\includegraphics[height=2.5cm]{KITTIRecordingPlatform.jpg}
		   	\end{figure}
		\end{column}
		\begin{column}{0.3\textwidth}
		    \vspace{-0.5cm}
		   	\begin{figure}
				\includegraphics[height=2.5cm]{ComplexUrbanDatasetRecordingPlatform.png}
		   	\end{figure}
		\end{column}
		\begin{column}{0.3\textwidth}
		    \vspace{-0.5cm}
		   	\begin{figure}
				\includegraphics[height=2.5cm]{GSDC2023RecordingPlatform.png}
		   	\end{figure}
		\end{column}
	\end{columns}
\end{frame}

\begin{frame}
	\frametitle{国内外研究现状}
	\framesubtitle{数据驱动行人航迹推算方法}
	\vspace{-0.5cm}
	\begin{columns}[t]
		\begin{column}{0.5\textwidth}
		{
		    \tiny
		    \setlength{\tabcolsep}{2pt}
			\begin{tabular*}{\textwidth}{@{\extracolsep{\fill}}lclc}
				\toprule
				\multicolumn{1}{c}{模型} & 日期 & 期刊/会议 & 数据集 \\
				\midrule
				\multirow{2}{*}{IONet} & 2018 & AAAI                            & \multirow{2}{*}{OxIOD} \\
				                       & 2019 & IEEE. Trans. Mob. Comput.       &                        \\
				               L-IONet & 2020 & IEEE Internet Things J.         & OxIOD                  \\
				                 RoNIN & 2020 & ICRA                            & RoNIN                  \\
				        Extended IONet & 2021 & Expert Syst. Appl.              & OxIOD                  \\ 
				                   DIO & 2021 & IEEE Trans. Instrum. Meas.      & RoNIN \& Private       \\
				                  IDOL & 2021 & AAAI                            & IDOL                   \\
				                   RIO & 2022 & CVPR                            & Public \& Private      \\
				                 HNNTA & 2022 & IEEE Trans. Instrum. Meas.      & RoNIN \& Private       \\
				                  CTIN & 2022 & AAAI                            & Public \& Private      \\
				                  RIOT & 2023 & Sensors                         & OxIOD                  \\
				              DO IONet & 2023 & IEEE Access                     & OxIOD                  \\
				                 SSHNN & 2023 & IEEE Sens. J.                   & Private                \\
				                  SIMD & 2023 & IEEE Trans. Instrum. Meas.      & SIMD                   \\
				                IMUNet & 2024 & IEEE Trans. Instrum. Meas.      & Public                 \\
				              ResMixer & 2024 & IEEE Sens. J.                   & RoNIN \& Private       \\
				\bottomrule
			\end{tabular*}     
		}
		\end{column}
		\begin{column}{0.5\textwidth}
		    \vspace{-1.0cm}
			\begin{block}{数据驱动模型估计精度}
			    {
			        \footnotesize
					\begin{itemize}
						\item 应用先进深度神经网络模型
						\item 设计合理损失函数
						\item 设计合理损失函数
					\end{itemize}			    
			    }
			\end{block}
		    \begin{block}{数据驱动模型输出维度}
  			    {
  			        \footnotesize
					\begin{itemize}
						\item 三维位姿表达
						\item 位置和姿态多任务学习
					\end{itemize}
				}
			\end{block}
			\begin{block}{数据驱动模型计算量}
			    {
			        \footnotesize
					\begin{itemize}
						\item 应用轻量级网络减小计算量
					\end{itemize}
				}
			\end{block}
		\end{column}
	\end{columns}	
\end{frame}

\begin{frame}
	\frametitle{数据和模型双驱动行人航迹推算方法}
	\begin{columns}[t]
		\begin{column}{0.5\textwidth}
		{
		    \tiny
		    \setlength{\tabcolsep}{2pt}
			\begin{tabular*}{\linewidth}{@{\extracolsep{\fill}}cccc}
				\toprule
				\multicolumn{1}{c}{模型} & 日期 & 期刊/会议 & 数据集 \\
				\midrule
				    \rowcolor{gray!50} RIDI & 2018 & ECCV                            & RIDI                     \\
				                    EKF+CNN & 2018 & MLSP                            & Public                    \\
				  \multirow{2}{*}{EKF+LSTM} & 2018 & IPIN                            & \multirow{2}{*}{Private} \\
				                            & 2019 & IEEE Sens. J.                   &                          \\
				                      AZUPT & 2019 & GLOBECOM                        & Private                  \\	
				    \rowcolor{gray!50} TLIO & 2019 & IEEE Robot. Autom. Lett.        & Private                  \\
				\rowcolor{gray!50} IEKF+CNN & 2020 & IROS                            & RIDI \& Private          \\
				                   EKF+LSTM & 2020 & IROS                            & Private                  \\
				                     DeepIT & 2021 & IMWUT                           & Private                  \\
				                       MINN & 2022 & IEEE Sens. J.                   & SLE \& WDE               \\
				                   TinyOdom & 2022 & IMWUT                           & Public \& Private        \\
				                 MSCKF+LLIO & 2022 & IEEE Trans. Instrum. Meas.      & Private                  \\
				                 SCEKF+LLIO & 2022 & IEEE Internet Things J.         & Private                  \\
				                 EKF+ResNet & 2023 & ICARM                           & RoNIN                    \\
				\bottomrule
			\end{tabular*}          
		}
		\end{column}   
		\begin{column}{0.5\textwidth}
			\begin{block}{数据驱动模型估计精度}
			   {
			       \footnotesize
					\begin{itemize}
						\item 三维位姿表达
					\end{itemize}
				}
			\end{block}
		    \begin{block}{数据驱动模型输出观测量}数据驱动模型估计精度
  			    {
  			        \footnotesize
					\begin{itemize}
						\item 三维位姿表达
						\item 位置和姿态多任务学习
					\end{itemize}
				}
			\end{block}
			\begin{block}{数据驱动模型计算量}
			    {
			        \footnotesize
					\begin{itemize}
						\item 应用轻量级网络减小计算量
					\end{itemize}
				}
			\end{block}
		\end{column}
	\end{columns}	
\end{frame}

\begin{frame}
	\frametitle{数据驱动载具航迹推算方法}
	\begin{columns}[t]
		\begin{column}{0.5\textwidth}
		{
		    \tiny
		    \setlength{\tabcolsep}{2pt}
			\begin{tabular*}{\linewidth}{@{\extracolsep{\fill}}cccc}
				\toprule
				\multicolumn{1}{c}{模型} & 日期 & 期刊/会议 & 数据集 \\
				\midrule
				OriNet  & 2020 & IEEE Robot. Autom. Lett.  & EuRoC MAV \\
				DeepVIP & 2022 & IEEE Trans. Veh. Technol. & Private   \\ % IEEE Transactions on Vehicular Technology ( Volume: 71, Issue: 12, December 2022) | 17 August 2022
				LSTM    & 2023 & Machines                  & JKK       \\
				DI-EME  & 2024 & IEEE Sens. J.             & Private   \\
				\bottomrule
			\end{tabular*}         
		}
		\end{column}   
		\begin{column}{0.5\textwidth}
		    \begin{block}{研究目标}
				\begin{itemize}
					\item 提升模型估计精度
					\item 降低模型计算开销
				\end{itemize}
			\end{block}
			\begin{block}{提升模型估计精度}
				\begin{itemize}
					\item maxplus
				\end{itemize}
			\end{block}
		\end{column}
	\end{columns}	
\end{frame}

\begin{frame}
    \frametitle{数据和模型双驱动载具航迹推算方法}
	\begin{columns}[t]
		\begin{column}{0.5\textwidth}
		{
		    \tiny
		    \setlength{\tabcolsep}{2pt}
	   		\begin{tabular*}{\linewidth}{@{\extracolsep{\fill}}cccc}
				\toprule
				\multicolumn{1}{c}{模型} & 日期 & 期刊/会议 & 数据集 \\
				\midrule
				AbolDeepIO & 2019 & IEEE Trans. Intell. Transp. Syst. & EuRoC          \\ % IEEE Transactions on Intelligent Transportation Systems
				RINS-W     & 2019 & IROS                              & KAIST          \\ % 2019 IEEE/RSJ International Conference on Intelligent Robots and Systems (IROS)
				AI-IMU     & 2020 & IEEE T. Intell. Veh.              & KITTI Odometry \\ % IEEE Transactions on Intelligent Vehicles
				RAN        & 2021 & IEEE Trans. Veh. Technol.         & Private         \\ % IEEE Transactions on Vehicular Technology
				RNN        & 2021 & ICRA                              & EuRoC \& KAIST    \\ 
				TinyOdom   & 2022 & IMWUT                             & Public \& Private \\
				RNN-IEKF   & 2022 & IROS                              & KITTI Odometry    \\
				OdoNet     & 2022 & IEEE Sens. J.                     & Private         \\ % IEEE Sensors Journal
				SdoNet     & 2023 & IEEE Internet Things J.           & KITTI Odometry \\ % IEEE Internet of Things Journal
				DeepOdo    & 2023 & IEEE Trans. Instrum. Meas.        & Private         \\ % IEEE Transactions on Instrumentation and Measurement
				IEKF+CNN   & 2023 & IEEE Trans. Ind. Electron.        & KITTI \& Private         \\ % IEEE Transactions on Industrial Electronics | 15 August 2023
				KF+VHRNet  & 2025 & IEEE Sens. J.                     & Private         \\ % IEEE Sensors Journal
				\bottomrule
	   		\end{tabular*}
		}
		\end{column}   
		\begin{column}{0.5\textwidth}
		    \begin{block}{研究目标}
				\begin{itemize}
					\item 提升模型估计精度
					\item 降低模型计算开销
				\end{itemize}
			\end{block}
			\begin{block}{提升模型估计精度}
				\begin{itemize}
					\item maxplus
				\end{itemize}
			\end{block}
		\end{column}
	\end{columns}
\end{frame}

% !TeX encoding = UTF-8
% !TeX root = ../main.tex

%% ------------------------------------------------------------------------
%% Copyright (C) 2021-2023 SJTUG
%% 
%% SJTUBeamer Example Document by SJTUG
%% 
%% SJTUBeamer Example Document is licensed under a
%% Creative Commons Attribution-NonCommercial-ShareAlike 4.0 International License.
%% 
%% You should have received a copy of the license along with this
%% work. If not, see <http://creativecommons.org/licenses/by-nc-sa/4.0/>.
%% -----------------------------------------------------------------------

\section{研究内容}

\begin{frame}{常见 \LaTeX{} 困惑}
  \begin{itemize}
    \item \alert{编译不通过} 缺少必要宏包,命令拼写错误,括号未配对等
    \item \alert{表格图片乱跑} 非问题,\LaTeX{} 浮动定位算法
          \link{https://liam.page/2017/04/30/floats-in-LaTeX-the-positioning-algorithm/}
    \item \alert{段落间距变大} 非问题,\LaTeX{} 排版算法
    \item \alert{参考文献} 推荐使用 \BibTeX{} 或者 Bib\LaTeX{}(视模板而定),也可以手写 \cmd{bibitem}
          \link{https://github.com/hushidong/biblatex-gb7714-2015}
  \end{itemize}
\end{frame}

% !TeX encoding = UTF-8
% !TeX root = ../main.tex

%% ------------------------------------------------------------------------
%% Copyright (C) 2021-2023 SJTUG
%% 
%% SJTUBeamer Example Document by SJTUG
%% 
%% SJTUBeamer Example Document is licensed under a
%% Creative Commons Attribution-NonCommercial-ShareAlike 4.0 International License.
%% 
%% You should have received a copy of the license along with this
%% work. If not, see <http://creativecommons.org/licenses/by-nc-sa/4.0/>.
%% -----------------------------------------------------------------------

\section{总结与展望}

\begin{frame}{常见 \LaTeX{} 困惑}
  \begin{itemize}
    \item \alert{编译不通过} 缺少必要宏包,命令拼写错误,括号未配对等
    \item \alert{表格图片乱跑} 非问题,\LaTeX{} 浮动定位算法
          \link{https://liam.page/2017/04/30/floats-in-LaTeX-the-positioning-algorithm/}
    \item \alert{段落间距变大} 非问题,\LaTeX{} 排版算法
    \item \alert{参考文献} 推荐使用 \BibTeX{} 或者 Bib\LaTeX{}(视模板而定),也可以手写 \cmd{bibitem}
          \link{https://github.com/hushidong/biblatex-gb7714-2015}
  \end{itemize}
\end{frame}

% Magic Comments
% Encoding
% !TeX encoding = UTF-8
% TeX Root
% !TeX root = ../whu-defense-qianlong.tex

%% ------------------------------------------------------------------------
%% Copyright (C) 2021-2023 SJTUG
%% 
%% SJTUBeamer Example Document by SJTUG
%% 
%% SJTUBeamer Example Document is licensed under a
%% Creative Commons Attribution-NonCommercial-ShareAlike 4.0 International License.
%% 
%% You should have received a copy of the license along with this
%% work. If not, see <http://creativecommons.org/licenses/by-nc-sa/4.0/>.
%% -----------------------------------------------------------------------

\section{个人成果}

\begin{frame}
	% 8.2.6 The Frame Title
	% \frametitle<⟨overlay specification⟩>[⟨short frame title⟩]{⟨frame title text⟩}
	% 页标题
	\frametitle{攻博期间发表的论文}
	% \framesubtitle<⟨overlay specification⟩>{⟨frame subtitle text⟩}
	% 页子标题 
	% \framesubtitle{}

	\begin{enumerate}
		\item 陈锐志, \textbf{钱隆}, 牛晓光, 徐诗豪, 陈亮, 裘超. 2022. 基于数据与模型双驱动的音频/惯性传感器耦合定位方法[J]. 测绘学报, 51(7):1160-1171. (EI)
		
		\item 陈锐志, 郭光毅, 叶锋, \textbf{钱隆}, 徐诗豪, 李正. 2021. 智能手机音频信号与MEMS传感器的紧耦合室内定位方法[J]. 测绘学报, 50(2): 143-152. (EI)
		
		\item \textbf{QIAN L}, LIN X, NIU X, HUANG Q, LI L, GUO G, WANG Z, CHEN R. 2025. AVNet: learning attitude and velocity for vehicular dead reckoning using smartphone by adapting an invariant EKF[J]. Satellite Navigation, 6(1), 15. (SCI 一区)  
		
		\item LI Z, CHEN R, GUO G, YE F, \textbf{QIAN L}, XU S, HUANG L, CHEN L. 2024. Dual-step acoustic chirp signals detection using pervasive smartphones in multipath and NLOS indoor environments[J]. IEEE Internet of Things Journal, 11(4): 6494–6507. (SCI 二区) 

	\end{enumerate}
\end{frame}

\begin{frame}
	% 8.2.6 The Frame Title
	% \frametitle<⟨overlay specification⟩>[⟨short frame title⟩]{⟨frame title text⟩}
	% 页标题
	\frametitle{攻博期间发表的专利}
	% \framesubtitle<⟨overlay specification⟩>{⟨frame subtitle text⟩}
	% 页子标题 
	% \framesubtitle{}

	\begin{enumerate}
		
		\item 陈锐志, \textbf{钱隆}, 牛晓光, 徐诗豪. 基于数据与模型结合的多源融合定位方法、系统及终端:CN202210311275.1[P].2023-03-03.
		
		\item 郭光毅, 陈锐志, \textbf{钱隆}, 李正, 徐诗豪, 叶锋, 刘克强. 一种基于射频增强的广域音频室内定位方法、系统及终端:CN202211222425.8[P].2023-05-02.
  
		\item 陈锐志, 黄李雄, 刘克强, 叶锋, 郭光毅, 徐诗豪, \textbf{钱隆}, 李正, 林欣创. 一种基于音频的定位寻物方法. CN202210794893.6[P].2023-05-02.
		
		\item 郭光毅, 陈锐志, 徐诗豪, \textbf{钱隆}, 李正. 一种基于5G信号和声波信号的电子设备室内定位系统和方法:CN202210022292.3[P].2023-04-07.

	\end{enumerate}
\end{frame}

%\begin{frame}
%  \frametitle{参考文献}
%  \printbibliography[heading=none]
%\end{frame}

\makebottom

\end{document}
