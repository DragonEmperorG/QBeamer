% !TeX encoding = UTF-8
% !TeX root = ../whu-defense-qianlong.tex

%% ------------------------------------------------------------------------
%% Copyright (C) 2021-2023 SJTUG
%% 
%% SJTUBeamer Example Document by SJTUG
%% 
%% SJTUBeamer Example Document is licensed under a
%% Creative Commons Attribution-NonCommercial-ShareAlike 4.0 International License.
%% 
%% You should have received a copy of the license along with this
%% work. If not, see <http://creativecommons.org/licenses/by-nc-sa/4.0/>.
%% -----------------------------------------------------------------------

% 10.2 Adding Sections and Subsections
% \section<⟨mode specification⟩>[⟨short section name⟩]{⟨section name⟩}
% Starts a section. No heading is created. By default the ⟨section name⟩ is shown in the table of contents
% and in the navigation bars; if ⟨short section name⟩ is specified, it will be used in the navigation bars
% instead; if ⟨short section name⟩ is explicitly empty, it will not appear in the navigation
\section{研究背景与意义}

\begin{frame}[t]
	\frametitle{研究背景} 
	\framesubtitle{国家层面}
	中华人民共和国国民经济和社会发展第十四个五年规划和2035年远景目标纲要
	\begin{columns}[t]
		\begin{column}{0.3\textwidth}
		   	\begin{figure}
    			\includegraphics[height=4.2cm]{中华人民共和国国民经济和社会发展第十四个五年规划和2035年远景目标纲要_封面.png}
		   	\end{figure}
		\end{column}   
		\begin{column}{0.7\textwidth}
		    \begin{block}{专栏 4 制造业核心竞争力提升}
				\begin{tabularx}{\linewidth}{@{\extracolsep{\fill}}l X}
		            05 & 北斗产业化应用 \\
		               & 突破通信导航一体化融合等技术,建设北斗应用产业创新平台,在通信、金融、能源、民航等行业开展典型示范,\CJKunderline{推动北斗在车载导航、智能手机、穿戴设备等消费领域市场化规模化应用}。	
		   		\end{tabularx}		   		 
			\end{block}
		\end{column}
	\end{columns}
\end{frame}

\begin{frame}[t]
	\frametitle{研究背景} 
	\framesubtitle{产业层面}
	2025中国卫星导航与位置服务产业发展白皮书
	\begin{columns}[t]
		\begin{column}{0.3\textwidth}
%		    \vspace{-0.4cm}
		   	\begin{figure}
    			\includegraphics[height=4.6cm]{2025中国卫星导航与位置服务产业发展白皮书_封面.png}
		   	\end{figure}
		\end{column}   
		\begin{column}{0.7\textwidth}
		    \begin{block}{大众消费领域将成北斗规模应用主战场}
		        \begin{itemize}
		          \item 已有约2.88亿部智能手机支持北斗定位功能,占比达 \SI{98}{\percent}
		          \item 大众地图软件基于北斗高精度的车道级导航功能已覆盖全国 \SI{99}{\percent} 以上的城市和乡镇道路
		          \item 11家主要电子地图服务供应商提供位置服务日均超 1 万亿次,日均提供导航服务总里程超 40 亿公里
		        \end{itemize}
			\end{block}
		\end{column}
	\end{columns}  
\end{frame}
