% !TeX encoding = UTF-8
% !TeX root = ../whu-defense-qianlong.tex

%% ------------------------------------------------------------------------
%% Copyright (C) 2021-2023 SJTUG
%% 
%% SJTUBeamer Example Document by SJTUG
%% 
%% SJTUBeamer Example Document is licensed under a
%% Creative Commons Attribution-NonCommercial-ShareAlike 4.0 International License.
%% 
%% You should have received a copy of the license along with this
%% work. If not, see <http://creativecommons.org/licenses/by-nc-sa/4.0/>.
%% -----------------------------------------------------------------------

\section{国内外研究现状}

\begin{frame}
	\frametitle{行人航迹推算数据集}
	\begin{columns}[t]
		\begin{column}{0.6\textwidth}
		{
		    \tiny
		    \setlength{\tabcolsep}{2pt}
			\begin{tabular*}{1.2\textwidth}{@{\extracolsep{\fill}} cccrrrrcccccc}
				\toprule
				\multirow{3.5}{*}{数据集} & \multirow{3.5}{*}{日期} & \multirow{3.5}{*}{\makecell[c]{有\\效\\性}} & \multicolumn{6}{c}{规模与多样性} & \multicolumn{2}{c}{惯性传感器} & \multicolumn{2}{c}{参考真值}  \\ 
				\cmidrule{4-9} \cmidrule{10-11} \cmidrule{12-13}
				& & & \multirow{2}{*}{\makecell[c]{长度\\$\left(\unit{\km}\right)$}} 
				& \multirow{2}{*}{\makecell[c]{时长\\$\left(\unit{\hour}\right)$}} 
				& \multirow{2}{*}{轨迹} & \multirow{2}{*}{\makecell[c]{采集\\人数}} 
				& \multirow{2}{*}{\makecell[c]{携带\\方式}} 
				& \multirow{2}{*}{\makecell[c]{运动\\模式}}
				& \multirow{2}{*}{数量} 
				& \multirow{2}{*}{\makecell[c]{采样率\\$\left(\unit{\hertz}\right)$}} 
				& \multirow{2}{*}{设备} 
				& \multirow{2}{*}{\makecell[c]{采样率\\$\left(\unit{\hertz}\right)$}} \\
				& & & & & & & & & & & & \\
				\midrule
		        \href{https://yanhangpublic.github.io/ridi/}{RIDI}                          & 2017 & \ding{51} & 10.4 &   1.6 &   72 &         6 & 4 & 2 &   2 & 200 & Tango         & 200 \\
		   		\href{https://zenodo.org/records/1476931}{ADVIO}                            & 2018 & \ding{51} &  4.5 &   1.1 &   23 &         1 & 1 & 2 &   3 & 100 & Tango         & 100 \\
		   		\href{http://deepio.cs.ox.ac.uk/}{OxIOD}                                    & 2018 & \ding{51} & 42.5 &  14.7 &  158 &         5 & 4 & 2 &   4 & 100 & Vicon         & 250 \\
		   		\href{https://cvg.cit.tum.de/data/datasets/visual-inertial-dataset}{TUM VI} & 2018 & \ding{51} & 20   &   3.8 &   28 &         1 & 1 & 2 &   1 & 200 & OptiTrack     & 120 \\
		   		\href{https://ronin.cs.sfu.ca/}{RoNIN}                                      & 2020 & \ding{51} & 56.6 &  42.7 &  276 &       100 & 4 & 2 &   4 & 200 & Tango         & 200 \\
		   		\href{https://github.com/MAPS-Lab/smartphone-tracking-dataset}{MAPS Lab}    & 2021 & \ding{51} &  0.7 &   2.0 &    3 &         1 & 2 & 2 &   2 & 200 & Vicon         &   5 \\
		   		\href{https://github.com/LF1952987278/SIMD_Repository}{SIMD}                & 2023 & \ding{51} & 717  & 190   & 4562 & $\geq$150 & 4 & 2 & 572 &  50 & GNSS | Marker &   1 \\
		   		\href{https://github.com/BehnamZeinali/IMUNet}{IMUNet}                      & 2024 & \ding{51} & 35.9 &   9   &  126 &         4 & 4 & 2 &   5 & 200 & Tango         & 200 \\
				\bottomrule	
			\end{tabular*}        
		}
		\end{column}  
		\begin{column}{0.1\textwidth}
		\end{column} 
		\begin{column}{0.3\textwidth}
		    \begin{itemize}
				\item 数据量
				\item 多样性
			\end{itemize}
		\end{column}
	\end{columns}
   	\begin{figure}
		\includegraphics[height=2cm]{data_pipeline.png}
   	\end{figure}
\end{frame}

\begin{frame}
	\frametitle{载具航迹推算方法数据集}
	\vspace{-0.5cm}
	\begin{columns}[t]
		\begin{column}{0.7\textwidth}
		{
		    \tiny
		    \setlength{\tabcolsep}{2pt}
			\begin{tabular*}{\textwidth}{@{\extracolsep{\fill}}c c c l rrrc lc}
				\toprule
				\multirow{2}{*}{数据集} & \multirow{3}{*}{日期} & \multirow{2}{*}{\makecell[c]{有\\效\\性}} & \multirow{2}{*}{载具} &\multicolumn{4}{c}{规模} & \multicolumn{2}{c}{真值} \\
				\cmidrule{5-8} \cmidrule{9-10}
				& & & & 轨迹 & \makecell[c]{长度\\$\left(\unit{\km}\right)$} & \makecell[c]{时长\\$\left(\unit{\hour}\right)$} & \makecell[c]{采样率\\$\left(\unit{\hertz}\right)$} & \multicolumn{1}{c}{设备} & \makecell[c]{采样率\\$\left(\unit{\hertz}\right)$} \\
				\midrule
				\rowcolor{gray!50} \href{https://www.cvlibs.net/datasets/kitti/eval_odometry.php}{KITTI} 
				& 2012 & \ding{51} & LV                &  11 &   22.2 & 40.1 &    200 & GNSS/INS                       & 100       \\
				\href{https://projects.asl.ethz.ch/datasets/doku.php?id=kmavvisualinertialdatasets}{EuRoC MAV} 
				& 2016 & \ding{51} & MR                &  11 &    0.9 &  0.4 &    200 & Leica MS50 \& Vicon            & 20 \& 100 \\
				\href{https://sites.google.com/view/complex-urban-dataset/home}{Complex Urban Dataset} 
				& 2018 & \ding{51} & LV                &  41 &  356.1 &  $-$ &    100 & SLAM                           & 100       \\
				\href{https://github.com/onyekpeu/IO-VNBD}{IO-VNBD} 
				& 2021 & \ding{51} & LV                &  43 & 4400   & 58   &    100 & GNSS                           &  10       \\
				\rowcolor{gray!50} \href{https://www.kaggle.com/competitions/smartphone-decimeter-2022/data}{GSDC} 
				& 2022 & \ding{51} & LV                &  62 & 2123.6 & 30.3 & 50-200 & GNSS/INS                       &   1       \\
				\rowcolor{gray!50} \href{https://www.kaggle.com/competitions/smartphone-decimeter-2023}{GSDC} 
				& 2023 & \ding{51} & LV                &  65 & 1685.3 & 29.4 & 50-200 & GNSS/INS                       &   1       \\
				\href{https://figshare.com/articles/dataset/Multiple_and_Gyro-Free_Inertial_Datasets/26927089/1?file=48979765}{MAGF-ID} 
				& 2024 & \ding{51} & LV \& MR          & 115 &    $-$ &  5.6 &    200 & $\mathrm{GNSS}^{\mathrm{RTK}}$ & 200        \\
				\bottomrule	
			\end{tabular*}   
		}
		\end{column}   
		\begin{column}{0.3\textwidth}
		    \begin{itemize}
				\item 以智能手机为中心的车载数据采集
			\end{itemize}
		\end{column}
	\end{columns}
	\begin{columns}[t]
		\begin{column}{0.3\textwidth}
		   	\begin{figure}
				\includegraphics[height=2.5cm]{KITTIRecordingPlatform.jpg}
		   	\end{figure}
		\end{column}
		\begin{column}{0.3\textwidth}
		    \vspace{-0.5cm}
		   	\begin{figure}
				\includegraphics[height=2.5cm]{ComplexUrbanDatasetRecordingPlatform.png}
		   	\end{figure}
		\end{column}
		\begin{column}{0.3\textwidth}
		    \vspace{-0.5cm}
		   	\begin{figure}
				\includegraphics[height=2.5cm]{GSDC2023RecordingPlatform.png}
		   	\end{figure}
		\end{column}
	\end{columns}
\end{frame}

\begin{frame}
	\frametitle{国内外研究现状}
	\framesubtitle{数据驱动行人航迹推算方法}
	\begin{columns}[t]
		\begin{column}[t]{0.5\textwidth}
		{
		    \tiny
		    \SetTblrInner{rowsep=1pt,colsep=2pt}
			\begin{tblr}{
			  colspec = {l c l c},
			  row{2,3,4,5,7,14} = {bg=gray!50},
			}
				\toprule
				模型 & 日期 & 期刊/会议 & 数据集 \\
				\midrule
				\SetCell[r=2]{} IONet & 2018 & AAAI                            & \SetCell[r=2]{} OxIOD \\
				                      & 2019 & IEEE. Trans. Mob. Comput.       &                       \\
				              L-IONet & 2020 & IEEE Internet Things J.         & OxIOD                 \\
				                RoNIN & 2020 & ICRA                            & RoNIN                 \\
				       Extended IONet & 2021 & Expert Syst. Appl.              & OxIOD                 \\ 
				                  DIO & 2021 & IEEE Trans. Instrum. Meas.      & RoNIN \& Private      \\
				                 IDOL & 2021 & AAAI                            & IDOL                  \\
				                  RIO & 2022 & CVPR                            & Public {\&} Private   \\
				                HNNTA & 2022 & IEEE Trans. Instrum. Meas.      & RoNIN {\&} Private    \\
				                 CTIN & 2022 & AAAI                            & Public {\&} Private   \\
				                 RIOT & 2023 & Sensors                         & OxIOD                 \\
				             DO IONet & 2023 & IEEE Access                     & OxIOD                 \\
				                SSHNN & 2023 & IEEE Sens. J.                   & Private               \\
				                 SIMD & 2023 & IEEE Trans. Instrum. Meas.      & SIMD                  \\
				               IMUNet & 2024 & IEEE Trans. Instrum. Meas.      & Public                \\
				             ResMixer & 2024 & IEEE Sens. J.                   & RoNIN {\&} Private    \\
				\bottomrule
			\end{tblr}     
		}
		\end{column}
		\begin{column}[t]{0.5\textwidth}
		    \vspace{-3.0cm}
			\begin{block}{数据驱动模型估计精度}
			    {
			        \footnotesize
					\begin{itemize}
						\item 应用先进深度神经网络模型
						\item 设计合理损失函数
						\item 设计合理损失函数
					\end{itemize}			    
			    }
			\end{block}
		    \begin{block}{数据驱动模型输出维度}
  			    {
  			        \footnotesize
					\begin{itemize}
						\item 三维位姿表达
						\item 位置和姿态多任务学习
					\end{itemize}
				}
			\end{block}
			\begin{block}{数据驱动模型计算量}
			    {
			        \footnotesize
					\begin{itemize}
						\item 应用轻量级网络减小计算量
					\end{itemize}
				}
			\end{block}
		\end{column}
	\end{columns}	
\end{frame}

\begin{frame}
	\frametitle{数据和模型双驱动行人航迹推算方法}
	\begin{columns}[t]
		\begin{column}{0.5\textwidth}
		{
		    \tiny
		    \setlength{\tabcolsep}{2pt}
			\begin{tabular*}{\linewidth}{@{\extracolsep{\fill}}cccc}
				\toprule
				\multicolumn{1}{c}{模型} & 日期 & 期刊/会议 & 数据集 \\
				\midrule
				    \rowcolor{gray!50} RIDI & 2018 & ECCV                            & RIDI                     \\
				                    EKF+CNN & 2018 & MLSP                            & Public                   \\
				  \multirow{2}{*}{EKF+LSTM} & 2018 & IPIN                            & \multirow{2}{*}{Private} \\
				                            & 2019 & IEEE Sens. J.                   &                          \\
				                      AZUPT & 2019 & GLOBECOM                        & Private                  \\	
				    \rowcolor{gray!50} TLIO & 2019 & IEEE Robot. Autom. Lett.        & Private                  \\
				\rowcolor{gray!50} IEKF+CNN & 2020 & IROS                            & RIDI \& Private          \\
				                   EKF+LSTM & 2020 & IROS                            & Private                  \\
				                     DeepIT & 2021 & IMWUT                           & Private                  \\
				                       MINN & 2022 & IEEE Sens. J.                   & SLE \& WDE               \\
				                   TinyOdom & 2022 & IMWUT                           & Public \& Private        \\
				                 MSCKF+LLIO & 2022 & IEEE Trans. Instrum. Meas.      & Private                  \\
				                 SCEKF+LLIO & 2022 & IEEE Internet Things J.         & Private                  \\
				                 EKF+ResNet & 2023 & ICARM                           & RoNIN                    \\
				    \rowcolor{gray!50} 3DIO & 2024 & IEEE Internet Things J.         & Private                  \\
				\bottomrule
			\end{tabular*}          
		}
		\end{column}   
		\begin{column}{0.5\textwidth}
			\begin{block}{数据驱动模型估计精度}
			   {
			       \footnotesize
					\begin{itemize}
					    \item 优化滤波框架
						\item 优化数据驱动观测量精度
						\item 增加新的数据驱动观测量
					\end{itemize}
				}
			\end{block}
			\begin{block}{数据驱动模型计算量}
			    {
			        \footnotesize
					\begin{itemize}
						\item 应用轻量级网络减小计算量
					\end{itemize}
				}
			\end{block}
		\end{column}
	\end{columns}	
\end{frame}

\begin{frame}
	\frametitle{数据驱动载具航迹推算方法}
	\begin{columns}[t]
		\begin{column}{0.5\textwidth}
		{
		    \tiny
		    \setlength{\tabcolsep}{2pt}
			\begin{tabular*}{\linewidth}{@{\extracolsep{\fill}}cccc}
				\toprule
				\multicolumn{1}{c}{模型} & 日期 & 期刊/会议 & 数据集 \\
				\midrule
				OriNet  & 2020 & IEEE Robot. Autom. Lett.  & EuRoC MAV \\
				DeepVIP & 2022 & IEEE Trans. Veh. Technol. & Private   \\ % IEEE Transactions on Vehicular Technology ( Volume: 71, Issue: 12, December 2022) | 17 August 2022
				LSTM    & 2023 & Machines                  & JKK       \\
				DI-EME  & 2024 & IEEE Sens. J.             & Private   \\
				\bottomrule
			\end{tabular*}         
		}
		\end{column}   
		\begin{column}{0.5\textwidth}
			\begin{block}{数据驱动载具模型}
			    {
			        \footnotesize
					\begin{itemize}
						\item 研究端到端方法在车载定位中的应用
						\item 研究模型的性能
					\end{itemize}			    
			    }
			\end{block}
		\end{column}
	\end{columns}
	\vspace{-0.2cm}
	\begin{columns}[b]
		\begin{column}{0.5\textwidth}
		   	\begin{figure}
				\includegraphics[height=3.0cm]{abolf2-2959507-large.png}
		   	\end{figure}
		\end{column}
		\begin{column}{0.5\textwidth}
		    \vspace{-0.5cm}
		   	\begin{figure}
				\includegraphics[height=3.0cm]{zhou3-3199507-large.png}
		   	\end{figure}
		\end{column}
	\end{columns}
\end{frame}

\begin{frame}
    \frametitle{数据和模型双驱动载具航迹推算方法}
    \vspace{-0.5cm}
	\begin{columns}[t]
		\begin{column}{0.5\textwidth}
		{
		    \tiny
		    \SetTblrInner{rowsep=1pt,colsep=2pt}
			\begin{tblr}{
			  colspec = {cccc},
			  row{4,11} = {bg=gray!50},
			}
				\toprule
				模型       & 日期 & 期刊/会议                          & 数据集            \\
				\midrule
				AbolDeepIO & 2019 & IEEE Trans. Intell. Transp. Syst. & EuRoC             \\
				RINS-W     & 2019 & IROS                              & KAIST             \\
				AI-IMU     & 2020 & IEEE T. Intell. Veh.              & KITTI Odometry    \\
				RAN        & 2021 & IEEE Trans. Veh. Technol.         & Private           \\
				RNN        & 2021 & ICRA                              & EuRoC \& KAIST    \\ 
				TinyOdom   & 2022 & IMWUT                             & Public \& Private \\
				RNN-IEKF   & 2022 & IROS                              & KITTI Odometry    \\
				OdoNet     & 2022 & IEEE Sens. J.                     & Private           \\
				SdoNet     & 2023 & IEEE Internet Things J.           & KITTI Odometry    \\
				DeepOdo    & 2023 & IEEE Trans. Instrum. Meas.        & Private           \\ 
				IEKF+CNN   & 2023 & IEEE Trans. Ind. Electron.        & KITTI \& Private  \\
				KF+VHRNet  & 2025 & IEEE Sens. J.                     & Private           \\
				\bottomrule
			\end{tblr}   
		}
		\end{column}   
		\begin{column}{0.5\textwidth}
			\begin{block}{数据驱动模型估计精度}
			    {
			        \footnotesize
					\begin{itemize}
						\item 数据驱动自适应调节滤波器参数
						\item 数据驱动前向速度观测量
					\end{itemize}			    
			    }
			\end{block}
			\begin{block}{数据驱动模型计算量}
			    {
			        \footnotesize
					\begin{itemize}
						\item 应用轻量级网络减小计算量
					\end{itemize}
				}
			\end{block}
		\end{column}
	\end{columns}
	\vspace{-0.2cm}
	\begin{columns}[b]
		\begin{column}{0.5\textwidth}
		   	\begin{figure}
				\includegraphics[height=2cm]{wu2-3301531-large.png}
		   	\end{figure}
		\end{column}
		\begin{column}{0.5\textwidth}
		    \vspace{-0.5cm}
		   	\begin{figure}
				\includegraphics[height=2.5cm]{weng1-3240227-large.png}
		   	\end{figure}
		\end{column}
	\end{columns}
\end{frame}
