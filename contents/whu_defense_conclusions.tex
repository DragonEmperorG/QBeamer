% !TeX encoding = UTF-8
% !TeX root = ../whu-defense-qianlong.tex

%% ------------------------------------------------------------------------
%% Copyright (C) 2021-2023 SJTUG
%% 
%% SJTUBeamer Example Document by SJTUG
%% 
%% SJTUBeamer Example Document is licensed under a
%% Creative Commons Attribution-NonCommercial-ShareAlike 4.0 International License.
%% 
%% You should have received a copy of the license along with this
%% work. If not, see <http://creativecommons.org/licenses/by-nc-sa/4.0/>.
%% -----------------------------------------------------------------------

\section{总结与展望}

\begin{frame}
	\frametitle{全文工作总结}	
		\begin{block}{提出了数据驱动航向和步速约束的行人航迹推算方法}
		{   
		    \footnotesize
		    新增姿态状态量的关联下实现了航向和步速观测量约束的滤波更新。
		    通过RoNIN 公开数据集验证了本文方法的有效性,尤其当智能手机握持姿态保持较为固定时,本文方法可以取得更高的航迹推算精度。
		} 
		\end{block}
		
		\begin{block}{提出了数据驱动姿态和速度约束的车载航迹推算方法}
		{
		    \footnotesize
		    在自采数据集和 KITTI 公开数据集验证了本文方法的有效性,数据驱动方法对于低成本传感器和低速运动状态精度提升更为显著。
		} 
		\end{block}
		
		\begin{block}{提出了数据和模型双驱动的多源融合定位方法}
		{
		    \footnotesize
		    在室内大型铁路枢纽场景自采数据集和谷歌智能手机分米级挑战公开数据集验证了本文方法的有效性。
		    实验验证不变扩展卡尔曼滤波相比经典扩展卡尔曼滤波方法的优越性,数据和模型双驱动定位方法 可以在降低绝对观测量频率的条件下维持相当的定位精度。
		} 
		\end{block}
\end{frame}


\begin{frame}
	\frametitle{未来工作展望}
		\begin{block}{进一步提升数据驱动方法的泛化能力}
		{   
		    \footnotesize
		    研究大规模数据集构建方法以及评价体系。
		} 
		\end{block}
		
		\begin{block}{进一步改进数据和模型双驱动方法的模型架构}
		{
		    \footnotesize
		    在深度学习网络模型中融合不变性,将物理限制融入到网络模型的构建中
		} 
		\end{block}
		
		\begin{block}{进一步丰富模型驱动方法的感知维度}
		{
		    \footnotesize
		    研究在有限计算资源下融合智能手机全量传感器数据,实现 更高可用和高精度室内外无缝定位。
		} 
		\end{block}
\end{frame}
