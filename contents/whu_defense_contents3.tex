% !TeX encoding = UTF-8
% !TeX root = ../whu-defense-qianlong.tex

%% ------------------------------------------------------------------------
%% Copyright (C) 2021-2023 SJTUG
%% 
%% SJTUBeamer Example Document by SJTUG
%% 
%% SJTUBeamer Example Document is licensed under a
%% Creative Commons Attribution-NonCommercial-ShareAlike 4.0 International License.
%% 
%% You should have received a copy of the license along with this
%% work. If not, see <http://creativecommons.org/licenses/by-nc-sa/4.0/>.
%% -----------------------------------------------------------------------

\subsection{研究内容3 数据和模型双驱动定位方法研究}

\begin{frame}[t]
	\frametitle{数据和模型双驱动定位方法研究}	
	\begin{columns}[t]
		\begin{column}{0.5\textwidth}
		    \begin{block}{存在问题}
		    {
		    	\small
		        \begin{itemize}
					\item 单一模型驱动方法定位频率受限
		        \end{itemize}
    		 } 
			\end{block}
			\begin{block}{研究思路}
			 {
 		    	\small
				\begin{itemize}
					\item 数据和模型双驱动保持定位精度和鲁棒性
				\end{itemize}
			}
			\end{block}
		\end{column}   
		\begin{column}{0.5\textwidth}
		   	\begin{figure}
		   	\centering
		   	    \includegraphics[width=\textwidth]{AudioPositioningSystem.pdf}
		   	\end{figure}
		\end{column}
	\end{columns}
\end{frame}

\begin{frame}[t]
	\frametitle{数据和模型双驱动定位方法研究}
	\framesubtitle{模型驱动几何观测量}	
	\begin{columns}[t]
		\begin{column}{0.5\textwidth}
		    \vspace{-1.0cm}
			\begin{block}{模型驱动音频 TDoA 观测量}
			{
				\tiny
				\begin{equation*}
					\Delta t_{ji}\left(t_{n+1}\right)
					=
					f_{\mathrm{Acoustic}}\left(\tilde{\vb*{s}}_{\mathrm{Sound}},\tilde{\vb*{s}}_{\mathrm{BLE}}\left(t_{n+1}\right)\right)
				\end{equation*}
			}
				观测方程
			{
				\tiny	
				\begin{equation*}
					\vb*{H}^{\Delta r_{ji}} 
					= 
					\begin{pmatrix}
						\vb*{H}^{\Delta r_{ji}}_{\vb*{R}_s^w}
						~\vb*{H}^{\Delta r_{ji}}_{\vb*{v}_s^w}
						~\vb*{H}^{\Delta r_{ji}}_{\vb*{p}_s^w}
						~\vb*{H}^{\Delta r_{ji}}_{\vb*{\delta}_{\vb*{\omega}_s^s}} 
						~\vb*{H}^{\Delta r_{ji}}_{\vb*{\delta}_{\vb*{f}_s^s}} 
					\end{pmatrix}
				\end{equation*}		
		   	}
		   	\end{block}
		   	\vspace{1.5cm}
		   	{
				\tiny
				\begin{center}
					\setlength{\tabcolsep}{2pt}
					\begin{tabular*}{\linewidth}{@{\extracolsep{\fill}}lll}
						$ \vb*{H}^{\Delta r_{ji}} _{\vb*{R}_s^w}
						=
							       			\left(\frac{\vb*{P}_j-\hat{\vb*{p}}}{\left|\vb*{P}_j-\hat{\vb*{p}}\right|} - \frac{\vb*{P}_i-\hat{\vb*{p}}}{\left|\vb*{P}_i-\hat{\vb*{p}}\right|}\right)^{\mathrm{T}} \left(\hat{\vb*{p}}\right)_{\times}  $ 
						& $ \vb*{H}^{\Delta r_{ji}} _{\vb*{v}_s^w} = \vb*{0}_{1 \times 3} $ 
						& $ \vb*{H}^{\Delta r_{ji}} _{\vb*{p}_s^w}
							       			=
							       			-\left(\frac{\vb*{P}_i-\hat{\vb*{p}}}{\left|\vb*{P}_i-\hat{\vb*{p}}\right|}-\frac{\vb*{P}_j-\hat{\vb*{p}}}{\left|\vb*{P}_j-\hat{\vb*{p}}\right|}\right)^{\mathrm{T}} $
						\\
						$ \vb*{H}^{\Delta r_{ji}} _{\vb*{\delta}_{\vb*{\omega}_s^s}} = \vb*{0}_{1 \times 3} $ 
						& $ \vb*{H}^{\Delta r_{ji}}_{\vb*{\delta}_{\vb*{f}_s^s}} = \vb*{0}_{1 \times 3} $ 
						& 
			   		\end{tabular*}
		   		\end{center} 
			}
		\end{column}   
		\begin{column}{0.5\textwidth}
		    \vspace{-1.5cm}
		   	\begin{figure}
		   	\centering
		   	    \includegraphics[width=\textwidth]{PedestrianAcousticPositioningAlgorithmArchitecture.pdf}
		   	\end{figure}
		\end{column}
	\end{columns} 
\end{frame}
 
\begin{frame} 
 	\frametitle{数据和模型双驱动定位方法研究}
 	\framesubtitle{南京南高铁站数据集}
	\begin{columns}[t]
		\begin{column}{0.5\textwidth}
		   	\begin{figure}
			    \includegraphics[width=\textwidth]{NanjingNJNRSDatasetTrackOverview.png}
		   	\end{figure}  
		\end{column}   
		\begin{column}{0.5\textwidth}
		   	\begin{figure}
		   	\centering
		   	    \includegraphics[width=\textwidth]{NanjingNJNRSDatasetAcousticSignalZoomedMap.png}
		   	\end{figure}
		\end{column}
	\end{columns}
\end{frame}
 
\begin{frame} 
 	\frametitle{数据和模型双驱动定位方法研究}
 	\framesubtitle{南京南高铁站数据集不同定位方法评价结果}
	\begin{columns}[t]
		\begin{column}{0.5\textwidth}
	    {
	    	\tiny
	    	\setlength{\tabcolsep}{2pt}
			\begin{tabular*}{\linewidth}{@{\extracolsep{\fill}}llrrrrrr}
				\toprule
	            \multicolumn{2}{c}{轨迹} & L1 & L2 & Z1 & Z2 & Z3 & Dataset \\
				\midrule
				\multicolumn{2}{c}{参考里程$\left(\unit{m}\right)$} & 737 & 737 & 593 & 590 & 590 & 3247 \\
				\midrule
				\multicolumn{2}{c}{参考时长$\left(\unit{s}\right)$} & 566 & 548 & 508 & 486 & 461 & 2569 \\
				\midrule
				\multirow{1}{*}{\makecell[c]{MD (TDoA) LS}}
				& $\vb*{e}^{\mathrm{APE,horizontal}}_{MAE}$ & 0.422 & 0.284 & 0.673 & 1.149 & 0.444 & 0.584 \\
				\midrule
				\multirow{1}{*}{\makecell[l]{MD (TDoA) Classic EKF}}
				& $\vb*{e}^{\mathrm{APE,horizontal}}_{MAE}$ & 0.639 & 0.337 & 1.508 & 1.025 & 0.562 & 0.806 \\
				\midrule
				\multirow{1}{*}{\makecell[l]{MD (TDoA) Invariant EKF}}
				& $\vb*{e}^{\mathrm{APE,horizontal}}_{MAE}$ & 0.318 & 0.213 & 0.574 & 0.869 & 0.348 & 0.456 \\
				\midrule
				\multirow{1}{*}{\makecell[l]{DD ($\vb*{R}$) MD (TDoA) Invariant EKF}}
				& $\vb*{e}^{\mathrm{APE,horizontal}}_{MAE}$ & 0.370 & 0.208 & 0.494 & 0.611 & \textbf{0.312} & 0.395 \\
				\midrule
				\multirow{1}{*}{\makecell[l]{DD ($\vb*{R}$, $\vb*{v}$) MD (TDoA) Invariant EKF}}
				& $\vb*{e}^{\mathrm{APE,horizontal}}_{MAE}$ & \textbf{0.316} & \textbf{0.198} & \textbf{0.433} & \textbf{0.539} & 0.368 & \textbf{0.366} \\
				\bottomrule
			\end{tabular*}	
		}  
		\end{column}   
		\begin{column}{0.5\textwidth}
			\begin{columns}[t]
					\begin{column}{0.5\textwidth}
					   	\begin{figure}
						    \includegraphics[width=\textwidth]{NanjingNJNRSDatasetTrackOverview.png}
					   	\end{figure}  
					\end{column}   
					\begin{column}{0.5\textwidth}
					   	\begin{figure}
					   	\centering
					   	    \includegraphics[width=\textwidth]{NanjingNJNRSDatasetAcousticSignalZoomedMap.png}
					   	\end{figure}
					\end{column}
				\end{columns}
		\end{column}
	\end{columns}	
\end{frame}
 
\begin{frame}
 	\frametitle{数据和模型双驱动定位方法研究}
 	\framesubtitle{南京南高铁站数据集不同 TDoA 组合频率结果对比}
	{\small
		\setlength{\tabcolsep}{2pt}
		\begin{tabular*}{\linewidth}{@{\extracolsep{\fill}}rlrrrrrr}
			\toprule
		    \multicolumn{2}{c}{轨迹} & L1 & L2 & Z1 & Z2 & Z3 & Dataset \\
			\midrule
			\multicolumn{2}{c}{参考里程$\left(\unit{m}\right)$} & 737 & 737 & 593 & 590 & 590 & 3247 \\
			\midrule
			\multicolumn{2}{c}{参考时长$\left(\unit{s}\right)$} & 566 & 548 & 508 & 486 & 461 & 2569 \\
			\midrule
			\multirow{1}{*}{$\SI{1}{\hertz}$}
			& $\vb*{e}^{\mathrm{APE,horizontal}}_{MAE}$ & \textbf{0.316} & \textbf{0.198} & \textbf{0.433} & \textbf{0.539} & \textbf{0.368} & \textbf{0.366} \\
			\midrule
			\multirow{1}{*}{$\SI{0.5}{\hertz}$}
			& $\vb*{e}^{\mathrm{APE,horizontal}}_{MAE}$ & 0.351 & 0.227 & 0.518 & 0.733 & 0.464 & 0.450 \\
			\midrule
			\multirow{1}{*}{$\SI{0.3}{\hertz}$}
			& $\vb*{e}^{\mathrm{APE,horizontal}}_{MAE}$ & 0.374 & 0.263 & 0.690 & 0.674 & 0.519 & 0.495 \\
			\midrule
			\multirow{1}{*}{$\SI{0.2}{\hertz}$}
			& $\vb*{e}^{\mathrm{APE,horizontal}}_{MAE}$ & 0.438 & 0.330 & 0.740 & 0.916 & 0.723 & 0.616 \\		
			\bottomrule 
		\end{tabular*}
	}
\end{frame}

\begin{frame}
 	\frametitle{数据和模型双驱动定位方法研究}
 	\framesubtitle{南京南高铁站数据集不同智能手机结果对比}
	{\small
		\setlength{\tabcolsep}{2pt}
		\begin{tabular*}{\linewidth}{@{\extracolsep{\fill}}llrrrrrr}
			\toprule
			\multicolumn{2}{c}{轨迹} & L1 & L2 & Z1 & Z2 & Z3 & Dataset \\
			\midrule
			\multicolumn{2}{c}{参考里程$\left(\unit{m}\right)$} & 737 & 737 & 593 & 590 & 590 & 3247 \\
			\midrule
			\multicolumn{2}{c}{参考时长$\left(\unit{s}\right)$} & 566 & 548 & 508 & 486 & 461 & 2569 \\
			\midrule
			\multirow{1}{*}{\makecell[l]{HUAWEI Mate 30 LS}}
			& $\vb*{e}^{\mathrm{APE,horizontal}}_{MAE}$ & 0.675 & 0.607 & 2.586 & 1.397 & 1.365 & 1.276 \\
			\midrule
			\multirow{1}{*}{\makecell[l]{HUAWEI Mate 30 Invariant EKF}}
			& $\vb*{e}^{\mathrm{APE,horizontal}}_{MAE}$ & \textbf{0.184} & 0.199 & 0.865 & 0.611 & 0.485 & 0.447 \\
			\midrule
			\multirow{1}{*}{\makecell[l]{HUAWEI Nova 7 LS}}
			& $\vb*{e}^{\mathrm{APE,horizontal}}_{MAE}$ & 0.767 & 0.771 & 3.492 & 1.448 & 1.096 & 1.467 \\
			\midrule
			\multirow{1}{*}{\makecell[l]{HUAWEI Nova 7 Invariant EKF}}
			& $\vb*{e}^{\mathrm{APE,horizontal}}_{MAE}$ & 0.220 & 0.206 & 0.694 & \textbf{0.495} & 0.516 & 0.409 \\
			\midrule
			\multirow{1}{*}{\makecell[l]{SAMSUNG Galaxy S9 LS}}
			& $\vb*{e}^{\mathrm{APE,horizontal}}_{MAE}$ & 0.422 & 0.284 & 0.673 & 1.149 & 0.444 & 0.584 \\
			\midrule
			\multirow{1}{*}{\makecell[l]{SAMSUNG Galaxy S9 Invariant EKF}}
			& $\vb*{e}^{\mathrm{APE,horizontal}}_{MAE}$ & 0.316 & \textbf{0.198} & \textbf{0.433} & 0.539 & \textbf{0.368} & \textbf{0.366} \\
			\bottomrule 
		\end{tabular*} 
	}
\end{frame}

\begin{frame}
	\frametitle{数据和模型双驱动定位方法研究}
   	\begin{figure}
   	\centering
   	    \includegraphics[width=0.6\textwidth]{VehicleGnssPositioningAlgorithmArchitecture.pdf}
   	\end{figure} 
\end{frame}

\begin{frame}
	\frametitle{数据和模型双驱动定位方法研究}	
	\framesubtitle{量测更新}	
	{\small
		模型驱动观测量
		
		\begin{equation*}
			\vb*{p}_{s}^{n}\left(t_{n+1}\right)
			=
			f_{\mathrm{GNSS}}\left(\tilde{\vb*{s}}_{\mathrm{GNSS}}\left(t_i\right)\right)
		\end{equation*}
				
		观测方程
		
		\begin{equation*}
			\vb*{H}^{\vb*{p}^s_{n}} 
			= 
			\begin{pmatrix}
				\left(\hat{\vb*{p}}\right)_{\times}
				&\vb*{0}_3
				&\vb*{I}_3
				&\vb*{0}_3
				&\vb*{0}_3
				&\vb*{0}_3
				&\vb*{0}_3
			\end{pmatrix}
		\end{equation*}    
	} 
\end{frame}

 \begin{frame} 
 	\frametitle{数据和模型双驱动定位方法研究}
 	\framesubtitle{GSDC2023-2024 数据集训练集轨迹图}
	\begin{columns}
		\begin{column}{0.5\textwidth}
		   	\begin{figure}
			    \includegraphics[width=\textwidth]{GSDC2023DatasetSanFranciscoAreaTrackOverview.png}
			    \caption{旧金山地区}
		   	\end{figure} 
		\end{column}   
		\begin{column}{0.5\textwidth}
		   	\begin{figure}
			    \includegraphics[width=\textwidth]{GSDC2023DatasetLosAngelesAreaTrackOverview.png}
			    \caption{洛杉矶地区}
		   	\end{figure} 
		\end{column}
	\end{columns} 
\end{frame}
 
 \begin{frame} 
 	\frametitle{数据和模型双驱动定位方法研究}
 	\framesubtitle{GSDC2023-2024 数据集结果对比}
    {\small
        \setlength{\tabcolsep}{2pt}
		\begin{tabular*}{\linewidth}{@{\extracolsep{\fill}}lcccl rrr rrr}
			\toprule
			\multirow{2.5}{*}{轨迹采集名称} 
			& \multirow{2.5}{*}{编号} 
			& \multirow{2.5}{*}{\makecell[c]{里程\\$\left(\unit{\kilo\meter}\right)$}} 
			& \multirow{2.5}{*}{\makecell[c]{时长\\$\left(\unit{\second}\right)$}} 
			& \multirow{2.5}{*}{\makecell[c]{手机型号}} 
			& \multicolumn{3}{c}{\makecell[c]{RTKLIB\\$\vb*{e}^{\mathrm{APE,horizontal}}$}}
			& \multicolumn{3}{c}{\makecell[c]{RTKLIB Invariant EKF\\$\vb*{e}^{\mathrm{APE,horizontal}}$}} \\
			\cmidrule{6-8} \cmidrule{9-11}
			& & & & & $\SI{50}{\percent}$  & $\SI{95}{\percent}$  & score & $\SI{50}{\percent}$  & $\SI{95}{\percent}$  & score \\
			\midrule
			\multirow{3}{*}{\makecell[c]{2023-05-19-20-10\\-us-ca-mtv-ie2}} & \multirow{3}{*}{\makecell[c]{46}} & \multirow{3}{*}{\makecell[r]{5.9}} & \multirow{3}{*}{\makecell[r]{1229}} & Pixel 5 & 1.079 & 2.572 & 1.826 & 1.080 & 2.318 & \textbf{1.699} \\
			& & & & Pixel 7 Pro & 1.238 & 2.503 & 1.871 & 1.209 & 2.485 & \textbf{1.847} \\
			& & & & Galaxy S22 Ultra & 1.202 & 2.942 & \textbf{2.072} & 1.204 & 2.989 & 2.097 \\
			\midrule
			\multirow{3}{*}{\makecell[c]{2023-05-25-20-11\\-us-ca-sjc-he2}} & \multirow{3}{*}{\makecell[c]{51}} & \multirow{3}{*}{\makecell[r]{10.7}} & \multirow{3}{*}{\makecell[r]{1263}} & Pixel 5 & 0.942 & 2.321 & 1.632 & 0.925 & 2.147 & \textbf{1.536} \\
			& & & & Pixel 7 Pro & 0.668 & 2.005 & 1.336 & 0.669 & 1.783 & \textbf{1.226} \\
			& & & & Galaxy S22 Ultra & 0.660 & 1.463 & 1.062 & 0.592 & 1.360 & \textbf{0.976} \\
			\midrule
			\multirow{2}{*}{\makecell[c]{2023-05-26-18-51\\-us-ca-sjc-ge2}} & \multirow{2}{*}{\makecell[c]{53}} & \multirow{2}{*}{\makecell[r]{7.4}} & \multirow{2}{*}{\makecell[r]{1461}} & Pixel 5 & 1.309 & 2.320 & 1.814 & 1.265 & 2.252 & \textbf{1.759} \\
			& & & & Pixel 7 Pro & 1.231 & 2.716 & 1.974 & 1.198 & 2.427 & \textbf{1.813} \\
			\bottomrule
		\end{tabular*}	
	}	
\end{frame}

\begin{frame}
	\frametitle{数据和模型双驱动定位方法研究}
	\framesubtitle{2023-05-25-20-11-us-ca-sjc-he2/Galaxy S22 Ultra 轨迹 \SI{631}{\second}不同方法结果对比}
   	\begin{figure}
   	\centering
   	    \includegraphics[width=0.6\textwidth]{2023-05-25-20-11-us-ca-sjc-he2/sm-s908b/ComparedSubTraceT631GeographicPlot.png}
   	\end{figure} 
\end{frame}

\begin{frame}
	\frametitle{数据和模型双驱动定位方法研究}
	\framesubtitle{2023-05-26-18-51-us-ca-sjc-ge2/Pixel 7 Pro 轨迹 \SI{289}{\second}不同方法结果对比}
   	\begin{figure}
   	\centering
   	    \includegraphics[width=0.6\textwidth]{2023-05-26-18-51-us-ca-sjc-ge2/pixel7pro/ComparedSubTraceT289GeographicPlot.png}
   	\end{figure} 
\end{frame}
