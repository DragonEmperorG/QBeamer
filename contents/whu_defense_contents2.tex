% !TeX encoding = UTF-8
% !TeX root = ../whu-defense-qianlong.tex

%% ------------------------------------------------------------------------
%% Copyright (C) 2021-2023 SJTUG
%% 
%% SJTUBeamer Example Document by SJTUG
%% 
%% SJTUBeamer Example Document is licensed under a
%% Creative Commons Attribution-NonCommercial-ShareAlike 4.0 International License.
%% 
%% You should have received a copy of the license along with this
%% work. If not, see <http://creativecommons.org/licenses/by-nc-sa/4.0/>.
%% -----------------------------------------------------------------------

\subsection{研究内容2 数据驱动车载航迹推算方法研究}

\begin{frame}
	\frametitle{研究内容2 数据驱动车载航迹推算方法研究} 
	\vspace{-1cm}
	\begin{columns}[t]
		\begin{column}{0.5\textwidth}
		    \begin{block}{存在问题}
		    {
		    	\small
		        \begin{itemize}
					\item 车载智能手机航迹推算快速发散
					\item 地下停车场低速度大动态场景
		        \end{itemize}
    		 } 
			\end{block}
			\begin{block}{研究思路}
			 {
 		    	\small
				\begin{itemize}
					\item 改进滤波器建模
					\item 增加动态位姿约束下
				\end{itemize}
			}
			\end{block}
		\end{column}   
		\begin{column}{0.5\textwidth}
		    \begin{block}{状态定义}
      		{
  		    	\small
				\begin{equation*}
				   	\vb*{x}\left(t_n\right) := \left( \vb*{\chi}\left(t_n\right), \vb*{\theta}\left(t_n\right), \vb*{R}_s^v\left(t_n\right), \vb*{p}_s^v\left(t_n\right)\right) 
				\end{equation*}
			}
			\end{block} 
			\begin{figure}
				\centering
				\includegraphics[height=2.2cm]{VehicleDeadReckoningAlgorithmArchitecture.pdf}
			\end{figure}
		\end{column}
	\end{columns} 
	\begin{block}{数据驱动观测量}
		\begin{equation*}
			\vb*{R}_{s}^{n}\left(t_{n+1}\right), v^{v}_{v, \mathrm{lon}}\left(t_{n+1}\right)
			=
			f_{\mathrm{Estimator}}\left(\{\tilde{\vb*{\omega}}\left(t_i\right), \tilde{\vb*{f}}\left(t_i\right)\}_{i = n-W_{\mathrm{est}}}^{n}\right)
		\end{equation*}
	\end{block}
\end{frame}

\begin{frame}
	\frametitle{研究内容2 数据驱动车载航迹推算方法研究}	
	\framesubtitle{时间更新}
	\vspace{-0.4cm}
	\begin{block}{状态量时间更新}
	{
		\vspace{-0.4cm}
		{
			\scriptsize
			\begin{align*}
				\vb*{R}\left(t_{n+1}\right) &= \vb*{R}\left(t_{n+1}\right) + \vb*{R}\left(t_{n}\right) \vb*{\Gamma}_{0}\left( \vb*{\omega}\left(t_{n}\right) \Delta t \right) \\
				\vb*{v}\left(t_{n+1}\right) &= \vb*{v}\left(t_{n}\right) 
				+ \vb*{R}\left(t_{n}\right) \vb*{\Gamma}_{1}\left( \vb*{\omega}\left(t_{n}\right) \Delta t \right) \vb*{f}\left(t_{n}\right) \Delta t 
				+ \vb*{g}\Delta t \\
				\vb*{p}\left(t_{n+1}\right) &= \vb*{p}\left(t_{n}\right) 
				+ \vb*{v}\left(t_{n}\right) \Delta t
				+ \vb*{R}\left(t_{n}\right) \vb*{\Gamma}_{2}\left( \vb*{\omega}\left(t_{n}\right) \Delta t \right) \vb*{f}\left(t_{n}\right) \left(\Delta t\right)^2 
				+ \frac{1}{2}\vb*{g}\left(\Delta t\right)^2 \\
				\vb*{\delta \omega}\left(t_{n+1}\right) &= \vb*{\delta \omega}\left(t_{n}\right) \\
				\vb*{\delta f}\left(t_{n+1}\right) &= \vb*{\delta f}\left(t_{n}\right)
			\end{align*}
		}
		\vspace{-1.2cm}
		\begin{align*}
			\vb*{R}_s^v\left(t_{n+1}\right) &= \vb*{R}_s^v\left(t_{n}\right) \\
			\vb*{p}_s^v\left(t_{n+1}\right) &= \vb*{p}_s^v\left(t_{n}\right)  
		\end{align*}
	}
	\end{block}
	\begin{block}{状态量协方差时间更新}
	{
		\footnotesize 		
		\begin{equation*}
			\vb*{P}\left(t_{n+1}\right) = \vb*{F}\left(t_{n}\right) \vb*{P}\left(t_{n+1}\right) \vb*{F}^{\mathrm{T}}\left(t_{n}\right)
			+ \vb*{G}\left(t_{n}\right) \vb*{Q}\left(t_{n}\right) \vb*{G}^{\mathrm{T}}\left(t_{n}\right)
		\end{equation*}
	}
	\end{block} 
\end{frame}

\begin{frame}
	\frametitle{研究内容2 数据驱动车载航迹推算方法研究}	
	\framesubtitle{量测更新}
	\vspace{-0.8cm}
	\begin{columns}[t]
		\begin{column}{0.5\textwidth}
			\begin{block}{观测方程}					
    			\scalebox{0.6}{
    				{
    					\tiny
	    				\setlength{\tabcolsep}{2pt}
						\begin{tabular*}{\linewidth}{@{\extracolsep{\fill}}lll}
				            $ \tilde{\vb*{y}}_{\vb*{R}_s^n} = \tilde{\vb*{R}}_s^n $ 
				            & $ \hat{\vb*{y}}_{\vb*{R}_s^n} = \hat{\vb*{R}}_s^n $ 
				            & $ \vb*{r}_{\vb*{R}_s^n} = \log_{\mathrm{G}_{\mathrm{SO}\left(3\right)}}\left(\tilde{\vb*{y}}_{\vb*{R}_s^n}\left(\hat{\vb*{y}}_{\vb*{R}_s^n}\right)^{\mathrm{T}}\right) $ 
				            \\
				            $ \tilde{\vb*{y}}_{\vb*{v}_v^v} = 
				            \begin{pmatrix}
				           		0 & \tilde{v}_{\mathrm{lat}}^v & 0
				           	\end{pmatrix}^{\mathrm{T}} $ 
				            & $ \hat{\vb*{y}}_{\vb*{v}_v^v} = \hat{\vb*{R}}_s^v \left( \left(\hat{\vb*{R}}_s^n\right)^{\mathrm{T}} \hat{\vb*{v}}_s^n + \left(\vb*{\omega}^s\right)_{\times} \hat{\vb*{p}}_v^s \right) $ 
				            & $ \vb*{r}_{\vb*{v}_v^v} = \tilde{\vb*{y}}_{\vb*{v}_b^b} - \hat{\vb*{y}}_{\vb*{v}_b^b} $
				   		\end{tabular*}
			   		}
		   		}
		   		\vspace{-0.5cm}
		   		\scalebox{0.8}{\parbox{.5\linewidth}{%
			   		{\tiny
						\begin{align*}
						    \begin{pmatrix}
								\vb*{H}^{\vb*{R}_s^n}  \\
								\vb*{H}^{\vb*{v}^v_v}
							\end{pmatrix} 
							&= 
							\begin{pmatrix}
								\vb*{I}_3 & \vb*{0}_3 & \vb*{0}_3 & \vb*{0}_3 & \vb*{0}_3 & \vb*{0}_3 & \vb*{0}_3 \\
								\vb*{0}_3 & \hat{\vb*{R}}_s^v \left(\hat{\vb*{R}}_s^w\right)^{\mathrm{T}} & \vb*{0}_3 & \hat{\vb*{R}}_s^v \left(\hat{\vb*{p}}_v^s\right)_{\times} & \vb*{0}_3 & \vb*{H}^{\vb*{v}^v_{v}}_{\vb*{R}_s^v} & \hat{\vb*{R}}_s^v \left(\vb*{\omega}^s\right)_{\times}
							\end{pmatrix}
						\end{align*}
					}
		   		}}
			\end{block}
		\end{column}   
		\begin{column}{0.5\textwidth}
			\begin{block}{状态量量测更新}
			\vspace{-0.5cm}
			{
				\tiny	
				\begin{align*}
			    	\vb*{S}\left(t_{n+1}\right) &= \vb*{H}\left(t_{n+1}\right) \vb*{P}\left(t_{n+1}\right) \vb*{H}^{\mathrm{T}}\left(t_{n+1}\right) + \vb*{N}\left(t_{n+1}\right) \\
			    	\vb*{K}\left(t_{n+1}\right) &= \vb*{P}\left(t_{n+1}\right) \vb*{H}^{\mathrm{T}}\left(t_{n+1}\right) \vb*{S}^{-1}\left(t_{n+1}\right) \\
				    \vb*{e}\left(t_{n+1}\right) &= \vb*{H}\left(t_{n+1}\right) \vb*{r}\left(t_{n+1}\right) \\
					\vb*{\chi}\left(t_{n+1}^+\right)   &= \exp_{\mathrm{G}_{\mathrm{SE}_2\left(3\right)}}\left( \vb*{e}_{\vb*{\xi}_{\vb*{\chi}}}\left(t_{n+1}^+\right) \right) \vb*{\chi}\left(t_{n+1}\right)\\
					\vb*{\theta}\left(t_{n+1}^+\right) &= \vb*{\theta}\left(t_{n+1}\right) + \vb*{e}_{\vb*{\xi}_{\vb*{\theta}}}\left(t_{n+1}^+\right)
				\end{align*}
			}
			\vspace{-1.0cm}
			{
				\scriptsize	
				\begin{align*}
					\vb*{R}_s^v\left(t_{n+1}^+\right)  &= \exp_{\mathrm{G}_{\mathrm{SO}\left(3\right)}}\left( \vb*{e}_{\vb*{\xi}_{\vb*{R}_s^v}}\left(t_{n+1}^+\right) \right) \vb*{R}_s^vb\left(t_{n+1}\right) \\
					\vb*{p}_s^v\left(t_{n+1}^+\right)  &= \vb*{p}_s^v\left(t_{n+1}\right) + \vb*{e}_{\vb*{\xi}_{\vb*{p}_s^v}}\left(t_{n+1}^+\right)
				\end{align*}
			}
			\end{block} 
		\end{column}
	\end{columns} 
	\begin{block}{状态量协方差量测更新}
		{\scriptsize%
			\begin{equation*}
				\vb*{P}\left(t_{n+1}^+\right) = \left(\vb*{I} - \vb*{K}\left(t_{n+1}\right)\vb*{H}\left(t_{n+1}\right)\right) \vb*{P}\left(t_{n+1}\right) \left(\vb*{I} - \vb*{K}\left(t_{n+1}\right)\vb*{H}\left(t_{n+1}\right)\right)^{\mathrm{T}} + \vb*{K}\left(t_{n+1}\right)\vb*{N}\left(t_{n+1}\right) \vb*{K}^{\mathrm{T}}\left(t_{n+1}\right)
			\end{equation*}
		}
	\end{block} 	
\end{frame}

\begin{frame}[t] 
 	\frametitle{研究内容2 数据驱动车载航迹推算方法研究}
 	\framesubtitle{数据集}
 	\vspace{-0.4cm}
	\begin{columns}[t]
		\begin{column}{0.5\textwidth}
            重庆停车场数据集
		   	\begin{figure}
    			\includegraphics[width=\textwidth]{ChongqinDatasetOverview.png}
		   	\end{figure}
		   	\vspace{-0.5cm}
		   	\begin{figure}
				\includegraphics[height=1.2cm]{ChongqinDatasetVelocityDistribution.pdf}
		   	\end{figure}   
		\end{column}   
		\begin{column}{0.5\textwidth}
		    KITTI Odometry数据集
		   	\begin{figure}
                \includegraphics[width=\textwidth]{KITTIOdometryOverview.png}
		   	\end{figure}
		   	\vspace{-0.5cm} 
		   	\begin{figure}
				\includegraphics[height=1.2cm]{KITTIOdometryVelocityDistribution.pdf}
		   	\end{figure}  
		\end{column}
	\end{columns}
\end{frame}

\begin{frame}[t] 
 	\frametitle{研究内容2 数据驱动车载航迹推算方法研究}
 	\framesubtitle{数据集构建}
 	\vspace{-0.4cm}
	\begin{columns}[t]
		\begin{column}{0.5\textwidth}
		    \vspace{-0.4cm}
		    数据采集平台
			\begin{columns}[t]
				\begin{column}{0.5\textwidth} 
				   	\begin{figure}
		    			\includegraphics[width=0.9\textwidth]{GroundTruthSystemInstallationVehicleBack.png}
				   	\end{figure}
				   	\vspace{-0.4cm}
				   	\begin{figure}
						\includegraphics[width=0.9\textwidth]{SmartphoneInstallationVehicleBack.jpg}
				   	\end{figure}   
				\end{column}   
				\begin{column}{0.5\textwidth}
				   	\begin{figure}
		    			\includegraphics[width=0.9\textwidth]{GroundTruthSystemInstallationVehicleFront.png}
				   	\end{figure}
				   	\vspace{-0.4cm}
				   	\begin{figure}
						\includegraphics[width=0.9\textwidth]{SmartphoneInstallationVehicleFront.jpg}
				   	\end{figure}  		
				\end{column}
			\end{columns}  
		\end{column}   
		\begin{column}{0.5\textwidth}
		    数据预处理 \hspace{0.2cm}{\footnotesize 对比系统时钟和GNSS时钟}
		   	\begin{figure}
    			\includegraphics[width=0.8\textwidth]{AndroidSystemClockBootTimeStabilityGOOGLEPixel3.png}
		   	\end{figure}
		   	\vspace{-0.5cm}
		   	\begin{figure}
				\includegraphics[width=0.8\textwidth]{AndroidGnssClockBootTimeStabilityGOOGLEPixel3.png}
		   	\end{figure}
		   	\vspace{-0.5cm}
		   	\begin{figure}
				\includegraphics[width=0.8\textwidth]{AndroidSystemClockMotionSensorGyroscopeUncalibratedSampleInvervalStabilityGOOGLEPixel3.png}
		   	\end{figure} 
		\end{column}
	\end{columns}
\end{frame}

\begin{frame} 
 	\frametitle{研究内容2 数据驱动车载航迹推算方法研究}
 	\framesubtitle{基于重庆数据集对比输入窗口长度因素姿态估计结果}
	\vspace{-0.2cm}
	\begin{columns}[t]
		\begin{column}{0.6\textwidth}
			{   
				\tiny   
				\setlength{\tabcolsep}{2pt}     
				\begin{tabular*}{\linewidth}{@{\extracolsep{\fill}}ccrrrrrrrrrrr}
					\toprule
					\multicolumn{2}{c}{轨迹} & 08 & 09 & 10 & 11 & 12 & 13 & 14 & 15 & 16 & 17 & 18 \\
					\midrule
					\multirow{4}{*}{窗口长度\SI{1}{\second}} 
					& $\vb*{e}^{\mathrm{AOE}}_{MAE}$ & 7.6 & \textbf{1.6} & 9.0 & \textbf{7.6} & \textbf{2.4} & \textbf{6.7} & \textbf{4.2} & \textbf{10.1} & \textbf{8.9} & \textbf{4.6} & \textbf{20.0} \\
					& $\vb*{e}^{\mathrm{AOE}}_{\mathrm{RMSE}}$ & 8.0 & \textbf{1.8} & 10.1 & \textbf{8.3} & \textbf{2.6} & \textbf{7.4} & \textbf{5.0} & \textbf{10.9} & \textbf{11.1} & \textbf{5.4} & \textbf{22.9} \\
					& $\vb*{e}^{\mathrm{ROE}}_{MAE}$ & 12.4 & \textbf{2.5} & 17.3 & \textbf{14.5} & \textbf{3.7} & \textbf{13.1} & \textbf{6.8} & \textbf{20.0} & \textbf{16.8} & \textbf{7.8} & \textbf{39.5} \\
					& $\vb*{e}^{\mathrm{ROE}}_{\mathrm{RMSE}}$ & 12.5 & \textbf{2.8} & 18.6 & \textbf{15.2} & \textbf{3.9} & \textbf{14.1} & \textbf{7.6} & \textbf{20.9} & 20.2 & \textbf{9.2} & \textbf{44.4} \\ \addlinespace[1mm]
					\multirow{4}{*}{窗口长度\SI{2}{\second}}
					& $\vb*{e}^{\mathrm{AOE}}_{MAE}$ & \textbf{4.1} & 3.8 & \textbf{6.1} & 8.7 & 25.5 & 17.5 & 11.3 & 71.1 & 10.3 & 12.0 & 33.5 \\
					& $\vb*{e}^{\mathrm{AOE}}_{\mathrm{RMSE}}$ & \textbf{5.1} & 4.5 & \textbf{6.9} & 10.0 & 31.1 & 19.8 & 13.0 & 85.9 & 11.6 & 13.9 & 38.5 \\
					& $\vb*{e}^{\mathrm{ROE}}_{MAE}$ & \textbf{4.0} & 6.4 & \textbf{11.8} & 15.2 & 48.4 & 33.7 & 22.1 & 99.4 & 16.9 & 21.7 & 65.6 \\
					& $\vb*{e}^{\mathrm{ROE}}_{\mathrm{RMSE}}$ & \textbf{4.8} & 6.9 & \textbf{12.6} & 17.1 & 55.5 & 36.2 & 24.7 & 112.6 & \textbf{19.7} & 25.2 & 74.5 \\
					\bottomrule 
				\end{tabular*}    
		   	}
			\begin{itemize}
				\item 增加窗口长度并不能提升姿态估计精度
			\end{itemize}
		\end{column} 
		\begin{column}{0.4\textwidth}
    		\vspace{-1.0cm}  
		   	\begin{figure}
    			\includegraphics[width=0.9\textwidth]{CQDatasetTtack08DataDrivenOrientationInputWindowFactorHeading.png}
		   	\end{figure}
		   	\vspace{-0.5cm}
		   	\begin{figure}
				\includegraphics[width=0.9\textwidth]{CQDatasetTtack08DataDrivenOrientationInputWindowFactorPitch.png}
		   	\end{figure}
		   	\vspace{-0.5cm}
		   	\begin{figure}
				\includegraphics[width=0.9\textwidth]{CQDatasetTtack08DataDrivenOrientationInputWindowFactorRoll.png}
		   	\end{figure} 
		   	\vspace{-0.4cm}
		   	{\footnotesize 重庆数据集08轨迹姿态结果对比}
		\end{column}    
	\end{columns}
\end{frame}

\begin{frame} 
 	\frametitle{研究内容2 数据驱动车载航迹推算方法研究}
 	\framesubtitle{基于重庆数据集对比输出状态量因素速度估计结果}
	\vspace{-0.2cm}
	\begin{columns}[t]
		\begin{column}{0.6\textwidth}
			{   
				\tiny   
				\setlength{\tabcolsep}{2pt}     
				\begin{tabular*}{1.1\linewidth}{@{\extracolsep{\fill}}clccccccccccc}
					\toprule
					\multicolumn{2}{c}{轨迹} & 08 & 09 & 10 & 11 & 12 & 13 & 14 & 15 & 16 & 17 & 18 \\
					\midrule
					\multirow{4}{*}{$\vb*{v}_y$} 
					& $\vb*{e}^{\mathrm{AVE}}_{MAE}$ 
					& 0.410 & 0.401 & 0.336 & 0.434 & \textbf{0.282} & \textbf{0.333} & \textbf{0.266} & \textbf{0.297} & \textbf{0.510} & \textbf{0.274} & \textbf{0.426} \\         
					& $\vb*{e}^{\mathrm{AVE}}_{\mathrm{RMSE}}$          
					& \textbf{0.491} & 0.492 & 0.424 & 0.533 & \textbf{0.358} & \textbf{0.415} & \textbf{0.357} & \textbf{0.374} & \textbf{0.679} & \textbf{0.346} & \textbf{0.527} \\ 
					& $\vb*{e}^{\mathrm{RVE}}_{MAE, \Delta t}$ 
					& \textbf{0.419} & 0.584 & 0.407 & 0.644 & \textbf{0.376} & 0.382 & \textbf{0.393} & \textbf{0.414} & \textbf{0.607} & \textbf{0.401} & \textbf{0.509} \\
					& $\vb*{e}^{\mathrm{RVE}}_{RMSE,\Delta t}$ 
					& \textbf{0.510} & 0.753 & 0.487 & 0.824 & \textbf{0.463} & 0.531 & \textbf{0.522} & \textbf{0.515} & \textbf{0.777} & \textbf{0.507} & \textbf{0.646} \\ \addlinespace[1mm]
					\multirow{4}{*}{$\Delta\vb*{v}_y$} 
					& $\vb*{e}^{\mathrm{AVE}}_{MAE}$ 
					& 1.884 & 0.689 & \textbf{0.241} & 1.451 & 0.481 & 0.605 & 0.942 & 1.339 & 1.573 & 1.374 & 1.326 \\         
					& $\vb*{e}^{\mathrm{AVE}}_{\mathrm{RMSE}}$         
					& 2.155 & 0.883 & 0.406 & 1.596 & 0.680 & 0.673 & 1.065 & 1.427 & 1.848 & 1.519 & 1.646 \\
					& $\vb*{e}^{\mathrm{RVE}}_{MAE, \Delta t}$ 
					& 0.927 & \textbf{0.526} & \textbf{0.227} & 1.259 & 0.503 & \textbf{0.216} & 0.707 & 0.770 & 1.243 & 2.197 & 2.145 \\
					& $\vb*{e}^{\mathrm{RVE}}_{RMSE,\Delta t}$ 
					& 1.326 & 0.847 & \textbf{0.390} & 1.676 & 0.760 & \textbf{0.344} & 0.959 & 1.036 & 1.594 & 2.528 & 2.475 \\
					\bottomrule 
				\end{tabular*}   
		   	}
			\begin{itemize}
				\item 基于 RNN 架构直接估计速度精度更高
			\end{itemize}
		\end{column} 
		\begin{column}{0.4\textwidth}
    		\vspace{-1.4cm}  
		   	\begin{figure}
    			\includegraphics[width=0.8\textwidth]{CQDatasetTrack08DataDrivenVelocityOutputFactor.png}
		   	\end{figure}
		   	\vspace{-0.5cm}
		   	\hspace{0.5cm} {\tiny 重庆数据集 08 轨迹速度估计结果}
		   	\vspace{-0.2cm}
		   	\begin{figure}
				\includegraphics[width=0.8\textwidth]{CQDatasetTrack11DataDrivenVelocityOutputFactor.png}
		   	\end{figure}
		   	\vspace{-0.5cm}
		   	\hspace{0.5cm} {\tiny 重庆数据集 11 轨迹速度估计结果}
		\end{column}    
	\end{columns}	
 \end{frame}

\begin{frame} 
 	\frametitle{研究内容2 数据驱动车载航迹推算方法研究}
 	\framesubtitle{基于重庆数据集对比输入维度因素速度估计结果}
	\vspace{-0.2cm}
	\begin{columns}[t]
		\begin{column}{0.6\textwidth}
			\scalebox{0.9}{{   
				\tiny   
				\setlength{\tabcolsep}{2pt}     
				\begin{tabular*}{1.2\linewidth}{@{\extracolsep{\fill}}clccccccccccc}
					\toprule
					\multicolumn{2}{c}{轨迹} & 08 & 09 & 10 & 11 & 12 & 13 & 14 & 15 & 16 & 17 & 18 \\
					\midrule
					\multirow{4}{*}{\makecell{$\left( \vb*{\omega}_s^s, \vb*{f}_s^s, P_s^s \right)$\\$\mathbb{R}^7$}}
					& $\vb*{e}^{\mathrm{AVE}}_{MAE}$ 
					& 0.410 & 0.401 & 0.336 & 0.434 & 0.282 & 0.333 & 0.266 & 0.297 & 0.510 & 0.274 & 0.426 \\         
					& $\vb*{e}^{\mathrm{AVE}}_{\mathrm{RMSE}}$          
					& 0.491 & 0.492 & 0.424 & 0.533 & 0.358 & 0.415 & 0.357 & 0.374 & 0.679 & 0.346 & 0.527 \\ 
					& $\vb*{e}^{\mathrm{RVE}}_{MAE, \Delta t}$ 
					& 0.419 & 0.584 & \textbf{0.407} & 0.644 & 0.376 & 0.382 & 0.393 & 0.414 & \textbf{0.607} & 0.401 & \textbf{0.509} \\
					& $\vb*{e}^{\mathrm{RVE}}_{RMSE,\Delta t}$ 
					& 0.510 & 0.753 & \textbf{0.487} & 0.824 & 0.463 & 0.531 & 0.522 & \textbf{0.515} & \textbf{0.777} & 0.507 & \textbf{0.646} \\ \addlinespace[1mm]
					\multirow{4}{*}{\makecell{$\left( \vb*{\omega}_s^s, \vb*{f}_s^s\right)$\\$\mathbb{R}^6$}} 
					& $\vb*{e}^{\mathrm{AVE}}_{MAE}$ 
					& \textbf{0.208} & \textbf{0.315} & \textbf{0.325} & \textbf{0.381} & \textbf{0.232} & \textbf{0.141} & \textbf{0.241} & \textbf{0.290} & \textbf{0.481} & \textbf{0.245} & \textbf{0.383} \\      
					& $\vb*{e}^{\mathrm{AVE}}_{\mathrm{RMSE}}$         
					& \textbf{0.273} & \textbf{0.377} & \textbf{0.404} & \textbf{0.502} & \textbf{0.294} & \textbf{0.172} & \textbf{0.327} & \textbf{0.367} & \textbf{0.630} & \textbf{0.328} & \textbf{0.508} \\
					& $\vb*{e}^{\mathrm{RVE}}_{MAE, \Delta t}$ 
					& \textbf{0.303} & \textbf{0.483} & 0.439 & \textbf{0.581} & \textbf{0.336} & \textbf{0.121} & \textbf{0.361} & \textbf{0.413} & 0.623 & \textbf{0.353} & 0.540 \\
					& $\vb*{e}^{\mathrm{RVE}}_{RMSE,\Delta t}$ 
					& \textbf{0.369} & \textbf{0.598} & 0.540 & \textbf{0.772} & \textbf{0.418} & \textbf{0.154} & \textbf{0.467} & 0.522 & 0.809 & \textbf{0.449} & 0.713 \\
					\bottomrule 
				\end{tabular*}  
		   	}}
			\begin{itemize}
				\item 小范围数据集气压计传感器的增加对于数据驱动速度精度提升不显著
			\end{itemize}
		\end{column} 
		\begin{column}{0.4\textwidth}
    		\vspace{-2.2cm}  
		   	\begin{figure}
    			\includegraphics[width=0.8\textwidth]{CQDatasetTrack08DataDrivenVelocityOutputSensorsFactor.png}
		   	\end{figure}
		   	\vspace{-0.5cm}
		   	\hspace{0.5cm} {\tiny 重庆数据集 08 轨迹速度估计结果}
		   	\vspace{-0.2cm}
		   	\begin{figure}
				\includegraphics[width=0.8\textwidth]{CQDatasetTrack11DataDrivenVelocityOutputSensorsFactor.png}
		   	\end{figure}
		   	\vspace{-0.5cm}
		   	\hspace{0.5cm} {\tiny 重庆数据集 11 轨迹速度估计结果}
		\end{column}    
	\end{columns}	
\end{frame}

\begin{frame} 
 	\frametitle{研究内容2 数据驱动车载航迹推算方法研究}
 	\framesubtitle{基于重庆数据集对比输入采样率因素速度估计结果}
	\vspace{-0.2cm}
	\begin{columns}[t]
		\begin{column}{0.6\textwidth}
			\scalebox{0.9}{{   
				\tiny   
				\setlength{\tabcolsep}{2pt}     
				\begin{tabular*}{\linewidth}{@{\extracolsep{\fill}}clccccccccccc}
					\toprule
					\multicolumn{2}{c}{轨迹} & 08 & 09 & 10 & 11 & 12 & 13 & 14 & 15 & 16 & 17 & 18 \\
					\midrule
					\multirow{2}{*}{\SI{50}{Hz}} 
					& $\vb*{e}^{\mathrm{AVE}}_{MAE}$ 
					& 0.410 & 0.401 & 0.336 & 0.434 & 0.282 & 0.333 & 0.266 & 0.297 & 0.510 & 0.274 & 0.426 \\
					& $\vb*{e}^{\mathrm{RVE}}_{MAE, \Delta t}$ 
					& 0.419 & 0.584 & 0.407 & 0.644 & 0.376 & 0.382 & 0.393 & 0.414 & \textbf{0.607} & 0.401 & \textbf{0.509} \\ \addlinespace[1mm]
					\multirow{2}{*}{\SI{100}{Hz}} 
					& $\vb*{e}^{\mathrm{AVE}}_{MAE}$ 
					& \textbf{0.277} & \textbf{0.308} & \textbf{0.274} & \textbf{0.361} & \textbf{0.206} & \textbf{0.237} & \textbf{0.225} & 0.277 & 0.480 & \textbf{0.229} & 0.411 \\ 
					& $\vb*{e}^{\mathrm{RVE}}_{MAE, \Delta t}$ 
					& 0.411 & \textbf{0.448} & \textbf{0.382} & 0.587 & 0.309 & \textbf{0.302} & \textbf{0.319} & 0.391 & 0.640 & \textbf{0.345} & 0.573 \\ \addlinespace[1mm]
					\multirow{2}{*}{\SI{150}{Hz}} 
					& $\vb*{e}^{\mathrm{AVE}}_{MAE}$ 
					& 0.289 & 0.348 & 0.281 & 0.362 & 0.212 & 0.273 & 0.238 & \textbf{0.252} & \textbf{0.452} & 0.276 & \textbf{0.391} \\
					& $\vb*{e}^{\mathrm{RVE}}_{MAE, \Delta t}$ 
					& 0.412 & 0.485 & 0.403 & \textbf{0.574} & \textbf{0.301} & 0.359 & 0.339 & \textbf{0.351} & 0.625 & 0.395 & 0.493 \\ \addlinespace[1mm]
					\multirow{2}{*}{\SI{200}{Hz}} 
					& $\vb*{e}^{\mathrm{AVE}}_{MAE}$ 
					& 0.285 & 0.311 & 0.322 & 0.466 & 0.222 & 0.284 & 0.261 & 0.261 & 0.457 & 0.276 & 0.394 \\         
					& $\vb*{e}^{\mathrm{RVE}}_{MAE, \Delta t}$ 
					& \textbf{0.401} & 0.488 & 0.474 & 0.673 & 0.312 & 0.374 & 0.358 & 0.370 & 0.647 & 0.395 & 0.527 \\
					\bottomrule 
				\end{tabular*}
		   	}}
			\begin{itemize}
				\item 动态位姿约束可以提升位置估计精度
			\end{itemize}
		\end{column} 
		\begin{column}{0.4\textwidth}
    		\vspace{-3.2cm}  
		   	\begin{figure}
    			\includegraphics[width=0.8\textwidth]{CQDatasetTrack08DataDrivenVelocityInputSampleRateFactor.png}
		   	\end{figure}
		   	\vspace{-0.5cm}
		   	\hspace{0.5cm} {\tiny 重庆数据集 08 轨迹速度估计结果}
		   	\vspace{-0.2cm}
		   	\begin{figure}
				\includegraphics[width=0.8\textwidth]{CQDatasetTrack11DataDrivenVelocityInputSampleRateFactor.png}
		   	\end{figure}
		   	\vspace{-0.5cm}
		   	\hspace{0.5cm} {\tiny 重庆数据集 11 轨迹速度估计结果}
		\end{column}    
	\end{columns}	
\end{frame}

\begin{frame} 
 	\frametitle{研究内容2 数据驱动车载航迹推算方法研究}
 	\framesubtitle{重庆数据集万达广场(大渡口店)停车场航迹推算位置误差结果}
	\vspace{-0.2cm}
	\begin{columns}[t]
		\begin{column}{0.6\textwidth}
			\scalebox{0.9}{{   
				\tiny   
				\setlength{\tabcolsep}{2pt}     
				\begin{tabular*}{1.2\linewidth}{@{\extracolsep{\fill}}clrrrrrrrrrrr}
					\toprule
					\multicolumn{2}{c}{轨迹} & 08 & 09 & 10 & 11 & 12 & 13 & 14 & 15 & 16 & 17 & 18 \\
					\midrule
					\multicolumn{2}{c}{里程$\left(\unit{m}\right)$} & 282  & 285  & 285  & 520  & 525  & 524  & 844  & 1556 & 593   & 839  & 1553   \\
					\multicolumn{2}{c}{时长$\left(\unit{s}\right)$} & 124  & 122  & 208  & 168  & 216  & 370  & 316  & 529  & 459   & 418  & 807    \\
					\multirow{2}{*}{AI-IMU} & $\vb*{e}^{\mathrm{RPE}}_{MAE,unit \Delta l}$ & 15.9 & 24.1 & 65.2 & 11.3 & 26.0 & 1683.8 & 20.8 & 12.6 & 18383.1 & 4829.5 & 176752.2 \\
					& $\vb*{e}^{\mathrm{RPE,horizontal}}_{MAE,unit \Delta l}$ & 15.8 & 24.1 & 65.0 & 11.0 & 25.8 & 1656.0 & 20.7 & 12.4 & 6380.6 & 4098.8 & 36066.2 \\
					\multirow{2}{*}{DeepOdo} & $\vb*{e}^{\mathrm{RPE}}_{MAE,unit \Delta l}$ & 5.0 & 9.8 & 18.4 & 13.3 & 9.0 & 22.3 & 9.4 & 10.1 & 41.1 & 13.9 & 27.4 \\
					& $\vb*{e}^{\mathrm{RPE,horizontal}}_{MAE,unit \Delta l}$ & 4.7 & 9.7 & 18.4 & 13.2 & 9.0 & 22.3 & 9.3 & 10.0 & 41.0 & 13.6 & 27.2 \\
					\multirow{2}{*}{DeepOri} & $\vb*{e}^{\mathrm{RPE}}_{MAE,unit \Delta l}$ & 5.1 & 1.9 & 8.5 & 6.3 & 2.4 & 3.5 & 2.6 & 5.2 & 5.0 & 3.7 & 7.8 \\
					& $\vb*{e}^{\mathrm{RPE,horizontal}}_{MAE,unit \Delta l}$ & 4.5 & 1.4 & 8.4 & 5.2 & 1.7 & 3.4 & 1.4 & 2.7 & 4.2 & 2.2 & 6.0 \\
					\rowcolor{gray!50} \multirow{2}{*}{\textbf{Proposed}} & $\vb*{e}^{\mathrm{RPE}}_{MAE,unit \Delta l}$ & \textbf{2.1} & \textbf{1.2} & \textbf{0.9} & \textbf{2.8} & \textbf{1.4} & \textbf{0.8} & \textbf{1.9} & \textbf{3.3} & \textbf{2.8} & \textbf{2.9} & \textbf{3.4} \\
					\rowcolor{gray!50} & $\vb*{e}^{\mathrm{RPE,horizontal}}_{MAE,unit \Delta l}$ & \textbf{0.4} & \textbf{0.4} & \textbf{0.4} & \textbf{0.4} & \textbf{0.3} & \textbf{0.3} & \textbf{0.3} & \textbf{0.3} & \textbf{0.5} & \textbf{0.6} & \textbf{0.5} \\
					\bottomrule 
				\end{tabular*}
		   	}}
			\begin{itemize}
				\item \SI{100}{\hertz} 采样率可以实现相对高精度的速度估计
			\end{itemize}
		\end{column} 
		\begin{column}{0.4\textwidth}
    		\vspace{-3.2cm}  
		   	\begin{figure}
    			\includegraphics[width=0.8\textwidth]{CQDatasetTrack08ComparedMethodTrackResult.png}
		   	\end{figure}
		   	\vspace{-0.5cm}
		   	\hspace{0.5cm} {\tiny 重庆数据集 08 轨迹位置估计结果}
		   	\vspace{-0.2cm}
		   	\begin{figure}
				\includegraphics[width=0.8\textwidth]{CQDatasetTrack11ComparedMethodTrackResult.png}
		   	\end{figure}
		   	\vspace{-0.5cm}
		   	\hspace{0.5cm} {\tiny 重庆数据集 11 轨迹位置估计结果}
		\end{column}    
	\end{columns}
\end{frame}

\begin{frame} 
 	\frametitle{研究内容2 数据驱动车载航迹推算方法研究}
 	\framesubtitle{KITTI Odometry数据集航迹推算位置误差结果}
	\vspace{-0.2cm}
	\begin{columns}[t]
		\begin{column}{0.6\textwidth}
			{   
				\tiny   
				\setlength{\tabcolsep}{2pt}     
				\begin{tabular*}{\linewidth}{@{\extracolsep{\fill}}ccrrrrrrrrrr}
					\toprule
		            \multicolumn{2}{c}{轨迹} & 00 & 01 & 02 & 04 & 05 & 06 & 07 & 08 & 09 & 10 \\
					\midrule
					\multicolumn{2}{c}{里程$\left(\unit{m}\right)$} & 3722 & 2603 & 5089 & 416 & 2214 & 1245 & 693 & 4245 & 1715 & 922 \\
					\multicolumn{2}{c}{时长$\left(\unit{s}\right)$} &  471 &  121 &  483 &  29 &  288 &  115 & 114 &  537 &  165 & 126 \\
					\multirow{2}{*}{AI-IMU}            & $\vb*{e}^{\mathrm{RPE}}_{MAE,unit \Delta l}$ & 11.8 & 4.0 & 4.5 & 7.0 & 19.1 & 3.5 & 1.8 & 3.2 & 7.4 & 4.0 \\
					                                   & $\vb*{e}^{\mathrm{RPE,horizontal}}_{MAE,unit \Delta l}$ & 11.4 & 3.6 & 2.5 & 6.7 & 18.8 & 2.9 & 1.2 & 1.9 & 5.4 & 2.0 \\
					\multirow{2}{*}{DeepOdo}           & $\vb*{e}^{\mathrm{RPE}}_{MAE,unit \Delta l}$ & 7.4 & 5.0 & 3.9 & 2.6 & 3.7 & 2.3 & 1.7 & 4.8 & 5.2 & 5.0 \\
					                                   & $\vb*{e}^{\mathrm{RPE,horizontal}}_{MAE,unit \Delta l}$ & 6.8 & 4.0 & 1.3 & 1.3 & 3.3 & 1.5 & 0.9 & 3.7 & 2.7 & 3.0 \\
					\multirow{2}{*}{DeepOri}           & $\vb*{e}^{\mathrm{RPE}}_{MAE,unit \Delta l}$ & 4.1 & 4.4 & 4.4 & 9.8 & 2.1 & 3.6 & 1.7 & \textbf{3.0} & 5.2 & 3.8 \\
					                                   & $\vb*{e}^{\mathrm{RPE,horizontal}}_{MAE,unit \Delta l}$ & 3.3 & 3.5 & 2.3 & 9.8 & 1.6 & 3.1 & 0.9 & \textbf{1.7} & 2.8 & 1.8 \\
					\multirow{2}{*}{\textbf{Proposed}} & $\vb*{e}^{\mathrm{RPE}}_{MAE,unit \Delta l}$ & \textbf{2.1} & \textbf{3.2} & \textbf{3.3} & \textbf{2.0} & \textbf{1.3} & \textbf{1.4} & \textbf{1.5} & 8.2 & \textbf{4.1} & \textbf{3.7} \\
					                                   & $\vb*{e}^{\mathrm{RPE,horizontal}}_{MAE,unit \Delta l}$ & \textbf{1.2} & \textbf{1.9} & \textbf{1.0} & \textbf{0.4} & \textbf{0.4} & \textbf{0.5} & \textbf{0.7} & 4.1 & \textbf{1.2} & \textbf{1.1} \\
					\bottomrule 
				\end{tabular*}  
		   	}
			\begin{itemize}
				\item 动态位姿约束可以提升位置估计精度
			\end{itemize}
		\end{column} 
		\begin{column}{0.4\textwidth}
    		\vspace{-2.2cm}  
		   	\begin{figure}
    			\includegraphics[width=0.7\textwidth]{KITTIDatasetTrack00ComparedMethodPositionResult.png}
		   	\end{figure}
		   	\vspace{-0.5cm}
		   	\hspace{0.5cm} {\tiny KITTI 数据集 00 轨迹速度估计结果}
		   	\vspace{-0.2cm}
		   	\begin{figure}
				\includegraphics[width=0.7\textwidth]{KITTIDatasetTrack02ComparedMethodPositionResult.png}
		   	\end{figure}
		   	\vspace{-0.5cm}
		   	\hspace{0.5cm} {\tiny KITTI 数据集 02 轨迹速度估计结果}
		\end{column}    
	\end{columns} 		
\end{frame}

\begin{frame}
	\frametitle{研究内容2 小结}
		\begin{block}{提出了一种数据驱动位姿约束智能手机车载航迹推算方法}
		{   
		    \footnotesize
		    基于 HNN模型构建虚拟里程计和转角传感器,在不变扩展卡尔曼滤波框架下实现长时的位姿估计。
		} 
		\end{block}		
		\begin{block}{在公开 KITTI 数据集和自采重庆数据集上对该方法进行了验证}
		{
		    \footnotesize
		    实验深入研究分析了不同数据驱动观测量组合对位姿估计的影响。数据驱动姿态和速度观测值可以有效约束长时航迹推算误差累积。 
		} 
		\end{block}
\end{frame}
