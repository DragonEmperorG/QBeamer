% Magic Comments
% Encoding
% !TeX encoding = UTF-8
% TeX Root
% !TeX root = ../whu-defense-qianlong.tex

%% ------------------------------------------------------------------------
%% Copyright (C) 2021-2023 SJTUG
%% 
%% SJTUBeamer Example Document by SJTUG
%% 
%% SJTUBeamer Example Document is licensed under a
%% Creative Commons Attribution-NonCommercial-ShareAlike 4.0 International License.
%% 
%% You should have received a copy of the license along with this
%% work. If not, see <http://creativecommons.org/licenses/by-nc-sa/4.0/>.
%% -----------------------------------------------------------------------

\section{个人成果}

\begin{frame}
	% 8.2.6 The Frame Title
	% \frametitle<⟨overlay specification⟩>[⟨short frame title⟩]{⟨frame title text⟩}
	% 页标题
	\frametitle{攻博期间发表的论文}
	% \framesubtitle<⟨overlay specification⟩>{⟨frame subtitle text⟩}
	% 页子标题 
	% \framesubtitle{}

	\begin{enumerate}
		\item 陈锐志, \textbf{钱隆}, 牛晓光, 徐诗豪, 陈亮, 裘超. 2022. 基于数据与模型双驱动的音频/惯性传感器耦合定位方法[J]. 测绘学报, 51(7):1160-1171. (EI)
		
		\item 陈锐志, 郭光毅, 叶锋, \textbf{钱隆}, 徐诗豪, 李正. 2021. 智能手机音频信号与MEMS传感器的紧耦合室内定位方法[J]. 测绘学报, 50(2): 143-152. (EI)
		
		\item \textbf{QIAN L}, LIN X, NIU X, HUANG Q, LI L, GUO G, WANG Z, CHEN R. 2025. AVNet: learning attitude and velocity for vehicular dead reckoning using smartphone by adapting an invariant EKF[J]. Satellite Navigation, 6(1), 15. (SCI 一区)  
		
		\item LI Z, CHEN R, GUO G, YE F, \textbf{QIAN L}, XU S, HUANG L, CHEN L. 2024. Dual-step acoustic chirp signals detection using pervasive smartphones in multipath and NLOS indoor environments[J]. IEEE Internet of Things Journal, 11(4): 6494–6507. (SCI 二区) 

	\end{enumerate}
\end{frame}

\begin{frame}
	% 8.2.6 The Frame Title
	% \frametitle<⟨overlay specification⟩>[⟨short frame title⟩]{⟨frame title text⟩}
	% 页标题
	\frametitle{攻博期间发表的专利}
	% \framesubtitle<⟨overlay specification⟩>{⟨frame subtitle text⟩}
	% 页子标题 
	% \framesubtitle{}

	\begin{enumerate}
		
		\item 陈锐志, \textbf{钱隆}, 牛晓光, 徐诗豪. 基于数据与模型结合的多源融合定位方法、系统及终端:CN202210311275.1[P].2023-03-03.
		
		\item 郭光毅, 陈锐志, \textbf{钱隆}, 李正, 徐诗豪, 叶锋, 刘克强. 一种基于射频增强的广域音频室内定位方法、系统及终端:CN202211222425.8[P].2023-05-02.
  
		\item 陈锐志, 黄李雄, 刘克强, 叶锋, 郭光毅, 徐诗豪, \textbf{钱隆}, 李正, 林欣创. 一种基于音频的定位寻物方法. CN202210794893.6[P].2023-05-02.
		
		\item 郭光毅, 陈锐志, 徐诗豪, \textbf{钱隆}, 李正. 一种基于5G信号和声波信号的电子设备室内定位系统和方法:CN202210022292.3[P].2023-04-07.

	\end{enumerate}
\end{frame}